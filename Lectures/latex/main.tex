\documentclass[11pt]{book}

% Package imports
\usepackage[utf8]{inputenc}
\usepackage[T1]{fontenc}
\usepackage{lmodern}
\usepackage[a4paper, margin=1in]{geometry}
\usepackage{ragged2e}
\usepackage{setspace}
\usepackage{graphicx}
\usepackage{amsmath}
\usepackage{amssymb}
\usepackage{amsfonts}
\usepackage{listings}
\usepackage{xcolor}
\usepackage{hyperref}

% Image path
\graphicspath{{./img/}}

% Hyperref setup
\hypersetup{
    colorlinks=true,
    linkcolor=blue,
    filecolor=magenta,
    urlcolor=cyan,
    pdftitle={CO224 Lecture Notes},
    pdfauthor={Dr. Isuru Nawinne},
}

% Listings setup for code blocks
\lstset{
    basicstyle=\ttfamily\small,
    breaklines=true,
    frame=single,
    backgroundcolor=\color{gray!10}
}

\begin{document}

% Title page
\begin{titlepage}
    \centering
    \vspace*{2cm}
    {\Huge\bfseries CO224 Lecture Notes\\[0.5cm]}
    {\Large Computer Architecture\\[1.5cm]}
    {\large\bfseries Dr. Isuru Nawinne\\[0.3cm]}
    {\large Department of Computer Engineering\\[0.2cm]}
    {\large University of Peradeniya\\[2cm]}
    \vfill
    {\large \today}
\end{titlepage}

% Table of contents
\tableofcontents
\newpage

% Preface
\section*{Preface}
\addcontentsline{toc}{section}{Preface}
This document contains comprehensive lecture notes for CO224 - Computer Architecture course, covering fundamental concepts of computer systems, ARM assembly programming, processor design, memory hierarchy, and multiprocessor systems.

\newpage

% Learning Methods
\section*{Learning Methods}
\addcontentsline{toc}{section}{Learning Methods}
These lecture notes are designed to support active learning. Students are encouraged to:
\begin{itemize}
    \item Read the material before class
    \item Take additional notes during lectures
    \item Practice with the provided examples
    \item Complete assignments and exercises
\end{itemize}

\newpage

% Introduction
\section*{Introduction}
\addcontentsline{toc}{section}{Introduction}
Computer Architecture is the science and art of selecting and interconnecting hardware components to create computers that meet functional, performance, and cost goals. This course explores the design principles and implementation of modern computer systems.

\newpage

% Chapter 1: Fundamentals
\chapter{Fundamentals}

\input{lecture-01}
\input{lecture-02}
\section{Lecture 3: Understanding Performance}

\emph{By Dr. Isuru Nawinne}

\subsection{Introduction}

Understanding computer performance is fundamental to computer architecture and system design. This lecture explores how performance is measured, the factors that influence it, and the principles that guide performance optimization. We examine the metrics used to evaluate systems, the mathematical relationships between performance factors, and Amdahl's Law—a critical principle for understanding the limits of performance improvements.

\subsection{Defining and Measuring Performance}

\subsubsection{Response Time vs. Throughput}

\textbf{Response Time (Execution Time)}

\begin{itemize}
\item Time to complete a single task
\item Includes all overhead and waiting time
\item User-perceived performance metric
\item Example: Time for a program to run from start to finish
\end{itemize}

\textbf{Throughput (Bandwidth)}

\begin{itemize}
\item Number of tasks completed per unit time
\item Measures system capacity
\item Important for servers and data centers
\item Example: Number of transactions processed per second
\end{itemize}

\textbf{Relationship Between Metrics}

\begin{itemize}
\item Improving response time often improves throughput
\item Improving throughput doesn't always improve response time
\item Different optimization strategies for each metric
\item System design must balance both considerations
\end{itemize}

\subsubsection{Performance Definition}

\textbf{Mathematical Definition}

Performance = 1 / Execution Time

\textbf{Performance Comparison}

\begin{itemize}
\item If System A is faster than System B:
\item Execution Time\_A < Execution Time\_B
\item Performance\_A > Performance\_B
\end{itemize}

\textbf{Relative Performance}

Performance\_A / Performance\_B = Execution Time\_B / Execution Time\_A

Example: If System A is 2$\times$ faster than System B:

\begin{itemize}
\item Performance\_A / Performance\_B = 2
\item Execution Time\_B / Execution Time\_A = 2
\item System A takes half the time of System B
\end{itemize}

\subsection{CPU Time and Performance Factors}

\subsubsection{Components of Execution Time}

\textbf{Total Execution Time}

\begin{itemize}
\item CPU time: Time CPU spends computing the task
\item I/O time: Time waiting for input/output operations
\item Other system activities: OS overhead, other programs
\end{itemize}

\textbf{CPU Time Focus}

\begin{itemize}
\item Primary metric for processor performance
\item Excludes I/O and system effects
\item Directly reflects processor and memory system performance
\item Most relevant for comparing processor architectures
\end{itemize}

\subsubsection{The CPU Time Equation}

\textbf{Basic Formula}

CPU Time = Clock Cycles $\times$ Clock Period

Or equivalently:

CPU Time = Clock Cycles / Clock Rate

\textbf{Key Relationships}

\begin{itemize}
\item Clock Period = 1 / Clock Rate
\item Clock Rate measured in Hz (cycles/second)
\item Clock Cycles = total cycles to execute program
\item Higher clock rate $\rightarrow$ shorter clock period $\rightarrow$ faster execution
\end{itemize}

\textbf{Example Calculation}

Program requires 10 billion cycles
Processor runs at 4 GHz ($4 \times 10^9$ Hz)

CPU Time = $10 \times 10^9$ cycles / ($4 \times 10^9$ cycles/sec)
         = 2.5 seconds

\subsubsection{Instruction Count and CPI}

\textbf{Cycles Per Instruction (CPI)}

\begin{itemize}
\item Average number of clock cycles per instruction
\item Varies by instruction type and implementation
\item Key microarchitecture metric
\end{itemize}

\textbf{Extended CPU Time Equation}

CPU Time = Instruction Count $\times$ CPI $\times$ Clock Period

Or:

CPU Time = (Instruction Count $\times$ CPI) / Clock Rate

\textbf{Three Performance Factors}

\begin{enumerate}
\item \textbf{Instruction Count}: Number of instructions executed
\item \textbf{CPI}: Average cycles per instruction
\item \textbf{Clock Rate}: Speed of the processor clock
\end{enumerate}

\textbf{Factor Dependencies}

\begin{itemize}
\item Instruction Count: Determined by algorithm, compiler, ISA
\item CPI: Determined by processor implementation (microarchitecture)
\item Clock Rate: Determined by hardware technology and organization
\end{itemize}

\subsection{Understanding CPI in Detail}

\subsubsection{CPI Variability}

\textbf{Different Instructions, Different CPIs}

\begin{itemize}
\item Simple operations: May complete in 1 cycle (ADD, AND)
\item Memory operations: May take multiple cycles (LOAD, STORE)
\item Branch instructions: Variable cycles (depends on prediction)
\item Multiply/Divide: Often take many cycles
\end{itemize}

\textbf{Calculating Average CPI}

Average CPI = $\Sigma$ (CPI\_i $\times$ Instruction Count\_i) / Total Instruction Count

Where:

\begin{itemize}
\item CPI\_i = cycles per instruction for instruction type i
\item Instruction Count\_i = number of times instruction i executed
\end{itemize}

\subsubsection{CPI Example Calculation}

\textbf{Given:}

\begin{itemize}
\item Program executes 100,000 instructions
\item 50,000 ALU operations (CPI = 1)
\item 30,000 load instructions (CPI = 3)
\item 20,000 branch instructions (CPI = 2)
\end{itemize}

\textbf{Calculation:}

Total Cycles = (50,000 $\times$ 1) + (30,000 $\times$ 3) + (20,000 $\times$ 2)
             = 50,000 + 90,000 + 40,000
             = 180,000 cycles

Average CPI = 180,000 / 100,000 = 1.8

\subsubsection{Instruction Classes}

\textbf{Common Instruction Categories}

\begin{enumerate}
\item \textbf{Integer arithmetic}: ADD, SUB, AND, OR
\item \textbf{Data transfer}: LOAD, STORE
\item \textbf{Control flow}: BRANCH, JUMP, CALL
\item \textbf{Floating-point}: FADD, FMUL, FDIV
\end{enumerate}

\textbf{CPI Characteristics by Class}

\begin{itemize}
\item Integer arithmetic: Usually 1 cycle
\item Data transfer: 1-3 cycles (cache hit) or more (cache miss)
\item Control flow: 1-2 cycles (correct prediction) or more (misprediction)
\item Floating-point: 2-20+ cycles depending on operation
\end{itemize}

\subsection{Performance Optimization Principles}

\subsubsection{Make the Common Case Fast}

\textbf{Core Principle}

\begin{itemize}
\item Optimize frequent operations rather than rare ones
\item Greater impact on overall performance
\item Focus resources where they matter most
\end{itemize}

\textbf{Examples}

\begin{itemize}
\item Optimize ALU operations (common) over division (rare)
\item Fast cache for recent data (commonly accessed)
\item Branch prediction for likely paths
\item Simple instructions execute quickly
\end{itemize}

\textbf{Application in Design}

\begin{itemize}
\item Identify common operations through profiling
\item Allocate hardware resources accordingly
\item Accept slower performance for rare cases
\item Trade-offs guided by usage patterns
\end{itemize}

\subsubsection{Amdahl's Law}

\textbf{The Fundamental Principle}
The speedup that can be achieved by improving a particular part of a system is limited by the fraction of time that part is used.

\textbf{Mathematical Formula}

Speedup\_overall = 1 / [(1 - P) + (P / S)]

Where:

\begin{itemize}
\item P = Proportion of execution time that can be improved
\item S = Speedup of the improved portion
\item (1 - P) = Proportion that cannot be improved
\end{itemize}

\textbf{Alternative Formulation}

Execution Time\_new = Execution Time\_old $\times$ [(1 - P) + (P / S)]

\subsubsection{Amdahl's Law Examples}

\textbf{Example 1: Multiply Operation Speedup}

Given:

\begin{itemize}
\item Multiply operations take 80\% of execution time
\item New hardware makes multiplies 10$\times$ faster
\end{itemize}

Calculation:

P = 0.80 (80\% can be improved)
S = 10 (10$\times$ speedup)

Speedup\_overall = 1 / [(1 - 0.80) + (0.80 / 10)]
                = 1 / [0.20 + 0.08]
                = 1 / 0.28
                = 3.57$\times$

\textbf{Key Insight:} Despite 10$\times$ improvement in multiplies, overall speedup is only 3.57$\times$ because 20\% of time is unaffected.

\textbf{Example 2: Limited Improvement Fraction}

Given:

\begin{itemize}
\item Only 30\% of execution can be improved
\item Improvement is 100$\times$ faster
\end{itemize}

Calculation:

P = 0.30
S = 100

Speedup\_overall = 1 / [(1 - 0.30) + (0.30 / 100)]
                = 1 / [0.70 + 0.003]
                = 1 / 0.703
                = 1.42$\times$

\textbf{Key Insight:} Even with 100$\times$ improvement, overall speedup is only 1.42$\times$ because only 30\% of execution benefits.

\subsubsection{Implications of Amdahl's Law}

\textbf{Limitations of Parallelization}

\begin{itemize}
\item Serial portions limit parallel speedup
\item As parallelism increases, serial portion dominates
\item Cannot achieve infinite speedup regardless of cores
\end{itemize}

\textbf{Optimization Strategy}

\begin{itemize}
\item Focus on largest contributors to execution time
\item Consider what fraction can realistically be improved
\item Multiple small improvements may beat one large improvement
\item Balance improvements across components
\end{itemize}

\textbf{Example: Multicore Scaling}

If 90\% of program parallelizes perfectly:
2 cores:  Speedup = 1.82$\times$
4 cores:  Speedup = 3.08$\times$
8 cores:  Speedup = 4.71$\times$
16 cores: Speedup = 6.40$\times$
$\infty$ cores:  Speedup = 10.00$\times$ (maximum possible)

The 10\% serial portion ultimately limits speedup to 10$\times$.

\subsection{Complete Performance Analysis}

\subsubsection{The Complete Performance Equation}

\textbf{Bringing It All Together}

CPU Time = (Instruction Count $\times$ CPI $\times$ Clock Period)

Expanded:

CPU Time = (Instructions) $\times$ (Cycles/Instruction) $\times$ (Seconds/Cycle)

\textbf{What Affects Each Factor}

\textbf{Instruction Count:}

\begin{itemize}
\item Algorithm: Efficient algorithms execute fewer instructions
\item Programming language: High-level vs low-level
\item Compiler: Optimization quality
\item ISA: Instruction complexity and capabilities
\end{itemize}

\textbf{CPI:}

\begin{itemize}
\item ISA: Instruction complexity
\item Microarchitecture: Pipeline depth, branch prediction
\item Cache performance: Hit rates affect memory access CPI
\item Instruction mix: Distribution of instruction types
\end{itemize}

\textbf{Clock Period (or Clock Rate):}

\begin{itemize}
\item Technology: Transistor speed (nm process)
\item Organization: Pipeline depth, critical path length
\item Power constraints: Higher frequency requires more power
\item Cooling limitations: Heat dissipation capacity
\end{itemize}

\subsubsection{Performance Comparison Example}

\textbf{Scenario:}
Compare two implementations of the same ISA

\begin{itemize}
\item System A: Clock Rate = 2 GHz, CPI = 2.0
\item System B: Clock Rate = 3 GHz, CPI = 3.0
\item Same program with 1 million instructions
\end{itemize}

\textbf{System A:}

CPU Time\_A = ($1 \times 10^6$ instructions) $\times$ (2.0 cycles/instruction) / ($2 \times 10^9$ cycles/sec)
           = $2 \times 10^6$ cycles / ($2 \times 10^9$ cycles/sec)
           = 0.001 seconds = 1 millisecond

\textbf{System B:}

CPU Time\_B = ($1 \times 10^6$ instructions) $\times$ (3.0 cycles/instruction) / ($3 \times 10^9$ cycles/sec)
           = $3 \times 10^6$ cycles / ($3 \times 10^9$ cycles/sec)
           = 0.001 seconds = 1 millisecond

\textbf{Result:} Both systems have identical performance despite different clock rates and CPIs.

\subsubsection{Trade-offs in Design}

\textbf{Clock Rate vs. CPI Trade-off}

\begin{itemize}
\item Higher clock rate may require deeper pipeline
\item Deeper pipeline often increases CPI (more stalls)
\item Must balance frequency gains against CPI losses
\end{itemize}

\textbf{Instruction Count vs. CPI Trade-off}

\begin{itemize}
\item Complex instructions reduce instruction count
\item But complex instructions may increase CPI
\item CISC vs RISC architecture debate
\end{itemize}

\textbf{Power vs. Performance}

\begin{itemize}
\item Higher clock rate increases power consumption
\item Power = Capacitance $\times$ Voltage² $\times$ Frequency
\item Mobile systems prioritize power over peak performance
\end{itemize}

\subsection{Practical Performance Considerations}

\subsubsection{Benchmarking}

\textbf{Purpose of Benchmarks}

\begin{itemize}
\item Measure real-world performance
\item Compare different systems objectively
\item Standard workloads for reproducibility
\end{itemize}

\textbf{Types of Benchmarks}

\begin{itemize}
\item Synthetic: Artificial programs (e.g., Dhrystone, Whetstone)
\item Application: Real programs (e.g., SPEC CPU, databases)
\item Workload: Representative task mixes
\end{itemize}

\textbf{Benchmark Pitfalls}

\begin{itemize}
\item May not represent your workload
\item Can be optimized for unfairly
\item Need multiple benchmarks for complete picture
\end{itemize}

\subsubsection{Performance Metrics in Practice}

\textbf{MIPS (Million Instructions Per Second)}

MIPS = Instruction Count / (Execution Time $\times 10^6$)
     = Clock Rate / (CPI $\times 10^6$)

\textbf{Limitations of MIPS:}

\begin{itemize}
\item Doesn't account for instruction complexity
\item Different ISAs have different instruction capabilities
\item Higher MIPS doesn't guarantee better performance
\item "Meaningless Indication of Processor Speed"
\end{itemize}

\textbf{Better Metrics:}

\begin{itemize}
\item Execution time for specific workloads
\item Throughput for server applications
\item Energy efficiency (performance per watt)
\item Performance per dollar
\end{itemize}

\subsubsection{Power and Energy Considerations}

\textbf{Power Wall}

\begin{itemize}
\item Cannot increase clock rate indefinitely
\item Power consumption limits frequency scaling
\item Led to multi-core era
\end{itemize}

\textbf{Dynamic Power Equation}

Power = Capacitance $\times$ Voltage² $\times$ Frequency

\textbf{Energy Equation}

Energy = Power $\times$ Time

\textbf{Implications:}

\begin{itemize}
\item Lowering voltage reduces power dramatically (squared effect)
\item Higher frequency increases power linearly
\item Faster execution may save energy overall (less time)
\item Energy efficiency increasingly important metric
\end{itemize}

\subsection{Key Takeaways}

\begin{enumerate}
\item \textbf{Performance is the inverse of execution time} - faster systems have shorter execution times and higher performance values.
\end{enumerate}

\begin{enumerate}
\item \textbf{Three key factors determine CPU performance:}
\end{enumerate}

\begin{itemize}
\item Instruction Count (algorithm, compiler, ISA)
\item CPI (microarchitecture, instruction mix)
\item Clock Rate (technology, organization)
\end{itemize}

\begin{enumerate}
\item \textbf{Amdahl's Law limits speedup} - the potential speedup from improving any part of a system is limited by how much time that part is used.
\end{enumerate}

\begin{enumerate}
\item \textbf{"Make the common case fast"} - optimize frequently executed operations for maximum impact on overall performance.
\end{enumerate}

\begin{enumerate}
\item \textbf{CPI varies by instruction type} - average CPI depends on the mix of instructions and their individual costs.
\end{enumerate}

\begin{enumerate}
\item \textbf{Trade-offs are fundamental} - improvements in one area (e.g., clock rate) may harm another (e.g., CPI or power consumption).
\end{enumerate}

\begin{enumerate}
\item \textbf{Benchmarking is essential} - real workloads provide the most meaningful performance measurements.
\end{enumerate}

\begin{enumerate}
\item \textbf{Power is a critical constraint} - modern performance optimization must consider power and energy efficiency, not just speed.
\end{enumerate}

\begin{enumerate}
\item \textbf{Multiple factors must be optimized together} - focusing on only one aspect (like clock rate) can be counterproductive.
\end{enumerate}

\begin{enumerate}
\item \textbf{Understanding performance equations} enables rational design decisions and accurate performance predictions.
\end{enumerate}

\subsection{Summary}

Performance analysis is central to computer architecture, providing the foundation for making informed design decisions. By understanding the relationship between instruction count, CPI, and clock rate, architects can identify optimization opportunities and predict the impact of changes. Amdahl's Law reminds us that the benefit of any improvement is constrained by what fraction of execution time it affects, emphasizing the importance of focusing on the common case. As we design systems, we must balance competing factors—clock rate, CPI, power consumption, and cost—to achieve the best overall performance for target applications. The principles covered in this lecture provide the analytical framework for evaluating processor designs and optimization strategies throughout the study of computer architecture.


% Chapter 2: ARM Assembly Programming
\chapter{ARM Assembly Programming}

\section{Lecture 4: Introduction to ARM Assembly}

\emph{By Dr. Kisaru Liyanage}

\subsection{Introduction}

This lecture introduces ARM assembly language programming, providing the foundation for understanding how high-level programs translate to machine code. We explore the ARM instruction set architecture (ISA), focusing on its RISC design philosophy, register organization, basic instruction formats, and the toolchain used for development. Understanding assembly language is essential for comprehending how processors execute programs and for optimizing performance-critical code.

\subsection{ARM Architecture Overview}

\subsubsection{RISC Philosophy}

\textbf{Reduced Instruction Set Computer (RISC)}

\begin{itemize}
\item Simple, uniform instruction format
\item Fixed instruction length (32 bits in ARM)
\item Load/store architecture (only LOAD/STORE access memory)
\item Large number of general-purpose registers
\item Few addressing modes
\item Hardware simplicity for higher clock rates

\textbf{Contrasted with CISC (Complex Instruction Set Computer)}

| Feature | RISC | CISC |
|---------|------|------|
| \textbf{Instruction Format} | Simple, uniform format | Variable-length instructions |
| \textbf{Instruction Complexity} | Simple instructions, more instructions per program | Complex operations |
| \textbf{Memory Access} | Load/store architecture (only LOAD/STORE access memory) | Memory operands in arithmetic operations |
| \textbf{Registers} | Large number of general-purpose registers | Fewer registers |
| \textbf{Hardware Design} | Hardware simplicity for higher clock rates | More complex hardware |
| \textbf{Pipelining} | Regular structure enables efficient pipelining | More difficult to pipeline |

\textbf{ARM Design Principles}

\begin{itemize}
\item Simplicity enables high performance
\item Regular instruction encoding aids decoding
\item Load/store architecture simplifies memory access
\item Large register file reduces memory traffic
\item Consistent design across instruction types

\subsubsection{ARM Registers}

\textbf{General-Purpose Registers}

\begin{itemize}
\item \textbf{R0 to R15}: 16 registers total
\item \textbf{32 bits wide}: Can hold integers, addresses, or data
\item \textbf{R0-R12}: General computation and data storage
\item \textbf{R13 (SP)}: Stack Pointer - points to top of stack
\item \textbf{R14 (LR)}: Link Register - stores return address
\item \textbf{R15 (PC)}: Program Counter - address of next instruction

\textbf{Register Usage Conventions}
\begin{figure}[h]
\centering
\includegraphics[width=0.7\textwidth]{img/Chapter 2 ARM Conventions.jpg}
\caption{ARM Register Conventions}
\end{figure}

\textbf{R0-R3: Argument/result registers}
\begin{itemize}
\item Pass parameters to functions
\item Return values from functions
\item Scratch registers (not preserved)
\end{itemize}

\textbf{R4-R11: Local variable registers}
\begin{itemize}
\item Must be preserved across function calls
\item Callee saves/restores if used
\end{itemize}

\textbf{R12: Intra-procedure-call scratch register}
\begin{itemize}
\item Can be corrupted by function calls
\item Not preserved
\end{itemize}

\textbf{R13 (SP): Stack Pointer}
\begin{itemize}
\item Points to top of stack
\item Must always be valid
\end{itemize}

\textbf{R14 (LR): Link Register}
\begin{itemize}
\item Stores return address on function call
\item Contains address to return to
\end{itemize}

\textbf{R15 (PC): Program Counter}
\begin{itemize}
\item Always points to next instruction
\item Modifying PC changes execution flow
\end{itemize}

\textbf{Why So Many Registers?}

\begin{itemize}
\item Reduces memory accesses (faster than cache/RAM)
\item Enables register allocation by compiler
\item Supports efficient function calls
\item Improves performance through locality
\end{itemize}

\subsubsection{Memory Organization}

\textbf{Little-Endian Byte Ordering}

\begin{itemize}
\item Least significant byte at lowest address
\item Example: 0x12345678 stored as:

\begin{verbatim}
Address:  [base+0] [base+1] [base+2] [base+3]
Content:     78       56       34       12
\end{verbatim}

\textbf{Word Alignment}

\begin{itemize}
\item Words are 32 bits (4 bytes)
\item Word addresses should be multiples of 4
\item Accessing unaligned words may cause errors or slowdown

\textbf{Address Space}

\begin{itemize}
\item 32-bit addresses can access 2³² bytes = 4 GB
\item Byte-addressable memory
\item Instructions and data in same address space (Von Neumann architecture)

\subsection{ARM Instruction Format}

\subsubsection{Instruction Structure}

\textbf{Fixed 32-Bit Length}

\begin{itemize}
\item Every instruction exactly 32 bits
\item Simplifies instruction fetch and decode
\item Enables predictable pipeline operation

\textbf{Typical Instruction Fields}

\begin{verbatim}
[Condition][Opcode][Operands]
  4 bits    varies   varies
\end{verbatim}

\textbf{Example: ADD Instruction}

\begin{lstlisting}[language=assembly]
ADD R1, R2, R3    ; R1 = R2 + R3
\\end{lstlisting}

Encoding includes:
\begin{itemize}
\item Condition code (usually "always")
\item Opcode for ADD operation
\item Destination register (R1)
\item Source register 1 (R2)
\item Source register 2 (R3)

\subsubsection{Instruction Types}

\textbf{Data Processing Instructions}

\begin{itemize}
\item Arithmetic: ADD, SUB, RSB (reverse subtract)
\item Logical: AND, ORR, EOR (XOR), BIC (bit clear)
\item Comparison: CMP, CMN, TST, TEQ
\item Move: MOV, MVN (move negated)
\item Shift/Rotate: LSL, LSR, ASR, ROR

\textbf{Data Transfer Instructions}

\begin{itemize}
\item Load: LDR (word), LDRB (byte), LDRH (halfword)
\item Store: STR (word), STRB (byte), STRH (halfword)
\item Multiple: LDM, STM (load/store multiple registers)

\textbf{Control Flow Instructions}

\begin{itemize}
\item Branch: B (unconditional), BEQ, BNE, BGE, BLT, etc.
\item Function call: BL (branch and link)
\item Return: MOV PC, LR

\subsubsection{Operand Types}

\textbf{Register Operands}

\begin{lstlisting}[language=assembly]
ADD R0, R1, R2    ; R0 = R1 + R2 (all registers)
\\end{lstlisting}

\textbf{Immediate Operands}

\begin{lstlisting}[language=assembly]
ADD R0, R1, #5    ; R0 = R1 + 5 (# indicates immediate)
MOV R2, #100      ; R2 = 100
\\end{lstlisting}

\textbf{Immediate Value Constraints}

\begin{itemize}
\item Limited to certain patterns due to 32-bit instruction encoding
\item 8-bit immediate + 4-bit rotation
\item Assembler warns if immediate cannot be encoded

\textbf{Shifted Register Operands}

\begin{lstlisting}[language=assembly]
ADD R0, R1, R2, LSL #2    ; R0 = R1 + (R2 << 2)
SUB R3, R4, R5, LSR #1    ; R3 = R4 - (R5 >> 1)
\\end{lstlisting}

\subsection{Basic ARM Instructions}

\subsubsection{Arithmetic Instructions}

\textbf{Addition}

\begin{lstlisting}[language=assembly]
ADD Rd, Rn, Rm       ; Rd = Rn + Rm
ADD Rd, Rn, #imm     ; Rd = Rn + immediate
\\end{lstlisting}

Examples:
\begin{lstlisting}[language=assembly]
ADD R0, R1, R2       ; R0 = R1 + R2
ADD R3, R3, #1       ; R3 = R3 + 1 (increment)
\\end{lstlisting}

\textbf{Subtraction}

\begin{lstlisting}[language=assembly]
SUB Rd, Rn, Rm       ; Rd = Rn - Rm
SUB Rd, Rn, #imm     ; Rd = Rn - immediate
RSB Rd, Rn, #imm     ; Rd = immediate - Rn (reverse subtract)
\\end{lstlisting}

Examples:
\begin{lstlisting}[language=assembly]
SUB R0, R1, R2       ; R0 = R1 - R2
SUB R4, R4, #10      ; R4 = R4 - 10 (decrement)
RSB R5, R6, #0       ; R5 = 0 - R6 (negate)
\\end{lstlisting}

\textbf{Multiplication} (covered in later tutorials)

\begin{lstlisting}[language=assembly]
MUL Rd, Rn, Rm       ; Rd = Rn × Rm (lower 32 bits)
\\end{lstlisting}

\subsubsection{Logical Instructions}

\textbf{AND Operation}

\begin{lstlisting}[language=assembly]
AND Rd, Rn, Rm       ; Rd = Rn AND Rm
AND Rd, Rn, #imm     ; Rd = Rn AND immediate
\\end{lstlisting}

Usage: Bit masking, clearing specific bits

Example:
\begin{lstlisting}[language=assembly]
AND R0, R0, #0xFF    ; Keep only lower 8 bits
\\end{lstlisting}

\textbf{OR Operation}

\begin{lstlisting}[language=assembly]
ORR Rd, Rn, Rm       ; Rd = Rn OR Rm (ORR in ARM)
ORR Rd, Rn, #imm     ; Rd = Rn OR immediate
\\end{lstlisting}

Usage: Setting specific bits

Example:
\begin{lstlisting}[language=assembly]
ORR R1, R1, #0x80    ; Set bit 7
\\end{lstlisting}

\textbf{Exclusive OR}

\begin{lstlisting}[language=assembly]
EOR Rd, Rn, Rm       ; Rd = Rn XOR Rm
EOR Rd, Rn, #imm     ; Rd = Rn XOR immediate
\\end{lstlisting}

Usage: Toggling bits, fast comparison

Example:
\begin{lstlisting}[language=assembly]
EOR R2, R2, R2       ; R2 = 0 (XOR with itself)
\\end{lstlisting}

\textbf{Move and Move Not}

\begin{lstlisting}[language=assembly]
MOV Rd, Rm           ; Rd = Rm
MOV Rd, #imm         ; Rd = immediate
MVN Rd, Rm           ; Rd = NOT Rm (bitwise complement)
\\end{lstlisting}

Examples:
\begin{lstlisting}[language=assembly]
MOV R0, R1           ; Copy R1 to R0
MOV R2, #0           ; Clear R2
MVN R3, R4           ; R3 = ~R4 (invert all bits)
\\end{lstlisting}

\subsubsection{Shift Operations}

\textbf{Logical Shift Left (LSL)}

\begin{lstlisting}[language=assembly]
LSL Rd, Rn, #shift   ; Rd = Rn << shift
MOV Rd, Rn, LSL #shift
\\end{lstlisting}

Effect: Multiplies by 2^shift

Example:
\begin{lstlisting}[language=assembly]
LSL R0, R1, #2       ; R0 = R1 × 4
\\end{lstlisting}

\textbf{Logical Shift Right (LSR)}

\begin{lstlisting}[language=assembly]
LSR Rd, Rn, #shift   ; Rd = Rn >> shift (unsigned)
MOV Rd, Rn, LSR #shift
\\end{lstlisting}

Effect: Divides by 2^shift (unsigned)

Example:
\begin{lstlisting}[language=assembly]
LSR R0, R1, #3       ; R0 = R1 / 8
\\end{lstlisting}

\textbf{Arithmetic Shift Right (ASR)}

\begin{lstlisting}[language=assembly]
ASR Rd, Rn, #shift   ; Rd = Rn >> shift (signed)
\\end{lstlisting}

Effect: Divides by 2^shift, preserves sign

Example:
\begin{lstlisting}[language=assembly]
ASR R0, R1, #2       ; R0 = R1 / 4 (signed)
\\end{lstlisting}

\textbf{Rotate Right (ROR)}

\begin{lstlisting}[language=assembly]
ROR Rd, Rn, #shift   ; Rotate Rn right by shift
\\end{lstlisting}

Effect: Bits rotated off right end reappear at left

Example:
\begin{lstlisting}[language=assembly]
ROR R0, R1, #8       ; Rotate R1 right by 8 bits
\\end{lstlisting}

\subsection{Memory Access Instructions}

\subsubsection{Load Instructions}

\textbf{Load Word (LDR)}

\begin{lstlisting}[language=assembly]
LDR Rd, [Rn]         ; Rd = Memory[Rn]
LDR Rd, [Rn, #offset]; Rd = Memory[Rn + offset]
\\end{lstlisting}

Examples:
\begin{lstlisting}[language=assembly]
LDR R0, [R1]         ; Load word from address in R1
LDR R2, [R3, #4]     ; Load from address R3+4
LDR R4, [R5, #-8]    ; Load from address R5-8
\\end{lstlisting}

\textbf{Load Byte (LDRB)}

\begin{lstlisting}[language=assembly]
LDRB Rd, [Rn, #offset]; Load one byte, zero-extend to 32 bits
\\end{lstlisting}

Example:
\begin{lstlisting}[language=assembly]
LDRB R0, [R1]        ; R0 = (byte at R1), upper 24 bits = 0
\\end{lstlisting}

\textbf{Load Halfword (LDRH)}

\begin{lstlisting}[language=assembly]
LDRH Rd, [Rn, #offset]; Load 16 bits, zero-extend to 32 bits
\\end{lstlisting}

Example:
\begin{lstlisting}[language=assembly]
LDRH R0, [R1, #2]    ; R0 = (halfword at R1+2), upper 16 bits = 0
\\end{lstlisting}

\textbf{Pseudo-Instruction for Loading Addresses}

\begin{lstlisting}[language=assembly]
LDR Rd, =label       ; Load address of label into Rd
LDR Rd, =value       ; Load 32-bit constant into Rd
\\end{lstlisting}

Examples:
\begin{lstlisting}[language=assembly]
LDR R0, =array       ; R0 = address of array
LDR R1, =0x12345678  ; R1 = 0x12345678 (large immediate)
\\end{lstlisting}

\subsubsection{Store Instructions}

\textbf{Store Word (STR)}

\begin{lstlisting}[language=assembly]
STR Rd, [Rn]         ; Memory[Rn] = Rd
STR Rd, [Rn, #offset]; Memory[Rn + offset] = Rd
\\end{lstlisting}

Examples:
\begin{lstlisting}[language=assembly]
STR R0, [R1]         ; Store R0 to address in R1
STR R2, [R3, #8]     ; Store R2 to address R3+8
\\end{lstlisting}

\textbf{Store Byte (STRB)}

\begin{lstlisting}[language=assembly]
STRB Rd, [Rn, #offset]; Store lower 8 bits of Rd
\\end{lstlisting}

Example:
\begin{lstlisting}[language=assembly]
STRB R0, [R1]        ; Store lower byte of R0 to address R1
\\end{lstlisting}

\textbf{Store Halfword (STRH)}

\begin{lstlisting}[language=assembly]
STRH Rd, [Rn, #offset]; Store lower 16 bits of Rd
\\end{lstlisting}

Example:
\begin{lstlisting}[language=assembly]
STRH R0, [R1, #4]    ; Store lower halfword of R0 to R1+4
\\end{lstlisting}

\subsubsection{Addressing Modes}

\textbf{Offset Addressing}

\begin{lstlisting}[language=assembly]
LDR R0, [R1, #4]     ; R0 = Memory[R1 + 4], R1 unchanged
\\end{lstlisting}

\textbf{Pre-indexed Addressing}

\begin{lstlisting}[language=assembly]
LDR R0, [R1, #4]!    ; R1 = R1 + 4, then R0 = Memory[R1]
                      ; ! indicates update base register
\\end{lstlisting}

\textbf{Post-indexed Addressing}

\begin{lstlisting}[language=assembly]
LDR R0, [R1], #4     ; R0 = Memory[R1], then R1 = R1 + 4
\\end{lstlisting}

\textbf{Register Offset}

\begin{lstlisting}[language=assembly]
LDR R0, [R1, R2]     ; R0 = Memory[R1 + R2]
LDR R0, [R1, R2, LSL #2] ; R0 = Memory[R1 + (R2 << 2)]
\\end{lstlisting}

\subsection{Assembly Program Structure}

\subsubsection{Directives}

\textbf{Section Directives}

\begin{lstlisting}[language=assembly]
.text                ; Code section (instructions)
.data                ; Data section (initialized variables)
.bss                 ; Uninitialized data section
\\end{lstlisting}

\textbf{Global and External}

\begin{lstlisting}[language=assembly]
.global main         ; Make symbol visible to linker
.extern printf       ; Declare external symbol
\\end{lstlisting}

\textbf{Data Definition}

\begin{lstlisting}[language=assembly]
.word value          ; Define 32-bit word
.byte value          ; Define byte
.asciz "string"      ; Define null-terminated string
.space n             ; Reserve n bytes of space
\\end{lstlisting}

\subsubsection{Labels}

\textbf{Purpose}

\begin{itemize}
\item Mark locations in code or data
\item Provide symbolic names for addresses
\item Enable jumps and references

\textbf{Syntax}

\begin{lstlisting}[language=assembly]
label:               ; Label for instruction
    MOV R0, #1
    ADD R1, R0, R2

array:               ; Label for data
    .word 1, 2, 3, 4
\\end{lstlisting}

\subsubsection{Simple Program Example}

\begin{lstlisting}[language=assembly]
    .text
    .global main

main:
    MOV R0, #5       ; R0 = 5
    MOV R1, #10      ; R1 = 10
    ADD R2, R0, R1   ; R2 = R0 + R1 = 15
    MOV R0, R2       ; R0 = R2 (return value)
    MOV PC, LR       ; Return from main

    .data
message:
    .asciz "Hello, ARM!"
\\end{lstlisting}

\subsection{ARM Development Tools}

\subsubsection{Toolchain Components}

\textbf{Cross-Compiler}

\begin{itemize}
\item \texttt{arm-linux-gnueabi-gcc}: Compiles C to ARM code
\item Runs on x86 PC, produces ARM binaries
\item Necessary because development machine $\neq$ target machine

\textbf{Assembler}

\begin{itemize}
\item \texttt{arm-linux-gnueabi-as}: Assembles ARM assembly to object code
\item Part of binutils package

\textbf{Linker}

\begin{itemize}
\item \texttt{arm-linux-gnueabi-ld}: Links object files to executable
\item Resolves symbols, combines code sections

\textbf{Emulator}

\begin{itemize}
\item \texttt{qemu-arm}: Emulates ARM processor on x86
\item Allows running ARM binaries on PC
\item Useful for testing without ARM hardware

\subsubsection{Compilation Process}

\textbf{From C to Executable}

\begin{verbatim}
C Source (.c)
    --> [gcc -S]
Assembly (.s)
    --> [as]
Object Code (.o)
    --> [ld]
Executable (a.out)
    --> [qemu-arm]
Execution
\end{verbatim}

\textbf{Command Examples}

\begin{lstlisting}[language=bash]
# Compile C to assembly
arm-linux-gnueabi-gcc -S program.c -o program.s

# Assemble to object code
arm-linux-gnueabi-as program.s -o program.o

# Link to executable
arm-linux-gnueabi-gcc program.o -o program

# Run with emulator
qemu-arm program
\\end{lstlisting}

\textbf{One-Step Compilation}

\begin{lstlisting}[language=bash]
# Compile, assemble, and link in one command
arm-linux-gnueabi-gcc program.c -o program
\\end{lstlisting}

\subsubsection{Debugging and Inspection}

\textbf{GDB (GNU Debugger)}

\begin{lstlisting}[language=bash]
# Debug with QEMU and GDB
qemu-arm -g 1234 program &     # Start QEMU, wait for debugger
arm-linux-gnueabi-gdb program  # Start GDB
(gdb) target remote :1234      # Connect to QEMU
(gdb) break main               # Set breakpoint
(gdb) continue                 # Run to breakpoint
(gdb) step                     # Execute one instruction
(gdb) info registers           # Show register values
\\end{lstlisting}

\textbf{Objdump}

\begin{lstlisting}[language=bash]
# Disassemble binary to assembly
arm-linux-gnueabi-objdump -d program
\\end{lstlisting}

\textbf{nm}

\begin{lstlisting}[language=bash]
# List symbols in object file
arm-linux-gnueabi-nm program.o
\\end{lstlisting}

\subsection{Programming in ARM Assembly}

\subsubsection{Translating C to ARM}

\textbf{C Code:}

\begin{lstlisting}[language=c]
int a = 5;
int b = 10;
int c = a + b;
\\end{lstlisting}

\textbf{ARM Assembly:}

\begin{lstlisting}[language=assembly]
    MOV R0, #5       ; a = 5
    MOV R1, #10      ; b = 10
    ADD R2, R0, R1   ; c = a + b
\\end{lstlisting}

\textbf{C Code with Array:}

\begin{lstlisting}[language=c]
int arr[3] = {1, 2, 3};
int x = arr[1];
\\end{lstlisting}

\textbf{ARM Assembly:}

\begin{lstlisting}[language=assembly]
    .data
arr:
    .word 1, 2, 3

    .text
    LDR R0, =arr     ; R0 = address of arr
    LDR R1, [R0, #4] ; R1 = arr[1] (offset 4 bytes)
\\end{lstlisting}

\subsubsection{Common Patterns}

\textbf{Clearing a Register}

\begin{lstlisting}[language=assembly]
MOV R0, #0           ; Method 1
EOR R0, R0, R0       ; Method 2 (XOR with itself)
\\end{lstlisting}

\textbf{Negating a Value}

\begin{lstlisting}[language=assembly]
RSB R0, R0, #0       ; R0 = 0 - R0
MVN R0, R0           ; R0 = ~R0 (bitwise, not arithmetic)
ADD R0, R0, #1       ; Then add 1 (two's complement)
\\end{lstlisting}

\textbf{Multiplying by Powers of 2}

\begin{lstlisting}[language=assembly]
LSL R0, R1, #3       ; R0 = R1 × 8 (faster than MUL)
\\end{lstlisting}

\textbf{Dividing by Powers of 2}

\begin{lstlisting}[language=assembly]
LSR R0, R1, #2       ; R0 = R1 / 4 (unsigned)
ASR R0, R1, #2       ; R0 = R1 / 4 (signed)
\\end{lstlisting}

\textbf{Swapping Two Registers}

\begin{lstlisting}[language=assembly]
EOR R0, R0, R1       ; XOR-based swap (no temporary)
EOR R1, R0, R1
EOR R0, R0, R1
\end{lstlisting}

\subsection{Key Takeaways}

\begin{itemize}
\item \textbf{ARM follows RISC principles} - simple instructions, load/store architecture, large register file, fixed instruction length
\item \textbf{16 registers (R0-R15)} with special purposes: R13 (SP), R14 (LR), R15 (PC), and calling conventions for R0-R11
\item \textbf{16 general-purpose registers} (R0-R15) with special roles for SP, LR, and PC
\item \textbf{Three main instruction categories} - data processing (arithmetic/logic), data transfer (load/store), control flow (branches)
\item \textbf{Fixed 32-bit instruction format} simplifies hardware and enables efficient pipelining
\item \textbf{Little-endian byte ordering} - least significant byte stored at lowest address
\item \textbf{Immediate values} indicated by \# symbol, with encoding constraints due to fixed instruction size
\item \textbf{Memory access only through LOAD/STORE} - arithmetic operations work on registers only (load/store architecture)
\item \textbf{Rich addressing modes} - offset, pre-indexed, post-indexed, register offset with optional shifts
\item \textbf{Cross-compilation toolchain} - arm-linux-gnueabi-gcc, as, ld, and qemu-arm for development on x86
\item \textbf{Assembly programming requires understanding} of register allocation, instruction selection, and calling conventions
\end{itemize}

\subsection{Summary}

ARM assembly language provides the low-level interface between software and hardware, revealing how high-level constructs translate to machine operations. The ARM architecture's RISC design emphasizes simplicity and regularity, with a uniform 32-bit instruction format, a generous 16-register set, and a clean separation between computation (using registers) and memory access (through explicit load/store instructions). Understanding ARM assembly is crucial for optimizing performance-critical code, implementing system-level software, and comprehending how processors execute programs. The development toolchain---including cross-compilers, assemblers, linkers, and emulators---enables efficient development and testing of ARM software. Mastering these fundamentals prepares us for more advanced topics including function calling conventions, stack management, and processor microarchitecture implementation.

\section{Lecture 5: Number Representation and Instruction Encoding}

\emph{By Dr. Kisaru Liyanage}

\subsection{Introduction}

This lecture delves into how computers represent and manipulate data at the binary level. We explore number systems, two's complement representation for signed integers, instruction encoding formats in ARM assembly, and logical operations for bit manipulation. Understanding these fundamentals is essential for programming efficiently in assembly language and comprehending how processors execute arithmetic and logical operations.

\subsection{Number Representation Systems}

\subsubsection{Unsigned Binary Integers}

\textbf{Binary System Basics}

\begin{itemize}
\item Base-2 number system using digits 0 and 1
\item Each bit position represents a power of 2
\item Rightmost bit is least significant (LSB)
\item Leftmost bit is most significant (MSB)

\textbf{Place Value Calculation}

\begin{verbatim}
Binary: 1011
Value = (1 × 2³) + (0 × 2²) + (1 × 2¹) + (1 × 2⁰)
      = 8 + 0 + 2 + 1
      = 11 (decimal)
\end{verbatim}

\textbf{N-Bit Unsigned Range}

\begin{itemize}
\item N bits can represent 2^N different values
\item Range: 0 to (2^N - 1)
\item 8 bits: 0 to 255
\item 32 bits: 0 to 4,294,967,295

\textbf{Binary to Decimal Conversion}

\begin{verbatim}
Example: 10110101
= 1×128 + 0×64 + 1×32 + 1×16 + 0×8 + 1×4 + 0×2 + 1×1
= 128 + 32 + 16 + 4 + 1
= 181
\end{verbatim}

\subsubsection{Two's Complement Representation}

\textbf{Purpose of Two's Complement}

\begin{itemize}
\item Represents both positive and negative integers
\item Simplifies hardware (same adder for signed/unsigned)
\item Unique zero representation
\item Natural overflow behavior

\textbf{Sign Bit}

\begin{itemize}
\item MSB indicates sign
\item MSB = 0: Positive number
\item MSB = 1: Negative number

\textbf{Positive Numbers}

\begin{itemize}
\item Same as unsigned binary
\item MSB is always 0
\item Example: +5 in 8 bits = 00000101

\textbf{Negative Numbers}

\begin{itemize}
\item Represented as 2^N - |value|
\item Example: -5 in 8 bits:
\begin{verbatim}
  2^8 - 5 = 256 - 5 = 251 = 11111011
\end{verbatim}

\textbf{Two's Complement Conversion}
Method 1 (Invert and Add):

\begin{enumerate}
\item Write positive value in binary
\item Invert all bits (0$\rightarrow$1, 1$\rightarrow$0)
\item Add 1 to result

Example: -5 in 8 bits

\begin{verbatim}
+5:        00000101
Invert:    11111010
Add 1:     11111011  (this is -5)
\end{verbatim}

Method 2 (Subtraction):

\begin{verbatim}
-5 = 2^8 - 5 = 256 - 5 = 251 = 11111011
\end{verbatim}

\textbf{N-Bit Signed Range}

\begin{itemize}
\item Range: -(2^(N-1)) to +(2^(N-1) - 1)
\item 8 bits: -128 to +127
\item 32 bits: -2,147,483,648 to +2,147,483,647

\textbf{Special Cases}

\begin{itemize}
\item Zero: 00000000 (unique representation)
\item Most negative: 10000000 (-128 in 8 bits)
\item Has no positive counterpart!
\item Negating gives overflow

\subsubsection{Sign Extension}

\textbf{Purpose}

\begin{itemize}
\item Extend smaller signed value to larger width
\item Preserve numerical value
\item Required when loading bytes/halfwords into 32-bit registers

\textbf{Process}

\begin{itemize}
\item Replicate the sign bit (MSB) to fill new bits
\item Preserves positive/negative value

\textbf{Examples}

\begin{verbatim}
8-bit to 32-bit:
00000101 (+5) $\rightarrow$ 00000000 00000000 00000000 00000101 (+5)
11111011 (-5) $\rightarrow$ 11111111 11111111 11111111 11111011 (-5)
\end{verbatim}

\textbf{ARM Instructions for Sign Extension}

\begin{itemize}
\item \textbf{LDRH}: Load halfword (16 bits), zero-extend to 32 bits
\item \textbf{LDRSH}: Load signed halfword, sign-extend to 32 bits
\item \textbf{LDRB}: Load byte (8 bits), zero-extend to 32 bits
\item \textbf{LDRSB}: Load signed byte, sign-extend to 32 bits

\textbf{Example Usage}

\begin{lstlisting}[language=assembly]
LDRH R0, [R1]     ; R0 = 0x0000ABCD (zero-extended)
LDRSH R0, [R1]    ; R0 = 0xFFFFABCD (sign-extended if bit 15 = 1)

LDRB R0, [R1]     ; R0 = 0x000000AB (zero-extended)
LDRSB R0, [R1]    ; R0 = 0xFFFFFFAB (sign-extended if bit 7 = 1)
\end{verbatim}

\subsubsection{Hexadecimal Notation}

\textbf{Why Hexadecimal?}

\begin{itemize}
\item Compact representation of binary
\item One hex digit = 4 binary bits
\item Easier to read than long binary strings
\item Common in programming and debugging

\textbf{Hex Digits}

\begin{verbatim}
Binary  | Hex | Decimal
--------|-----|--------
0000    |  0  |   0
0001    |  1  |   1
0010    |  2  |   2
0011    |  3  |   3
0100    |  4  |   4
0101    |  5  |   5
0110    |  6  |   6
0111    |  7  |   7
1000    |  8  |   8
1001    |  9  |   9
1010    |  A  |  10
1011    |  B  |  11
1100    |  C  |  12
1101    |  D  |  13
1110    |  E  |  14
1111    |  F  |  15
\end{verbatim}

\textbf{Conversion Examples}

\begin{verbatim}
Binary: 1011 0110 1101 0010
Hex:      B    6    D    2
Result: 0xB6D2

Hex: 0x3F
Binary: 0011 1111
Decimal: 63
\end{verbatim}

\textbf{ARM Hexadecimal Usage}

\begin{lstlisting}[language=assembly]
MOV R0, #0xFF        ; R0 = 255
MOV R1, #0x100       ; R1 = 256
LDR R2, =0xDEADBEEF  ; R2 = 3735928559
\end{verbatim}

\subsection{ARM Instruction Encoding}

\subsubsection{Fixed-Length Instructions}

\textbf{32-Bit Instruction Format}

\begin{itemize}
\item Every ARM instruction is exactly 32 bits
\item Simplifies instruction fetch and decode
\item Enables efficient pipelining

\textbf{Advantages}

\begin{itemize}
\item Predictable instruction boundaries
\item Simple PC increment (always +4)
\item Fast decode logic

\textbf{Trade-offs}

\begin{itemize}
\item Some instructions may "waste" bits
\item Immediate values limited in size
\item Code density lower than variable-length (e.g., x86)

\subsubsection{Data Processing Instruction Format}

\textbf{Format Structure}

\begin{verbatim}
[Cond][00][I][Opcode][S][Rn][Rd][Operand2]
 4-bit 2  1   4-bit   1  4   4   12-bit
\end{verbatim}

\textbf{Field Descriptions}

\textbf{Condition (4 bits, bits 28-31)}

\begin{itemize}
\item Conditional execution feature
\item 0000 = EQ (equal, Z=1)
\item 0001 = NE (not equal, Z=0)
\item 1010 = GE (greater or equal, signed)
\item 1110 = AL (always execute, default)

\textbf{I bit (bit 25)}

\begin{itemize}
\item 0 = Operand2 is register
\item 1 = Operand2 is immediate value

\textbf{Opcode (4 bits, bits 21-24)}

\begin{itemize}
\item Specifies operation (AND, EOR, SUB, ADD, etc.)
\item 0100 = ADD
\item 0010 = SUB
\item 0000 = AND
\item 1100 = ORR

\textbf{S bit (bit 20)}

\begin{itemize}
\item 0 = Don't update condition flags
\item 1 = Update flags (CPSR)

\textbf{Rn (4 bits, bits 16-19)}

\begin{itemize}
\item First operand register number
\item 0000 = R0, 0001 = R1, etc.

\textbf{Rd (4 bits, bits 12-15)}

\begin{itemize}
\item Destination register number

\textbf{Operand2 (12 bits, bits 0-11)}

\begin{itemize}
\item If I=0: Shift amount and second register
\item If I=1: 8-bit immediate + 4-bit rotation

\textbf{Example: ADD R0, R1, R2}

\begin{verbatim}
Encoding fields:
- Cond: 1110 (always)
- I: 0 (register operand)
- Opcode: 0100 (ADD)
- S: 0 (don't update flags)
- Rn: 0001 (R1)
- Rd: 0000 (R0)
- Operand2: 0002 (R2, no shift)

Result: 0xE0810002
\end{verbatim}

\subsubsection{Data Transfer Instruction Format}

\textbf{Format Structure}

\begin{verbatim}
[Cond][01][I][P][U][B][W][L][Rn][Rd][Offset]
 4-bit 2  1  1  1  1  1  1  4   4   12-bit
\end{verbatim}

\textbf{Key Fields}

\textbf{L bit (bit 20)}

\begin{itemize}
\item 0 = Store (STR)
\item 1 = Load (LDR)

\textbf{B bit (bit 22)}

\begin{itemize}
\item 0 = Word transfer (32 bits)
\item 1 = Byte transfer (8 bits)

\textbf{P bit (bit 24)}

\begin{itemize}
\item 0 = Post-indexed addressing
\item 1 = Pre-indexed or offset addressing

\textbf{U bit (bit 23)}

\begin{itemize}
\item 0 = Subtract offset from base
\item 1 = Add offset to base

\textbf{W bit (bit 21)}

\begin{itemize}
\item 0 = No write-back
\item 1 = Write-back (update base register)

\textbf{Rn (base register)}

\begin{itemize}
\item Contains memory address or base address

\textbf{Rd (data register)}

\begin{itemize}
\item For Load: Destination register
\item For Store: Source register

\textbf{Offset (12 bits)}

\begin{itemize}
\item Memory address offset
\item Can be immediate or register

\textbf{Example: LDR R0, [R1, \#4]}

\begin{verbatim}
Encoding fields:
- Cond: 1110 (always)
- L: 1 (load)
- B: 0 (word)
- P: 1 (offset addressing)
- U: 1 (add offset)
- Rn: 0001 (R1)
- Rd: 0000 (R0)
- Offset: 004 (immediate 4)

Result: 0xE5910004
\end{verbatim}

\subsubsection{Immediate Value Encoding}

\textbf{Challenge}

\begin{itemize}
\item 32-bit instruction must fit: opcode, registers, immediate
\item Cannot fit full 32-bit immediate

\textbf{ARM Solution: 8-bit + 4-bit Rotation}

\begin{itemize}
\item Immediate field: 12 bits total
\item Lower 8 bits: Immediate value (0-255)
\item Upper 4 bits: Rotation amount (0-15)
\item Rotation: Right by (2 $\times$ rotation field) bits

\textbf{Calculation}

\begin{verbatim}
Actual Value = Immediate × ROR (2 × Rotation)
\end{verbatim}

\textbf{Examples}

\begin{verbatim}
Immediate=0xFF, Rotation=0:
  Value = 0xFF ROR 0 = 0x000000FF

Immediate=0xFF, Rotation=8:
  Value = 0xFF ROR 16 = 0x00FF0000

Immediate=0xFF, Rotation=12:
  Value = 0xFF ROR 24 = 0xFF000000
\end{verbatim}

\textbf{Allowed Immediates}

\begin{itemize}
\item Not all 32-bit values can be encoded
\item Valid: 0xFF, 0xFF00, 0xFF0000, 0xFF000000
\item Valid: 0xFF000000FF (rotation wraps around)
\item Invalid: 0x123 (cannot be formed by rotation)

\textbf{Assembler Handling}

\begin{itemize}
\item Assembler checks if immediate is valid
\item Gives error if immediate cannot be encoded
\item Use LDR pseudo-instruction for arbitrary values:
\begin{lstlisting}[language=assembly]
  LDR R0, =0x12345678  ; Loads from literal pool
\end{verbatim}

\subsection{Logical Operations}

\subsubsection{Bitwise AND}

\textbf{Operation}

\begin{itemize}
\item Performs logical AND on each bit pair
\item Result bit = 1 only if both input bits are 1

\textbf{Truth Table}

\begin{verbatim}
A | B | A AND B
--|---|--------
0 | 0 |   0
0 | 1 |   0
1 | 0 |   0
1 | 1 |   1
\end{verbatim}

\textbf{ARM Instruction}

\begin{lstlisting}[language=assembly]
AND Rd, Rn, Rm       ; Rd = Rn AND Rm
AND Rd, Rn, #imm     ; Rd = Rn AND immediate
\end{verbatim}

\textbf{Common Uses}

\textbf{Bit Masking (Extract Specific Bits)}

\begin{lstlisting}[language=assembly]
; Extract lower 8 bits of R1
MOV R0, R1
AND R0, R0, #0xFF    ; R0 = R1 & 0xFF (keep bits 0-7)

; Extract bits 8-15
MOV R0, R1
AND R0, R0, #0xFF00  ; R0 = R1 & 0xFF00 (keep bits 8-15)
\end{verbatim}

\textbf{Clearing Specific Bits}

\begin{lstlisting}[language=assembly]
; Clear bit 5 of R1
AND R1, R1, #0xFFFFFFDF  ; Bit 5 mask: ~(1 << 5)
\end{verbatim}

\textbf{Checking if Bit Set}

\begin{lstlisting}[language=assembly]
AND R2, R1, #0x80    ; Check if bit 7 is set
CMP R2, #0           ; Compare with zero
BEQ bit_clear        ; Branch if bit was clear
\end{verbatim}

\subsubsection{Bitwise OR}

\textbf{Operation}

\begin{itemize}
\item Performs logical OR on each bit pair
\item Result bit = 1 if either input bit is 1

\textbf{Truth Table}

\begin{verbatim}
A | B | A OR B
--|---|-------
0 | 0 |   0
0 | 1 |   1
1 | 0 |   1
1 | 1 |   1
\end{verbatim}

\textbf{ARM Instruction}

\begin{lstlisting}[language=assembly]
ORR Rd, Rn, Rm       ; Rd = Rn OR Rm (ORR in ARM)
ORR Rd, Rn, #imm     ; Rd = Rn OR immediate
\end{verbatim}

\textbf{Common Uses}

\textbf{Setting Specific Bits}

\begin{lstlisting}[language=assembly]
; Set bit 3 of R1
ORR R1, R1, #0x08    ; Bit 3 mask: (1 << 3) = 0x08

; Set bits 4 and 5
ORR R1, R1, #0x30    ; Mask: 0x30 = 0b00110000
\end{verbatim}

\textbf{Combining Values}

\begin{lstlisting}[language=assembly]
; Combine lower byte of R1 with upper bytes of R2
AND R1, R1, #0xFF        ; Keep only lower byte
AND R2, R2, #0xFFFFFF00  ; Keep only upper bytes
ORR R0, R1, R2           ; Combine
\end{verbatim}

\subsubsection{Bitwise XOR (Exclusive OR)}

\textbf{Operation}

\begin{itemize}
\item Performs logical XOR on each bit pair
\item Result bit = 1 if input bits differ

\textbf{Truth Table}

\begin{verbatim}
A | B | A XOR B
--|---|--------
0 | 0 |   0
0 | 1 |   1
1 | 0 |   1
1 | 1 |   0
\end{verbatim}

\textbf{ARM Instruction}

\begin{lstlisting}[language=assembly]
EOR Rd, Rn, Rm       ; Rd = Rn EOR Rm (EOR in ARM)
EOR Rd, Rn, #imm     ; Rd = Rn EOR immediate
\end{verbatim}

\textbf{Common Uses}

\textbf{Toggling Specific Bits}

\begin{lstlisting}[language=assembly]
; Toggle bit 2 of R1
EOR R1, R1, #0x04    ; Bit 2 mask: (1 << 2)
; If bit was 0, becomes 1; if was 1, becomes 0
\end{verbatim}

\textbf{Fast Zero}

\begin{lstlisting}[language=assembly]
EOR R0, R0, R0       ; R0 = 0 (XOR with itself)
\end{verbatim}

\textbf{Comparison}

\begin{lstlisting}[language=assembly]
; Check if R1 and R2 are equal
EOR R3, R1, R2       ; R3 = R1 XOR R2
CMP R3, #0           ; If R3 = 0, R1 == R2
BEQ values_equal
\end{verbatim}

\textbf{Swapping Without Temporary}

\begin{lstlisting}[language=assembly]
; Swap R0 and R1 without using another register
EOR R0, R0, R1
EOR R1, R0, R1
EOR R0, R0, R1
; Now R0 and R1 are swapped
\end{verbatim}

\subsubsection{Bitwise NOT}

\textbf{Operation}

\begin{itemize}
\item Inverts all bits (0$\rightarrow$1, 1$\rightarrow$0)
\item Also called complement

\textbf{ARM Instruction}

\begin{lstlisting}[language=assembly]
MVN Rd, Rm           ; Rd = NOT Rm (Move Not)
MVN Rd, #imm         ; Rd = NOT immediate
\end{verbatim}

\textbf{Common Uses}

\textbf{Creating Bit Masks}

\begin{lstlisting}[language=assembly]
; Create mask with all bits set except bit 3
MOV R0, #0x08        ; 0x08 = 0b00001000
MVN R1, R0           ; R1 = 0xFFFFFFF7 (all except bit 3)
\end{verbatim}

\textbf{Negation (with ADD)}

\begin{lstlisting}[language=assembly]
; Negate R1 (two's complement)
MVN R1, R1           ; Invert all bits
ADD R1, R1, #1       ; Add 1
; Now R1 = -R1 (original)
\end{verbatim}

\subsubsection{Shift Operations}

\textbf{Logical Shift Left (LSL)}

\begin{lstlisting}[language=assembly]
LSL Rd, Rn, #shift   ; Rd = Rn << shift
MOV Rd, Rn, LSL #shift
\end{verbatim}

\begin{itemize}
\item Shifts bits left, fills right with zeros
\item Each shift left multiplies by 2
\item Example: 0b00001010 LSL 2 = 0b00101000

\textbf{Logical Shift Right (LSR)}

\begin{lstlisting}[language=assembly]
LSR Rd, Rn, #shift   ; Rd = Rn >> shift (unsigned)
MOV Rd, Rn, LSR #shift
\end{verbatim}

\begin{itemize}
\item Shifts bits right, fills left with zeros
\item Each shift right divides by 2 (unsigned)
\item Example: 0b10100000 LSR 2 = 0b00101000

\textbf{Arithmetic Shift Right (ASR)}

\begin{lstlisting}[language=assembly]
ASR Rd, Rn, #shift   ; Rd = Rn >> shift (signed)
\end{verbatim}

\begin{itemize}
\item Shifts bits right, fills left with sign bit
\item Preserves sign for signed division
\item Example: 0b11110000 ASR 2 = 0b11111100 (sign preserved)

\textbf{Rotate Right (ROR)}

\begin{lstlisting}[language=assembly]
ROR Rd, Rn, #shift   ; Rotate Rn right by shift
\end{verbatim}

\begin{itemize}
\item Bits shifted out right reappear at left
\item No information lost
\item Example: 0b10000001 ROR 1 = 0b11000000

\textbf{Common Shift Applications}

\textbf{Fast Multiplication/Division by Powers of 2}

\begin{lstlisting}[language=assembly]
LSL R0, R1, #3       ; R0 = R1 × 8 (2^3)
LSR R0, R1, #2       ; R0 = R1 / 4 (unsigned)
ASR R0, R1, #2       ; R0 = R1 / 4 (signed)
\end{verbatim}

\textbf{Bit Extraction}

\begin{lstlisting}[language=assembly]
; Extract bits 8-11 from R1
LSR R0, R1, #8       ; Shift bits 8-11 to bits 0-3
AND R0, R0, #0xF     ; Mask to keep only 4 bits
\end{verbatim}

\textbf{Bit Positioning}

\begin{lstlisting}[language=assembly]
; Move bit 0 to bit 7
LSL R0, R1, #7       ; Shift left 7 positions
AND R0, R0, #0x80    ; Keep only bit 7
\end{verbatim}

\subsection{Practical Bit Manipulation Examples}

\subsubsection{Extracting Bit Fields}

\textbf{Extract bits 16-23}

\begin{lstlisting}[language=assembly]
LSR R0, R1, #16      ; Shift right to position
AND R0, R0, #0xFF    ; Mask to 8 bits
\end{verbatim}

\textbf{Extract bits 4-9 (6 bits)}

\begin{lstlisting}[language=assembly]
LSR R0, R1, #4       ; Shift to position 0
AND R0, R0, #0x3F    ; Mask to 6 bits (0b111111)
\end{verbatim}

\subsubsection{Setting and Clearing Bits}

\textbf{Set bits 8-15}

\begin{lstlisting}[language=assembly]
ORR R1, R1, #0xFF00  ; Set bits 8-15
\end{verbatim}

\textbf{Clear bits 16-23}

\begin{lstlisting}[language=assembly]
LDR R0, =0xFF00FFFF  ; Mask with bits 16-23 clear
AND R1, R1, R0       ; Clear bits 16-23 of R1
\end{verbatim}

\textbf{Toggle bits 0-7}

\begin{lstlisting}[language=assembly]
EOR R1, R1, #0xFF    ; Toggle lower byte
\end{verbatim}

\subsubsection{Checking Flags}

\textbf{Check if any of bits 4-7 are set}

\begin{lstlisting}[language=assembly]
AND R2, R1, #0xF0    ; Mask bits 4-7
CMP R2, #0           ; Check if zero
BNE bits_set         ; Branch if any bit was set
\end{verbatim}

\textbf{Check if specific pattern matches}

\begin{lstlisting}[language=assembly]
; Check if bits 8-11 are 0b1010
LSR R0, R1, #8       ; Position bits
AND R0, R0, #0xF     ; Mask 4 bits
CMP R0, #0xA         ; Compare with 0b1010
BEQ pattern_match
\end{verbatim}

\subsubsection{Color Packing/Unpacking}

\textbf{Pack RGB values (8 bits each)}

\begin{lstlisting}[language=assembly]
; R0 = Red, R1 = Green, R2 = Blue
LSL R1, R1, #8       ; Green << 8
LSL R2, R2, #16      ; Blue << 16
ORR R3, R0, R1       ; Combine Red and Green
ORR R3, R3, R2       ; Combine with Blue
; R3 now contains 0x00BBGGRR
\end{verbatim}

\textbf{Unpack RGB values}

\begin{lstlisting}[language=assembly]
; R0 contains 0x00BBGGRR
AND R1, R0, #0xFF      ; Extract Red
LSR R2, R0, #8
AND R2, R2, #0xFF      ; Extract Green
LSR R3, R0, #16
AND R3, R3, #0xFF      ; Extract Blue
\end{verbatim}

\subsection{Key Takeaways}

\begin{enumerate}
\item \textbf{Unsigned binary integers} represent values from 0 to 2^N - 1 using N bits.

\begin{enumerate}
\item \textbf{Two's complement} represents signed integers, with MSB as sign bit and range -(2^(N-1)) to +(2^(N-1) - 1).

\begin{enumerate}
\item \textbf{Sign extension} preserves value when expanding narrower signed values to wider registers.

\begin{enumerate}
\item \textbf{Hexadecimal notation} provides compact representation with one hex digit per 4 binary bits.

\begin{enumerate}
\item \textbf{ARM instructions are fixed 32-bit length}, simplifying fetch/decode but limiting immediate values.

\begin{enumerate}
\item \textbf{Data processing format} includes condition, opcode, source/destination registers, and operand.

\begin{enumerate}
\item \textbf{Data transfer format} specifies load/store, byte/word, addressing mode, and offset.

\begin{enumerate}
\item \textbf{Immediate encoding} uses 8-bit value + 4-bit rotation, limiting which constants can be encoded directly.

\begin{enumerate}
\item \textbf{Bitwise AND} used for masking (extracting specific bits) and clearing bits.

10. \textbf{Bitwise OR} used for setting specific bits and combining values.

11. \textbf{Bitwise XOR} used for toggling bits, fast zero, and comparisons.

12. \textbf{Shift operations} enable fast multiplication/division by powers of 2 and bit positioning.

13. \textbf{Bit manipulation} is fundamental for low-level programming, hardware control, and optimization.

14. \textbf{Understanding encoding} helps write efficient assembly and debug machine code issues.

\subsection{Summary}

Number representation and instruction encoding form the foundation of low-level programming. Two's complement enables efficient signed arithmetic with simple hardware, while sign extension preserves values across different data sizes. ARM's fixed 32-bit instruction format provides regularity but imposes constraints on immediate values, solved through clever encoding schemes. Logical operations—AND, OR, XOR, and NOT—combined with shift operations, provide powerful tools for bit manipulation essential in systems programming, embedded development, and performance optimization. Mastering these concepts enables efficient assembly programming and deeper understanding of how high-level operations translate to machine instructions. These fundamentals prepare us for more complex topics including branching, function calls, and memory management.

% \section{Lecture 6: Branching and Control Flow}

\emph{By Dr. Kisaru Liyanage}

\subsection{Introduction}

Control flow is what distinguishes computers from simple calculators—the ability to make decisions and alter execution based on conditions. This lecture explores conditional operations and branching in ARM assembly, covering comparison instructions, conditional branches, loop implementation, and PC-relative addressing. Understanding these mechanisms is essential for translating high-level control structures (if statements, loops) into assembly code and for comprehending how processors implement dynamic program behavior.

\subsection{Fundamentals of Conditional Execution}

\subsubsection{Decision-Making in Computers}

\textbf{What Makes Computers Powerful}

\begin{itemize}
\item Ability to make decisions based on data
\item Execute different instructions depending on conditions
\item Implement if statements, loops, and function calls
\item Respond dynamically to input and computed values
\end{itemize}

\textbf{Control Flow Concepts}

\begin{itemize}
\item \textbf{Sequential execution}: Default behavior (PC += 4)
\item \textbf{Conditional branching}: Jump if condition is true
\item \textbf{Unconditional branching}: Always jump
\item \textbf{Function calls}: Branch with return address saving
\end{itemize}

\subsubsection{Program Status Register (PSR)}

\textbf{Status Flags}

\begin{itemize}
\item \textbf{N (Negative)}: Set if result is negative (bit 31 = 1)
\item \textbf{Z (Zero)}: Set if result is zero
\item \textbf{C (Carry)}: Set if unsigned overflow occurred
\item \textbf{V (oVerflow)}: Set if signed overflow occurred
\end{itemize}

\textbf{How Flags Are Set}

\begin{itemize}
\item Comparison instructions (CMP, CMN, TST, TEQ)
\item Arithmetic/logic instructions with S suffix (ADDS, SUBS)
\item Flags reflect the result of the operation
\item Used by subsequent conditional branches
\end{itemize}

\textbf{Example}

\begin{verbatim}
CMP R1, R2           ; Compare R1 and R2 (computes R1 - R2)
                      ; Sets flags based on result
\end{verbatim}

If R1 = 5, R2 = 3:

\begin{itemize}
\item Result of R1 - R2 = 2 (positive, non-zero)
\item N = 0 (not negative)
\item Z = 0 (not zero)
\item C = 1 (no borrow needed)
\item V = 0 (no overflow)
\end{itemize}

\subsection{Comparison Instructions}

\subsubsection{Compare (CMP)}

\textbf{Syntax}

\begin{verbatim}
CMP Rn, Rm           ; Compare Rn with Rm
CMP Rn, #imm         ; Compare Rn with immediate
\end{verbatim}

\textbf{Operation}

\begin{itemize}
\item Performs Rn - Rm (subtraction)
\item Updates PSR flags based on result
\item Does NOT store the result
\item Does NOT modify any register
\end{itemize}

\textbf{Example Usage}

\begin{verbatim}
MOV R1, #10
MOV R2, #5
CMP R1, R2           ; Compares 10 with 5
                      ; Result: 10 - 5 = 5 (positive, non-zero)
                      ; Z = 0, N = 0
\end{verbatim}

\subsubsection{Compare Negative (CMN)}

\textbf{Syntax}

\begin{verbatim}
CMN Rn, Rm           ; Compare Negative
CMN Rn, #imm
\end{verbatim}

\textbf{Operation}

\begin{itemize}
\item Performs Rn + Rm (addition)
\item Updates PSR flags
\item Equivalent to CMP Rn, -Rm
\item Useful for checking if sum equals zero
\end{itemize}

\subsubsection{Test (TST)}

\textbf{Syntax}

\begin{verbatim}
TST Rn, Rm           ; Test bits
TST Rn, #imm
\end{verbatim}

\textbf{Operation}

\begin{itemize}
\item Performs Rn AND Rm (bitwise AND)
\item Updates PSR flags
\item Result not stored
\item Used to test if specific bits are set
\end{itemize}

\textbf{Example: Check if bit 5 is set}

\begin{verbatim}
TST R1, #0x20        ; Test bit 5
BEQ bit_clear        ; Branch if bit was clear (Z=1)
\end{verbatim}

\subsubsection{Test Equivalence (TEQ)}

\textbf{Syntax}

\begin{verbatim}
TEQ Rn, Rm           ; Test Equivalence
TEQ Rn, #imm
\end{verbatim}

\textbf{Operation}

\begin{itemize}
\item Performs Rn XOR Rm (exclusive OR)
\item Updates PSR flags
\item Z=1 if values are equal
\item Used to compare values without affecting C or V flags
\end{itemize}

\subsection{Conditional Branch Instructions}

\subsubsection{Branch if Equal (BEQ)}

\textbf{Syntax}

\begin{verbatim}
BEQ label            ; Branch if equal (Z=1)
\end{verbatim}

\textbf{Condition}

\begin{itemize}
\item Branches if Zero flag is set (Z = 1)
\item Typically used after CMP to check equality
\end{itemize}

\textbf{Example}

\begin{verbatim}
CMP R1, R2           ; Compare R1 and R2
BEQ equal_label      ; Jump to equal_label if R1 == R2
; Code if not equal
equal_label:
; Code if equal
\end{verbatim}

\subsubsection{Branch if Not Equal (BNE)}

\textbf{Syntax}

\begin{verbatim}
BNE label            ; Branch if not equal (Z=0)
\end{verbatim}

\textbf{Condition}

\begin{itemize}
\item Branches if Zero flag is clear (Z = 0)
\item Opposite of BEQ
\end{itemize}

\textbf{Example}

\begin{verbatim}
CMP R3, #0
BNE not_zero         ; Jump if R3 != 0
; Code if R3 is zero
not_zero:
; Code if R3 is non-zero
\end{verbatim}

\subsubsection{Signed Comparison Branches}

\textbf{Branch if Greater or Equal (BGE)}

\begin{verbatim}
BGE label            ; Branch if Rn >= Rm (signed)
                      ; Condition: N == V
\end{verbatim}

\textbf{Branch if Less Than (BLT)}

\begin{verbatim}
BLT label            ; Branch if Rn < Rm (signed)
                      ; Condition: N != V
\end{verbatim}

\textbf{Branch if Greater Than (BGT)}

\begin{verbatim}
BGT label            ; Branch if Rn > Rm (signed)
                      ; Condition: Z==0 AND N==V
\end{verbatim}

\textbf{Branch if Less or Equal (BLE)}

\begin{verbatim}
BLE label            ; Branch if Rn <= Rm (signed)
                      ; Condition: Z==1 OR N!=V
\end{verbatim}

\textbf{Example}

\begin{verbatim}
CMP R1, R2
BGE greater_equal    ; Branch if R1 >= R2 (signed)
; Code if R1 < R2
greater_equal:
; Code if R1 >= R2
\end{verbatim}

\subsubsection{Unsigned Comparison Branches}

\textbf{Branch if Higher or Same (BHS)} (also called BCS - Branch if Carry Set)

\begin{verbatim}
BHS label            ; Branch if Rn >= Rm (unsigned)
                      ; Condition: C == 1
\end{verbatim}

\textbf{Branch if Lower (BLO)} (also called BCC - Branch if Carry Clear)

\begin{verbatim}
BLO label            ; Branch if Rn < Rm (unsigned)
                      ; Condition: C == 0
\end{verbatim}

\textbf{Branch if Higher (BHI)}

\begin{verbatim}
BHI label            ; Branch if Rn > Rm (unsigned)
                      ; Condition: C==1 AND Z==0
\end{verbatim}

\textbf{Branch if Lower or Same (BLS)}

\begin{verbatim}
BLS label            ; Branch if Rn <= Rm (unsigned)
                      ; Condition: C==0 OR Z==1
\end{verbatim}

\subsubsection{Signed vs. Unsigned Example}

\textbf{Key Difference}

\begin{verbatim}
MOV R0, #0xFFFFFFFF  ; R0 = -1 (signed) or 4,294,967,295 (unsigned)
MOV R1, #1           ; R1 = 1
CMP R0, R1

BLO lower_unsigned   ; BRANCH NOT TAKEN
                      ; Unsigned: 4,294,967,295 > 1

BLT less_signed      ; BRANCH TAKEN
                      ; Signed: -1 < 1
\end{verbatim}

\textbf{When to Use Each}

\begin{itemize}
\item \textbf{Signed}: Comparing integers that can be negative (temperatures, offsets, differences)
\item \textbf{Unsigned}: Comparing addresses, array indices, sizes, counts
\end{itemize}

\subsubsection{Unconditional Branch}

\textbf{Syntax}

\begin{verbatim}
B label              ; Branch always
\end{verbatim}

\textbf{Purpose}

\begin{itemize}
\item Jump without checking any condition
\item Skip code sections
\item Implement infinite loops
\item Return to loop start
\end{itemize}

\textbf{Example}

\begin{verbatim}
B end                ; Skip this section
; Code to skip
end:
; Continue execution here
\end{verbatim}

\subsection{Labels in Assembly}

\subsubsection{Label Definition}

\textbf{Purpose}

\begin{itemize}
\item Mark specific instruction locations
\item Provide symbolic names for addresses
\item Enable branches and data references
\end{itemize}

\textbf{Syntax}

\begin{verbatim}
label:               ; Label definition (note colon)
    MOV R0, #1      ; Instruction at this label
\end{verbatim}

\textbf{Naming Rules}

\begin{itemize}
\item Can be almost any identifier
\item Common conventions: loop, exit, done, L1, L2
\item Cannot conflict with instruction mnemonics
\item Case-sensitive
\end{itemize}

\textbf{Example}

\begin{verbatim}
start:
    MOV R0, #0
loop:
    ADD R0, R0, #1
    CMP R0, #10
    BLT loop         ; Branch to loop label
    B start          ; Branch to start label
\end{verbatim}

\subsubsection{Label Resolution}

\textbf{Assembly Process}

\begin{enumerate}
\item First pass: Record label addresses
\item Second pass: Replace labels with addresses
\item Calculate offsets for PC-relative branches
\end{enumerate}

\textbf{Virtual Addresses}

\begin{itemize}
\item Assembler assigns virtual addresses from 0
\item First instruction: address 0
\item Second instruction: address 4
\item Third instruction: address 8
\item Physical addresses determined at load time
\end{itemize}

\subsection{Implementing Control Structures}

\subsubsection{If Statement}

\textbf{C Code}

\begin{verbatim}
if (i == j)
    f = g + h;
else
    f = g - h;
\end{verbatim}

\textbf{ARM Assembly (Method 1: Branch on False)}

\begin{verbatim}
    CMP R3, R4       ; Compare i (R3) and j (R4)
    BNE else         ; Branch to else if not equal
    ADD R0, R1, R2   ; f = g + h (then clause)
    B exit           ; Skip else clause

else:
    SUB R0, R1, R2   ; f = g - h (else clause)
exit:
    ; Continue...
\end{verbatim}

\textbf{ARM Assembly (Method 2: Conditional Execution)}

\begin{verbatim}
    CMP R3, R4       ; Compare i and j
    ADDEQ R0, R1, R2 ; f = g + h (executed only if equal)
    SUBNE R0, R1, R2 ; f = g - h (executed only if not equal)
\end{verbatim}

\subsubsection{If-Else Ladder}

\textbf{C Code}

\begin{verbatim}
if (x < 0)
    result = -1;
else if (x == 0)
    result = 0;
else
    result = 1;
\end{verbatim}

\textbf{ARM Assembly}

\begin{verbatim}
    CMP R1, #0       ; Compare x with 0
    BLT negative     ; Branch if x < 0
    BEQ zero         ; Branch if x == 0
    ; x > 0
    MOV R0, #1
    B done

negative:
    MOV R0, #-1
    B done
zero:
    MOV R0, #0
done:
    ; Continue...
\end{verbatim}

\subsubsection{While Loop}

\textbf{C Code}

\begin{verbatim}
while (i < n) {
    sum += i;
    i++;
}
\end{verbatim}

\textbf{ARM Assembly}

\begin{verbatim}
loop:
    CMP R1, R2       ; Compare i (R1) with n (R2)
    BGE end_loop     ; Exit if i >= n
    ADD R0, R0, R1   ; sum = sum + i
    ADD R1, R1, #1   ; i++
    B loop           ; Branch back to loop start
end_loop:
    ; Continue...
\end{verbatim}

\subsubsection{For Loop}

\textbf{C Code}

\begin{verbatim}
for (i = 0; i < 10; i++) {
    sum += i;
}
\end{verbatim}

\textbf{ARM Assembly}

\begin{verbatim}
    MOV R1, #0       ; i = 0 (initialization)

for_loop:
    CMP R1, #10      ; Compare i with 10
    BGE end_for      ; Exit if i >= 10
    ADD R0, R0, R1   ; sum = sum + i (loop body)
    ADD R1, R1, #1   ; i++ (increment)
    B for_loop       ; Branch back to loop start
end_for:
    ; Continue...
\end{verbatim}

\subsubsection{Do-While Loop}

\textbf{C Code}

\begin{verbatim}
do {
    sum += i;
    i++;
} while (i < n);
\end{verbatim}

\textbf{ARM Assembly}

\begin{verbatim}
do_loop:
    ADD R0, R0, R1   ; sum = sum + i (loop body first)
    ADD R1, R1, #1   ; i++
    CMP R1, R2       ; Compare i with n
    BLT do_loop      ; Branch back if i < n
    ; Continue...
\end{verbatim}

\textbf{Key Difference from While}

\begin{itemize}
\item Body executes at least once
\item Condition checked at end, not beginning
\end{itemize}

\subsection{Array Access in Loops}

\subsubsection{Static Array Indexing}

\textbf{C Code}

\begin{verbatim}
while (save[i] == k)
    i++;
\end{verbatim}

\textbf{ARM Assembly}

\begin{verbatim}
    ; R6 = base address of save array
    ; R3 = i (index)
    ; R5 = k (comparison value)

loop:
    ADD R12, R6, R3, LSL #2  ; address = base + (i * 4)
    LDR R0, [R12, #0]        ; R0 = save[i]
    CMP R0, R5               ; Compare save[i] with k
    BNE exit                 ; Exit if not equal
    ADD R3, R3, #1           ; i++
    B loop                   ; Continue loop
exit:
    ; Continue...
\end{verbatim}

\textbf{Dynamic Offset Calculation}

\begin{itemize}
\item \texttt{R3, LSL \#2} means R3 × 4 (shift left 2 = multiply by 4)
\item Words are 4 bytes, so array element i is at base + (i × 4)
\item Efficient: shift is faster than multiplication
\end{itemize}

\subsubsection{Array Traversal}

\textbf{C Code}

\begin{verbatim}
int sum = 0;
for (int i = 0; i < 10; i++) {
    sum += arr[i];
}
\end{verbatim}

\textbf{ARM Assembly}

\begin{verbatim}
    LDR R6, =arr     ; R6 = base address of array
    MOV R0, #0       ; sum = 0
    MOV R1, #0       ; i = 0

loop:
    CMP R1, #10
    BGE done
    ADD R12, R6, R1, LSL #2  ; address = base + i*4
    LDR R2, [R12]            ; R2 = arr[i]
    ADD R0, R0, R2           ; sum += arr[i]
    ADD R1, R1, #1           ; i++
    B loop
done:
    ; R0 contains sum
\end{verbatim}

\subsection{PC-Relative Addressing}

\subsubsection{Branch Instruction Encoding}

\textbf{32-Bit Format}

\begin{verbatim}
[Cond][1010][Offset]
 4-bit 4-bit 24-bit
\end{verbatim}

\textbf{Fields}

\begin{itemize}
\item \textbf{Cond}: Condition code (EQ, NE, LT, etc.)
\item \textbf{1010}: Fixed format field for branch
\item \textbf{Offset}: 24-bit signed offset
\end{itemize}

\subsubsection{Address Calculation}

\textbf{Problem with Absolute Addressing}

\begin{itemize}
\item 24 bits can address $2^{24}$ = 16 MB
\item Limits program size to 16 MB
\item Fixed addresses complicate relocation
\end{itemize}

\textbf{PC-Relative Solution}

\begin{itemize}
\item Store offset from current PC, not absolute address
\item Target = PC + offset
\item Can branch ±16 MB from current instruction
\item Total program can exceed 16 MB
\end{itemize}

\textbf{Offset Calculation}

\begin{verbatim}
Offset = (Target Address - PC) / 4
\end{verbatim}

\textbf{Why Divide by 4?}

\begin{itemize}
\item All instructions are 4-byte aligned
\item Least significant 2 bits always 00
\item Omit these bits in encoding
\item Effective range: ±64 MB (24-bit offset × 4)
\end{itemize}

\textbf{Example}

\begin{verbatim}
Current PC: 0x1000
Target: 0x1020
Offset = (0x1020 - 0x1000) / 4 = 0x20 / 4 = 8 instructions

Encoded offset in branch instruction: 8
At execution: PC = 0x1000 + (8 × 4) = 0x1020
\end{verbatim}

\subsubsection{Advantages of PC-Relative}

\textbf{Position-Independent Code}

\begin{itemize}
\item Code can load at any address
\item Branches remain correct regardless of location
\item Essential for libraries and shared code
\end{itemize}

\textbf{Simplified Linking}

\begin{itemize}
\item Linker doesn't need to patch all branches
\item Only external function calls need adjustment
\end{itemize}

\textbf{Branch Locality}

\begin{itemize}
\item Most branches are to nearby instructions
\item PC-relative naturally handles this case
\item Absolute addressing wastes bits for nearby targets
\end{itemize}

\subsection{Conditional Execution (Alternative to Branching)}

\subsubsection{Conditional Instruction Suffixes}

\textbf{Concept}

\begin{itemize}
\item Add condition code to instruction mnemonic
\item Instruction executes only if condition is true
\item Otherwise, instruction is skipped (NOP)
\end{itemize}

\textbf{Available Suffixes}

\begin{itemize}
\item EQ (equal), NE (not equal)
\item GT, LT, GE, LE (signed comparisons)
\item HI, LO, HS, LS (unsigned comparisons)
\item Many others (see ARM documentation)
\end{itemize}

\textbf{Examples}

\begin{verbatim}
CMP R1, R2
ADDEQ R0, R3, R4     ; Execute ADD only if R1 == R2
SUBNE R0, R3, R4     ; Execute SUB only if R1 != R2
MOVGT R5, #10        ; Execute MOV only if R1 > R2
\end{verbatim}

\subsubsection{Conditional Execution Example}

\textbf{C Code}

\begin{verbatim}
if (a == b)
    max = a;
else
    max = b;
\end{verbatim}

\textbf{Method 1: Branching}

\begin{verbatim}
    CMP R1, R2       ; Compare a and b
    BNE else
    MOV R0, R1       ; max = a
    B done

else:
    MOV R0, R2       ; max = b
done:
\end{verbatim}

\textbf{Method 2: Conditional Execution}

\begin{verbatim}
    CMP R1, R2       ; Compare a and b
    MOVEQ R0, R1     ; max = a (if equal)
    MOVNE R0, R2     ; max = b (if not equal)
\end{verbatim}

\subsubsection{Advantages and Limitations}

\textbf{Advantages}

\begin{itemize}
\item More compact code (fewer instructions)
\item No branch misprediction penalty
\item Faster for simple conditions
\item Clearer intent in some cases
\end{itemize}

\textbf{Limitations}

\begin{itemize}
\item Only works for simple, short sequences
\item Cannot conditionally execute blocks of code
\item All conditional instructions must fit in pipeline
\item May execute both paths (but discard one result)
\end{itemize}

\textbf{When to Use}

\begin{itemize}
\item Simple assignments
\item Min/max operations
\item Short computations with single result
\item Performance-critical paths where branches hurt
\end{itemize}

\subsection{Basic Blocks}

\subsubsection{Definition}

\textbf{Basic Block Characteristics}

\begin{itemize}
\item Sequence of instructions with:
  \begin{itemize}
  \item No embedded branches (except possibly at end)
  \item No branch targets (except possibly at beginning)
  \end{itemize}
\item Executed atomically: all or nothing
\item Single entry point, single exit point
\end{itemize}

\textbf{Example}

\begin{verbatim}
; Basic Block 1 (entry point)
    MOV R0, #0
    MOV R1, #10
    CMP R1, #10
    BNE block2       ; Exit point of block 1

; Basic Block 2 (entry and exit point)
block2:
    ADD R0, R0, #1
    CMP R0, R1
    BLT block2       ; Exit point of block 2
\end{verbatim}

\subsubsection{Importance in Compilation}

\textbf{Compiler Optimizations}

\begin{itemize}
\item Identify basic blocks for analysis
\item Optimize within blocks (register allocation, scheduling)
\item Build control flow graph from blocks
\item Apply inter-block optimizations
\end{itemize}

\textbf{Processor Optimizations}

\begin{itemize}
\item Predict block execution
\item Prefetch instructions in block
\item Schedule instructions more aggressively
\item Reduce branch overhead
\end{itemize}

\subsection{Key Takeaways}

\begin{enumerate}
\item \textbf{Conditional execution} distinguishes computers from calculators, enabling decision-making and dynamic behavior.

\item \textbf{CMP instruction} sets PSR flags by performing subtraction without storing the result.

\item \textbf{Conditional branches} (BEQ, BNE, BGE, BLT, etc.) check PSR flags to decide whether to jump.

\item \textbf{Signed vs. unsigned branches} interpret the same bit patterns differently based on context.

\item \textbf{Labels} provide symbolic names for addresses, enabling readable branch targets.

\item \textbf{If statements} translate to compare + conditional branch + unconditional branch to skip alternate path.

\item \textbf{Loops} use compare + conditional branch (to exit) + unconditional branch (to continue).

\item \textbf{Array access} in loops uses dynamic offset calculation with shifts (LSL \#2 for word arrays).

\item \textbf{PC-relative addressing} stores branch offset from current PC, enabling position-independent code and large programs.

\item \textbf{Word-based offsets} effectively quadruple branch range by encoding instruction count instead of byte offset.

\item \textbf{Conditional execution} provides alternative to branching for simple cases, improving performance and code density.

\item \textbf{Basic blocks} are atomic instruction sequences used by compilers and processors for optimization.

\item \textbf{Branch locality} means most branches target nearby instructions, making PC-relative addressing natural and efficient.
\end{enumerate}

\subsection{Summary}

Branching and conditional execution form the foundation of program control flow, translating high-level constructs like if statements and loops into machine instructions. The ARM architecture provides a rich set of conditional branches for both signed and unsigned comparisons, enabling efficient implementation of diverse control structures. Understanding the distinction between comparison (which sets flags) and branching (which checks flags) is essential for correct assembly programming. PC-relative addressing solves program size limitations while enabling position-independent code, and conditional execution offers a performant alternative to branching for simple cases. Mastering these concepts is crucial for translating algorithms into assembly code, optimizing performance-critical sections, and understanding how processors implement dynamic program behavior. These fundamentals prepare us for more advanced topics including function calls, stack management, and processor pipelining.

% \section{Lecture 7: Function Call and Return}

\emph{By Dr. Kisaru Liyanage}

\subsection{Introduction}

Function calling is a fundamental mechanism that enables modular programming and code reuse. This lecture explores how ARM assembly implements function calls, covering parameter passing, return value handling, the call stack, register preservation conventions, and recursion. Understanding these mechanisms is essential for translating high-level function-based programs into assembly and for comprehending how processors manage execution context across function boundaries.

\subsection{Function Calling Fundamentals}

\subsubsection{Function Calling Steps}

\textbf{Complete Call Sequence}

\begin{enumerate}
\item \textbf{Place parameters} in argument registers (R0-R3)
\item \textbf{Transfer control} to callee function using BL
\item \textbf{Acquire stack storage} for temporary values
\item \textbf{Back up registers} that need preservation (R4-R11)
\item \textbf{Perform function operations} (the actual work)
\item \textbf{Place result} in return register (R0)
\item \textbf{Restore backed-up registers} from stack
\item \textbf{Return to caller} using MOV PC, LR
\end{enumerate}

\textbf{Why This Complexity?}

\begin{itemize}
\item Enables nested and recursive function calls
\item Protects caller's data in registers
\item Provides local storage for function variables
\item Supports arbitrary call depth
\end{itemize}

\subsubsection{Why Use Functions?}

\textbf{Benefits}

\begin{itemize}
\item \textbf{Code reuse}: Write once, call many times
\item \textbf{Modularity}: Break complex problems into manageable pieces
\item \textbf{Abstraction}: Hide implementation details
\item \textbf{Maintainability}: Easier to debug and modify
\end{itemize}

\textbf{Example}

\begin{verbatim}
int add(int a, int b) {
    return a + b;
}

int main() {
    int result = add(5, 3);  // Function call
}
\end{verbatim}

\subsection{ARM Register Conventions}

\subsubsection{Register Usage Rules}

\textbf{Register Classification}

\begin{verbatim}
R0-R1:   Arguments and return results
         - Caller does NOT expect these preserved
         - Scratch registers

R2-R3:   Additional arguments
         - Also scratch registers
         - Caller does NOT expect preservation

R4-R11:  Local variables
         - MUST be preserved across function calls
         - Callee saves if it uses these registers

R12:     Intra-procedure-call scratch register
         - Can be corrupted by function calls
         - Not preserved

R13 (SP): Stack Pointer
         - Points to top of stack
         - MUST always be valid

R14 (LR): Link Register
         - Stores return address
         - Set by BL instruction

R15 (PC): Program Counter
         - Next instruction address
         - Modified to return from function
\end{verbatim}

\subsubsection{Shared Register File}

\textbf{Key Concept}

\begin{itemize}
\item ALL functions share the SAME 16 registers
\item No separate register sets per function
\item Registers are a shared resource requiring careful management
\end{itemize}

\textbf{Implications}

\begin{itemize}
\item Functions must coordinate register usage
\item Conventions prevent conflicts
\item Callee must preserve certain registers (R4-R11)
\item Caller can assume R4-R11 unchanged after call
\end{itemize}

\textbf{Example Scenario}

\begin{verbatim}
main:
    MOV R4, #10      ; main uses R4
    MOV R0, #5       ; Pass argument
    BL function      ; Call function
    ; R4 still contains 10 (guaranteed)
    ADD R5, R4, R0   ; Use preserved R4 and return value

function:
    ; Must preserve R4 if we use it
    ; Can freely modify R0-R3, R12
    MOV R0, #20      ; Return value
    MOV PC, LR       ; Return
\end{verbatim}

\subsection{Function Call Instructions}

\subsubsection{Branch and Link (BL)}

\textbf{Syntax}

\begin{verbatim}
BL function_label    ; Branch and Link
\end{verbatim}

\textbf{Operation}

\begin{enumerate}
\item \textbf{Save return address}: LR = address of next instruction
\item \textbf{Jump to function}: PC = function\_label address
\end{enumerate}

\textbf{Example}

\begin{verbatim}
    MOV R0, #10      ; Address: 0x1000
    BL fun           ; Address: 0x1004
    ADD R1, R0, #5   ; Address: 0x1008 (return point)


fun:
    ; LR contains 0x1008 (address after BL)
    ; Function code here
    MOV PC, LR       ; Return to 0x1008
\end{verbatim}

\textbf{Why "Link"?}

\begin{itemize}
\item Creates a "link" back to caller
\item LR provides the connection
\item Enables function to return
\end{itemize}

\subsubsection{Return from Function}

\textbf{Basic Return}

\begin{verbatim}
MOV PC, LR           ; Copy LR to PC
\end{verbatim}

\textbf{Operation}

\begin{itemize}
\item PC = LR (jump to return address)
\item Execution continues at instruction after BL
\item Simple and fast
\end{itemize}

\textbf{Alternative (older ARM)}

\begin{verbatim}
BX LR                ; Branch and Exchange
\end{verbatim}

\subsection{Parameter Passing}

\subsubsection{Using R0-R3}

\textbf{Convention}

\begin{itemize}
\item First 4 arguments in R0-R3
\item Arguments loaded before BL instruction
\item Callee reads R0-R3 to get parameters
\end{itemize}

\textbf{Example: Two Parameters}

\begin{verbatim}
int multiply(int a, int b) {
    return a * b;
}

int result = multiply(6, 7);
\end{verbatim}

\textbf{ARM Assembly}

\begin{verbatim}
    MOV R0, #6       ; First argument (a)
    MOV R1, #7       ; Second argument (b)
    BL multiply      ; Call function
    ; R0 now contains result (42)


multiply:
    MUL R0, R0, R1   ; R0 = R0 * R1
    MOV PC, LR       ; Return
\end{verbatim}

\subsubsection{More Than 4 Arguments}

\textbf{Solution: Use Stack}

\begin{itemize}
\item Arguments 1-4 in R0-R3
\item Additional arguments pushed to stack
\item Callee reads from stack
\end{itemize}

\textbf{Example: 6 Arguments}

\begin{verbatim}
int sum6(int a, int b, int c, int d, int e, int f) {
    return a + b + c + d + e + f;
}
\end{verbatim}

\textbf{ARM Assembly}

\begin{verbatim}
    MOV R0, #1       ; arg1
    MOV R1, #2       ; arg2
    MOV R2, #3       ; arg3
    MOV R3, #4       ; arg4
    MOV R4, #5
    MOV R5, #6
    SUB SP, SP, #8   ; Space for 2 more args
    STR R4, [SP, #0] ; arg5 on stack
    STR R5, [SP, #4] ; arg6 on stack
    BL sum6
    ADD SP, SP, #8   ; Clean up stack


sum6:
    ; R0-R3 have first 4 args
    ; Load arg5 and arg6 from stack
    LDR R4, [SP, #0] ; arg5
    LDR R5, [SP, #4] ; arg6
    ADD R0, R0, R1
    ADD R0, R0, R2
    ADD R0, R0, R3
    ADD R0, R0, R4
    ADD R0, R0, R5
    MOV PC, LR
\end{verbatim}

\subsection{Return Values}

\subsubsection{Primary Return Register (R0)}

\textbf{Convention}

\begin{itemize}
\item Result placed in R0
\item Caller reads R0 after function returns
\item Works for 32-bit values
\end{itemize}

\textbf{Example}

\begin{verbatim}
add:
    ADD R0, R0, R1   ; R0 = R0 + R1
    MOV PC, LR       ; Return with result in R0

main:
    MOV R0, #10
    MOV R1, #20
    BL add           ; Call function
    ; R0 now contains 30
\end{verbatim}

\subsubsection{64-Bit Return Values}

\textbf{Convention}

\begin{itemize}
\item Lower 32 bits in R0
\item Upper 32 bits in R1
\item Example: 64-bit integer or two 32-bit values
\end{itemize}

\textbf{Example}

\begin{verbatim}
long long multiply64(int a, int b) {
    return (long long)a * b;
}
\end{verbatim}

\textbf{ARM Assembly}

\begin{verbatim}
multiply64:
    SMULL R0, R1, R0, R1  ; Signed multiply long
    ; R0 = lower 32 bits
    ; R1 = upper 32 bits
    MOV PC, LR
\end{verbatim}

\subsection{The Stack}

\subsubsection{Stack Structure}

\textbf{Definition}

\begin{itemize}
\item Last In, First Out (LIFO) data structure
\item Part of main memory
\item Used for temporary storage
\end{itemize}

\textbf{Characteristics}

\begin{itemize}
\item \textbf{Starts at high address}: Top of memory
\item \textbf{Grows downward}: Toward lower addresses
\item \textbf{Stack Pointer (SP/R13)}: Points to top of stack
\item \textbf{Dynamic size}: Grows and shrinks as needed
\end{itemize}

\textbf{Memory Layout}

\begin{verbatim}
High Address
    +---------+
    |  Stack  |  SP points here
    |         |  (grows downward)
    |         |
    +---------+
    |  Heap   |
    |         |  (grows upward)
    +---------+
    |  Data   |  (static variables)
    +---------+
    |  Text   |  (instructions)
    +---------+
Low Address
\end{verbatim}

\subsubsection{Stack Uses}

\textbf{Primary Purposes}

\begin{enumerate}
\item \textbf{Saving register values} (preserve R4-R11)
\item \textbf{Storing local variables} (arrays, structures)
\item \textbf{Preserving return addresses} (nested calls)
\item \textbf{Extra function arguments} (beyond R0-R3)
\item \textbf{Storing local arrays} that don't fit in registers
\end{enumerate}

\subsection{Stack Operations}

\subsubsection{Allocating Stack Space (Pushing)}

\textbf{Decrement Stack Pointer}

\begin{verbatim}
SUB SP, SP, #4       ; Allocate 4 bytes (1 register)
SUB SP, SP, #12      ; Allocate 12 bytes (3 registers)
\end{verbatim}

\textbf{Why Subtract?}

\begin{itemize}
\item Stack grows toward lower addresses
\item Allocating space moves SP downward
\item Each 32-bit register needs 4 bytes
\end{itemize}

\subsubsection{Storing Values to Stack}

\textbf{Single Register}

\begin{verbatim}
SUB SP, SP, #4       ; Allocate space
STR R4, [SP, #0]     ; Store R4 at top of stack
\end{verbatim}

\textbf{Multiple Registers}

\begin{verbatim}
SUB SP, SP, #12      ; Space for 3 registers
STR R4, [SP, #0]     ; Store R4
STR R5, [SP, #4]     ; Store R5
STR R6, [SP, #8]     ; Store R6
\end{verbatim}

\textbf{Push Multiple (Convenient)}

\begin{verbatim}
PUSH {R4-R6}         ; Allocate and store in one instruction
\end{verbatim}

\subsubsection{Loading Values from Stack}

\textbf{Single Register}

\begin{verbatim}
LDR R4, [SP, #0]     ; Load R4 from stack
ADD SP, SP, #4       ; Release space
\end{verbatim}

\textbf{Multiple Registers}

\begin{verbatim}
LDR R4, [SP, #0]     ; Restore R4
LDR R5, [SP, #4]     ; Restore R5
LDR R6, [SP, #8]     ; Restore R6
ADD SP, SP, #12      ; Release space
\end{verbatim}

\textbf{Pop Multiple}

\begin{verbatim}
POP {R4-R6}          ; Restore and release in one instruction
\end{verbatim}

\subsubsection{Stack Space Lifecycle}

\textbf{Pattern}

\begin{enumerate}
\item \textbf{Allocate}: SUB SP, SP, \#n
\item \textbf{Use}: STR/LDR with [SP, offset]
\item \textbf{Release}: ADD SP, SP, \#n
\end{enumerate}

\textbf{Important: Balance}

\begin{itemize}
\item Every SUB must have corresponding ADD
\item Unbalanced stack causes bugs and crashes
\item SP must be restored before return
\end{itemize}

\subsection{Register Preservation}

\subsubsection{Why Preserve R4-R11?}

\textbf{Problem}

\begin{itemize}
\item All functions share same registers
\item Main function may be using R4-R11
\item Called function needs registers for its work
\item Must not corrupt caller's data
\end{itemize}

\textbf{Solution}

\begin{itemize}
\item Callee saves R4-R11 to stack at function start
\item Uses registers freely during execution
\item Restores R4-R11 from stack before return
\item Caller expects R4-R11 unchanged
\end{itemize}

\subsubsection{Preservation Pattern}

\textbf{Function Template}

\begin{verbatim}
function:
    ; Prologue: Save registers
    SUB SP, SP, #12      ; Allocate space
    STR R4, [SP, #0]     ; Save R4
    STR R5, [SP, #4]     ; Save R5
    STR R6, [SP, #8]     ; Save R6

    ; Function body: Use R4-R6 freely
    ; ...

    ; Epilogue: Restore registers
    LDR R4, [SP, #0]     ; Restore R4
    LDR R5, [SP, #4]     ; Restore R5
    LDR R6, [SP, #8]     ; Restore R6
    ADD SP, SP, #12      ; Release space
    MOV PC, LR           ; Return
\end{verbatim}

\textbf{Optimization}

\begin{itemize}
\item Only preserve registers actually used
\item If function doesn't use R5, don't save/restore it
\item Saves stack space and execution time
\end{itemize}

\subsection{Nested Function Calls (Non-Leaf Functions)}

\subsubsection{The Problem}

\textbf{Leaf Function}

\begin{itemize}
\item Doesn't call other functions
\item LR preserved automatically (not overwritten)
\item Simple return: MOV PC, LR
\end{itemize}

\textbf{Non-Leaf Function}

\begin{itemize}
\item Calls other functions
\item BL overwrites LR with new return address
\item Original LR lost!
\item Cannot return to original caller
\end{itemize}

\textbf{Example Problem}

\begin{verbatim}
main:
    BL funcA         ; LR = address after this BL

funcA:
    ; LR contains return address to main
    BL funcB         ; LR OVERWRITTEN with return to funcA!
    MOV PC, LR       ; Returns to funcA, not main (WRONG!)

funcB:
    MOV PC, LR       ; Correctly returns to funcA
\end{verbatim}

\subsubsection{Solution: Save LR to Stack}

\textbf{Pattern}

\begin{verbatim}
function:
    ; Save LR first!
    SUB SP, SP, #4
    STR LR, [SP, #0]

    ; Now safe to call other functions
    BL other_function

    ; Restore LR before return
    LDR LR, [SP, #0]
    ADD SP, SP, #4
    MOV PC, LR
\end{verbatim}

\textbf{Complete Example}

\begin{verbatim}
main:
    MOV R0, #5
    BL outer         ; LR = return_to_main
    ; Execution returns here

outer:
    SUB SP, SP, #4
    STR LR, [SP, #0] ; Save LR (return_to_main)

    MOV R1, R0
    ADD R0, R0, #10
    BL inner         ; LR = return_to_outer (overwrites!)

    ADD R0, R0, R1
    LDR LR, [SP, #0] ; Restore LR (return_to_main)
    ADD SP, SP, #4
    MOV PC, LR       ; Returns to main

inner:
    MUL R0, R0, R0
    MOV PC, LR       ; Returns to outer
\end{verbatim}

\subsection{Recursion Example: Factorial}

\subsubsection{Factorial Function}

\textbf{C Code}

\begin{verbatim}
int fact(int n) {
    if (n <= 1)
        return 1;
    else
        return n * fact(n-1);
}
\end{verbatim}

\textbf{Key Points}

\begin{itemize}
\item Base case: $n \le 1$, return 1
\item Recursive case: return $n \times$ fact$(n-1)$
\item Each call creates new stack frame
\item Stack unwinds as recursion returns
\end{itemize}

\subsubsection{ARM Assembly Implementation}

\begin{verbatim}
fact:
    ; Save LR and n
    SUB SP, SP, #8
    STR LR, [SP, #4]     ; Save return address
    STR R0, [SP, #0]     ; Save n

    ; Base case: if (n <= 1) return 1
    CMP R0, #1
    BGT recursive
    MOV R0, #1           ; Return 1
    B fact_end

recursive:
    ; Recursive case: n * fact(n-1)
    SUB R0, R0, #1       ; n-1
    BL fact              ; fact(n-1)
    LDR R1, [SP, #0]     ; Restore original n
    MUL R0, R0, R1       ; n * fact(n-1)

fact_end:
    ; Restore and return
    LDR LR, [SP, #4]
    ADD SP, SP, #8
    MOV PC, LR
\end{verbatim}

\subsubsection{Stack Growth During Recursion}

\textbf{Call: fact(3)}

\begin{verbatim}
Initial: SP = 0x1000

fact(3) call:
  SP = 0x0FF8: [LR_main, 3]

fact(2) call:
  SP = 0x0FF0: [LR_fact3, 2]

fact(1) call:
  SP = 0x0FE8: [LR_fact2, 1]

Base case returns 1
Unwinds to fact(2): returns 1*2 = 2
Unwinds to fact(3): returns 2*3 = 6
Returns to main with result 6

Final: SP = 0x1000 (restored)
\end{verbatim}

\textbf{Stack Space Per Call}

\begin{itemize}
\item 8 bytes (LR + n)
\item fact(5) needs 5 * 8 = 40 bytes
\item fact(10) needs 80 bytes
\item Deep recursion can overflow stack!
\end{itemize}

\subsection{Memory Layout and Stack vs. Heap}

\subsubsection{Complete Memory Layout}

\begin{verbatim}
High Address (0xFFFFFFFF)
  +--------------+
  |   Reserved   | OS and system
  +--------------+
  |    Stack     | <- SP (grows down)
  |      v       |  Automatic storage
  |              |  Function call data
  |              |  Local variables
  |              |
  |   (unused)   |
  |              |
  |      ^       |
  |    Heap      |  Dynamic allocation
  |              |  malloc/free, new/delete
  +--------------+
  | Static Data  |  Global variables
  |              |  String constants
  +--------------+
  |    Text      |  Program instructions
  | (Code)       |  Read-only
  +--------------+
Low Address (0x00000000)
\end{verbatim}

\subsubsection{Stack Characteristics}

\textbf{Automatic Storage}

\begin{itemize}
\item Allocated when function called
\item Released when function returns
\item Managed automatically by compiler/runtime
\end{itemize}

\textbf{Fast Access}

\begin{itemize}
\item Fixed addressing pattern
\item SP always points to top
\item Simple offset calculations
\end{itemize}

\textbf{Limited Size}

\begin{itemize}
\item Typically 1-8 MB
\item Stack overflow if exceeded
\item Recursion depth limited
\end{itemize}

\textbf{Scope}

\begin{itemize}
\item Local to function
\item Not accessible after return
\item Perfect for temporary data
\end{itemize}

\subsubsection{Heap Characteristics}

\textbf{Dynamic Allocation}

\begin{itemize}
\item malloc/free in C
\item new/delete in C++
\item Programmer controls lifetime
\end{itemize}

\textbf{Flexible Size}

\begin{itemize}
\item Can grow large (limited by available memory)
\item Variable-sized allocations
\end{itemize}

\textbf{Manual Management}

\begin{itemize}
\item Must explicitly free memory
\item Memory leaks if not freed
\item Fragmentation possible
\end{itemize}

\textbf{Global Scope}

\begin{itemize}
\item Persists until explicitly freed
\item Can pass pointers across functions
\item Suitable for data structures
\end{itemize}

\subsection{Key Takeaways}

\begin{enumerate}
\item \textbf{Function calling requires} parameter passing, return value handling, and register preservation.

\item \textbf{R0-R3 for arguments and returns} - caller doesn't expect preservation.

\item \textbf{R4-R11 must be preserved} by callee if used, protecting caller's data.

\item \textbf{BL instruction} saves return address in LR and jumps to function.

\item \textbf{Return via MOV PC, LR} copies link register to program counter.

\item \textbf{Stack is LIFO structure} growing downward from high addresses, pointed to by SP.

\item \textbf{Stack usage} includes saving registers, local variables, return addresses, and extra arguments.

\item \textbf{Allocate with SUB SP, release with ADD SP} - must balance allocations and releases.

\item \textbf{Non-leaf functions} must save LR to stack before making nested calls.

\item \textbf{Recursion} creates multiple stack frames, one per call, unwinding as calls return.

\item \textbf{Stack vs. Heap} - stack is automatic/local/fast/limited, heap is manual/global/flexible/larger.

\item \textbf{Register conventions} enable modularity and prevent conflicts in shared register file.
\end{enumerate}

\subsection{Summary}

Function calling mechanisms enable modular programming by providing structured ways to pass control, data, and return values between code sections. ARM's register conventions balance efficiency (passing arguments in registers) with safety (preserving callee-saved registers). The stack provides essential temporary storage for register preservation, local variables, and handling nested calls including recursion. Understanding these mechanisms is crucial for translating high-level function-based code to assembly, optimizing performance, and debugging stack-related issues. The interplay between registers, stack, and calling conventions forms the foundation for understanding how real programs execute, preparing us for more advanced topics like exception handling, operating systems, and compiler optimization.

% \section{Lecture 8: Memory Access and String Operations}

\emph{By Dr. Kisaru Liyanage}

\subsection{Introduction}

This lecture explores character data handling, string operations, and the compilation/linking/loading process. We examine byte and half-word memory operations, implement string manipulation functions, use library functions like scanf and printf, and understand how programs transform from source code to executable binaries. These topics bridge high-level programming concepts and low-level assembly implementation, essential for systems programming and understanding program execution.

\subsection{Character Data and Encoding}

\subsubsection{ASCII Encoding}

\textbf{Basic 7-Bit Standard}

\begin{itemize}
\item Represents 128 characters using 7 bits (2⁷ = 128)
\item 95 graphic symbols (printable): A-Z, a-z, 0-9, punctuation
\item 33 control symbols: newline ('\\n'), tab ('\\t'), null ('\\0')
\item Most basic and widely used encoding

\textbf{ASCII Examples}

\begin{verbatim}
'A' = 65 (0x41)
'a' = 97 (0x61)
'0' = 48 (0x30)
'\n' = 10 (0x0A)
'\0' = 0 (0x00) - null terminator
\end{verbatim}

\subsubsection{Latin-1 Encoding}

\textbf{Extended 8-Bit Standard}

\begin{itemize}
\item Supports 256 characters using 8 bits (2⁸ = 256)
\item Includes all ASCII characters (first 128)
\item Adds 96 additional graphic characters
\item European language support (accented characters)

\subsubsection{Unicode Encoding}

\textbf{Modern Universal Standard}

\begin{itemize}
\item Uses 32-bit character set (2³² possible characters)
\item Can represent most world alphabets and symbols
\item Used in modern languages (Java, C++, Python 3)
\item Variable-length encodings: UTF-8, UTF-16
\item UTF-8: 1-4 bytes per character (backward compatible with ASCII)

\textbf{Why Unicode?}

\begin{itemize}
\item Global language support
\item Emoji and special symbols
\item Mathematical and technical symbols
\item Historical scripts and languages

\subsection{Byte Load/Store Operations}

\subsubsection{Load Register Byte (LDRB)}

\textbf{Syntax}

\begin{lstlisting}[language=assembly]
LDRB Rd, [Rn, #offset]   ; Load byte from memory
\end{verbatim}

\textbf{Operation}

\begin{itemize}
\item Reads 8 bits (1 byte) from memory
\item Fills upper 24 bits of register with zeros (zero-extension)
\item Lower 8 bits contain the loaded byte

\textbf{Example}

\begin{lstlisting}[language=assembly]
; Memory[0x1000] = 0x42 ('B')
LDR R1, =0x1000
LDRB R0, [R1]
; R0 = 0x00000042
\end{verbatim}

\textbf{Use Cases}

\begin{itemize}
\item Loading single characters
\item Reading byte arrays
\item Accessing packed data structures
\item I/O port access

\subsubsection{Store Register Byte (STRB)}

\textbf{Syntax}

\begin{lstlisting}[language=assembly]
STRB Rd, [Rn, #offset]   ; Store byte to memory
\end{verbatim}

\textbf{Operation}

\begin{itemize}
\item Writes lower 8 bits of register to memory
\item Upper 24 bits of register ignored
\item Only affects 1 byte in memory

\textbf{Example}

\begin{lstlisting}[language=assembly]
MOV R0, #0x41        ; 'A'
LDR R1, =0x2000
STRB R0, [R1]        ; Memory[0x2000] = 0x41
\end{verbatim}

\subsubsection{Load Register Signed Byte (LDRSB)}

\textbf{Syntax}

\begin{lstlisting}[language=assembly]
LDRSB Rd, [Rn, #offset]  ; Load signed byte
\end{verbatim}

\textbf{Operation}

\begin{itemize}
\item Loads 8 bits from memory
\item Replicates sign bit (bit 7) to fill upper 24 bits
\item Sign-extension preserves signed value

\textbf{Example}

\begin{lstlisting}[language=assembly]
; Memory[0x1000] = 0xFE (-2 in signed byte)
LDR R1, =0x1000
LDRSB R0, [R1]
; R0 = 0xFFFFFFFE (-2 in 32-bit signed)

; Memory[0x1001] = 0x7F (+127)
LDRSB R0, [R1, #1]
; R0 = 0x0000007F (+127)
\end{verbatim}

\textbf{When to Use}

\begin{itemize}
\item Loading signed characters (int8_t)
\item Temperature values
\item Signed offsets or deltas

\subsubsection{Memory Alignment}

\textbf{LDRB Advantages}

\begin{itemize}
\item Can access ANY byte address
\item No alignment requirement
\item Example: addresses 0, 1, 2, 3, 4, 5...

\textbf{LDR Requirement}

\begin{itemize}
\item Must use word-aligned addresses (multiples of 4)
\item Valid addresses: 0, 4, 8, 12, 16...
\item Invalid: 1, 2, 3, 5, 6, 7, 9...
\item Unaligned access causes errors or performance penalties

\subsection{Half-Word Load/Store Operations}

\subsubsection{Load Register Half-word (LDRH)}

\textbf{Syntax}

\begin{lstlisting}[language=assembly]
LDRH Rd, [Rn, #offset]   ; Load 16 bits
\end{verbatim}

\textbf{Operation}

\begin{itemize}
\item Loads 16 bits (2 bytes) from memory
\item Fills upper 16 bits with zeros (zero-extension)

\textbf{Example}

\begin{lstlisting}[language=assembly]
; Memory[0x1000-0x1001] = 0xABCD
LDR R1, =0x1000
LDRH R0, [R1]
; R0 = 0x0000ABCD
\end{verbatim}

\textbf{Use Cases}

\begin{itemize}
\item Loading 16-bit integers (short)
\item Unicode characters (UTF-16)
\item 16-bit data types

\subsubsection{Store Register Half-word (STRH)}

\textbf{Syntax}

\begin{lstlisting}[language=assembly]
STRH Rd, [Rn, #offset]   ; Store 16 bits
\end{verbatim}

\textbf{Operation}

\begin{itemize}
\item Writes lower 16 bits of register to memory
\item Upper 16 bits ignored

\textbf{Example}

\begin{lstlisting}[language=assembly]
MOV R0, #0x1234
LDR R1, =0x2000
STRH R0, [R1]
; Memory[0x2000-0x2001] = 0x1234
\end{verbatim}

\subsubsection{Load Register Signed Half-word (LDRSH)}

\textbf{Syntax}

\begin{lstlisting}[language=assembly]
LDRSH Rd, [Rn, #offset]  ; Load signed 16-bit
\end{verbatim}

\textbf{Operation}

\begin{itemize}
\item Loads 16 bits from memory
\item Replicates sign bit (bit 15) to upper 16 bits
\item Sign-extension

\textbf{Example}

\begin{lstlisting}[language=assembly]
; Memory = 0x8000 (-32768 as signed 16-bit)
LDRSH R0, [R1]
; R0 = 0xFFFF8000 (-32768 as signed 32-bit)
\end{verbatim}

\subsection{String Copy Example (strcpy)}

\subsubsection{C Implementation}

\textbf{Code}

\begin{lstlisting}[language=c]
void strcpy(char x[], char y[]) {
    int i = 0;
    while ((x[i] = y[i]) != '\\0') {
        i++;
    }
}
\end{verbatim}

\textbf{Algorithm}

\begin{enumerate}
\item Copy characters from y to x one at a time
\item Stop when null terminator ('\\0') encountered
\item Null terminator also copied

\subsubsection{ARM Assembly Implementation}

\textbf{Register Allocation}

\begin{verbatim}
R0: Base address of x (destination)
R1: Base address of y (source)
R4: Loop counter i
R2: Address of y[i]
R3: Value of y[i]
R12: Address of x[i]
\end{verbatim}

\textbf{Complete Assembly}

\begin{lstlisting}[language=assembly]
strcpy:
    ; Prologue: Save R4 (must preserve)
    SUB SP, SP, #4
    STR R4, [SP, #0]

    ; Initialize counter
    MOV R4, #0           ; i = 0

loop:
    ; Calculate address of y[i]
    ADD R2, R4, R1       ; R2 = y + i

    ; Load y[i]
    LDRB R3, [R2, #0]    ; R3 = y[i]

    ; Calculate address of x[i]
    ADD R12, R4, R0      ; R12 = x + i

    ; Store to x[i]
    STRB R3, [R12, #0]   ; x[i] = y[i]

    ; Check for null terminator
    CMP R3, #0           ; Is y[i] == '\\0'?
    BEQ done             ; If yes, exit loop

    ; Increment counter
    ADD R4, R4, #1       ; i++
    B loop               ; Continue loop

done:
    ; Epilogue: Restore R4
    LDR R4, [SP, #0]
    ADD SP, SP, #4
    MOV PC, LR           ; Return
\end{verbatim}

\subsubsection{Key Points}

\textbf{Why LDRB/STRB?}

\begin{itemize}
\item Strings are char arrays (8-bit elements)
\item Must use byte operations

\textbf{Register Preservation}

\begin{itemize}
\item R4 must be saved/restored (callee-saved)
\item R12 doesn't need preservation (scratch register)

\textbf{Offsets Are Immediate}

\begin{itemize}
\item \texttt{[R2, #0]} uses immediate offset (hash symbol)
\item Cannot use \texttt{[R2, R3]} directly without proper syntax

\subsection{Library Functions: scanf and printf}

\subsubsection{scanf Function}

\textbf{Purpose}

\begin{itemize}
\item Read input from standard input (keyboard)
\item Parse formatted input

\textbf{C Signature}

\begin{lstlisting}[language=c]
int scanf(const char *format, ...);
\end{verbatim}

\textbf{Arguments}

\begin{itemize}
\item R0: Address of format string ("%d", "%c", "%s", etc.)
\item R1: Address where to store input (NOT the value!)
\item R2, R3: Additional addresses for more inputs

\textbf{Example: Read Integer}

\textbf{C Code}

\begin{lstlisting}[language=c]
int x;
scanf("%d", &x);  // Note: &x (address of x)
\end{verbatim}

\textbf{ARM Assembly}

\begin{lstlisting}[language=assembly]
.data
formatS: .asciz "%d"

.text
    ; Allocate space for variable
    SUB SP, SP, #4       ; Space for x

    ; Load format string address
    LDR R0, =formatS     ; R0 = address of "%d"

    ; Load stack address
    MOV R1, SP           ; R1 = address where to store

    ; Call scanf
    BL scanf

    ; Value now stored at [SP]
    LDR R2, [SP, #0]     ; R2 = x
\end{verbatim}

\subsubsection{printf Function}

\textbf{Purpose}

\begin{itemize}
\item Print output to standard output (screen)
\item Format and display data

\textbf{C Signature}

\begin{lstlisting}[language=c]
int printf(const char *format, ...);
\end{verbatim}

\textbf{Arguments}

\begin{itemize}
\item R0: Address of format string
\item R1, R2, R3: VALUES to print (not addresses!)

\textbf{Example: Print Integer}

\textbf{C Code}

\begin{lstlisting}[language=c]
printf("Result: %d\\n", result);
\end{verbatim}

\textbf{ARM Assembly}

\begin{lstlisting}[language=assembly]
.data
formatP: .asciz "Result: %d\\n"

.text
    ; Load value to print
    LDR R1, [SP, #0]     ; R1 = result (value, not address)

    ; Release stack space (before printf)
    ADD SP, SP, #4

    ; Load format string
    LDR R0, =formatP

    ; Call printf
    BL printf
\end{verbatim}

\subsubsection{Data Section and Format Strings}

\textbf{Data Section}

\begin{lstlisting}[language=assembly]
.data
formatS: .asciz "%d"      ; Input format
formatP: .asciz "Result: %d\\n"  ; Output format
array: .word 1, 2, 3, 4   ; Array
message: .asciz "Hello"   ; String
\end{verbatim}

\textbf{.asciz Directive}

\begin{itemize}
\item Defines null-terminated string
\item Automatically adds '\\0' at end
\item Stored in data section (separate from code)

\textbf{Pseudo-Operation: LDR Rd, =label}

\begin{lstlisting}[language=assembly]
LDR R0, =formatS     ; Loads ADDRESS of formatS into R0
\end{verbatim}

\begin{itemize}
\item Not actual LDR instruction
\item Assembler converts to appropriate instruction(s)
\item Loads memory address (pointer), not content

\subsubsection{scanf vs printf Argument Differences}

\textbf{scanf: Needs Addresses}

\begin{lstlisting}[language=assembly]
SUB SP, SP, #4
MOV R1, SP           ; R1 = address (where to store)
BL scanf
\end{verbatim}

\textbf{printf: Needs Values}

\begin{lstlisting}[language=assembly]
LDR R1, [SP]         ; R1 = value (what to print)
BL printf
\end{verbatim}

\textbf{Why This Difference?}

\begin{itemize}
\item scanf modifies variables (needs addresses to write to)
\item printf only reads values (copies values)

\subsubsection{Calling Convention Rules}

\textbf{Follow Exact Order}

\begin{itemize}
\item R0 first, R1 second, R2 third, R3 fourth
\item Library functions expect specific argument positions
\item Assembly won't check violations
\item Mistakes cause wrong behavior or crashes

\textbf{Know Function Signatures}

\begin{itemize}
\item Read documentation
\item Understand parameter types and order
\item Match assembly to C function prototype

\subsection{Compilation, Linking, and Loading}

\subsubsection{Translation Overview}

\textbf{Complete Process}

C Program (.c)
    $\downarrow$ [Compiler]
\begin{lstlisting}[language=assembly]
    $\downarrow$ [Assembler]
\end{verbatim}
Object Module (.o)
    $\downarrow$ [Linker]
Executable (a.out)
    $\downarrow$ [Loader]
Memory (running program)

\subsubsection{Compiler}

\textbf{Function}

\begin{itemize}
\item Converts high-level C code to assembly language
\item Complex task requiring sophisticated algorithms
\item Performs optimizations

\textbf{Optimizations}

\begin{itemize}
\item Register allocation
\item Instruction selection
\item Loop unrolling
\item Dead code elimination
\item Function inlining

\textbf{Example}

\begin{lstlisting}[language=c]
int add(int a, int b) {
    return a + b;
}
\end{verbatim}

$\downarrow$ Compiler

\begin{lstlisting}[language=assembly]
add:
    ADD R0, R0, R1
    MOV PC, LR
\end{verbatim}

\subsubsection{Assembler}

\textbf{Function}

\begin{itemize}
\item Converts assembly language to machine code (binary)
\item Simpler than compilation (mostly 1-to-1 mapping)
\item Produces object modules

\textbf{Tasks}

\begin{enumerate}
\item Translate instructions to binary opcodes
\item Resolve local labels to addresses
\item Generate symbol table
\item Create relocation information

\textbf{Object Module Structure}

\textbf{Header}

\begin{itemize}
\item Describes contents and sizes

\textbf{Text Segment}

\begin{itemize}
\item Machine instructions (binary code)

\textbf{Static Data Segment}

\begin{itemize}
\item Initialized global variables
\item String constants (format strings)

\textbf{Relocation Info}

\begin{itemize}
\item Instructions/data depending on absolute addresses
\item Needed when program loaded at different address

\textbf{Symbol Table}

\begin{itemize}
\item Global definitions: functions, variables defined here
\item External references: functions/variables from other modules
\item Enables linking

\textbf{Debug Info}

\begin{itemize}
\item Maps machine code to source code lines
\item Used by debuggers (gdb)

\subsubsection{Linker}

\textbf{Function}

\begin{itemize}
\item Combines multiple object modules into executable
\item Links program code with library code

\textbf{Tasks}

\textbf{1. Merge Segments}

\begin{verbatim}
program.o:      lib.o:          Result:
[Text1]         [Text2]     $\rightarrow$   [Text1+Text2]
[Data1]         [Data2]     $\rightarrow$   [Data1+Data2]
\end{verbatim}

\textbf{2. Resolve Labels}

\begin{itemize}
\item Convert symbolic names to actual addresses
\item Example: "printf" $\rightarrow$ 0x80481234
\item Processor only understands addresses

\textbf{3. Patch References}

\begin{itemize}
\item Update function calls to correct addresses
\item Fix relocatable addresses
\item May leave some for loader

\subsubsection{Static vs Dynamic Linking}

\textbf{Static Linking}

\begin{itemize}
\item Library code copied into executable at compile time
\item Larger executable files
\item Self-contained (no external dependencies)
\item All code in one file

\textbf{Advantages}

\begin{itemize}
\item No runtime dependencies
\item Faster load time
\item Predictable behavior

\textbf{Disadvantages}

\begin{itemize}
\item Large file sizes
\item No benefit from library updates
\item Memory duplication across programs

\textbf{Dynamic Linking}

\begin{itemize}
\item Library code loaded at runtime when called
\item Smaller executables
\item Shared libraries on system

\textbf{Advantages}

\begin{itemize}
\item Smaller executables
\item Shared libraries (less memory usage)
\item Automatic library updates
\item Less disk space

\textbf{Disadvantages}

\begin{itemize}
\item Requires libraries installed on system
\item "DLL not found" errors
\item Slightly slower initial load

\textbf{DLL (Dynamic Link Library) - Windows}

\begin{itemize}
\item File extension: .dll
\item Shared by multiple programs
\item Must be present on system
\item Example: msvcrt.dll (C runtime library)

\subsubsection{Loader}

\textbf{Function}

\begin{itemize}
\item Loads executable from disk into memory
\item Prepares program for execution
\item Initializes execution environment

\textbf{Loading Steps}

\textbf{1. Read Header}

\begin{itemize}
\item Determine segment sizes
\item Text segment size
\item Data segment size
\item Other metadata

\textbf{2. Create Virtual Address Space}

\begin{itemize}
\item Allocate memory for program
\item Set up page tables (virtual memory)
\item Map segments to physical memory

\textbf{3. Copy Segments to Memory}

\begin{itemize}
\item Text segment (instructions)
\item Initialized data
\item Set up page table entries
\item Mark text as read-only, data as read-write

\textbf{4. Set Up Arguments on Stack}

\begin{itemize}
\item Command-line arguments: argc, argv
\item Environment variables
\item Initial stack frame

\textbf{Example}

\begin{lstlisting}[language=bash]
./program arg1 arg2
\end{verbatim}

\begin{itemize}
\item argc = 3
\item argv[0] = "./program"
\item argv[1] = "arg1"
\item argv[2] = "arg2"

\textbf{5. Initialize Registers}

\begin{itemize}
\item Set up register file
\item PC points to entry point (\_start)
\item SP points to top of stack
\item Other registers to initial values

\textbf{6. Jump to Startup Routine}

\begin{itemize}
\item Calls C runtime initialization
\item Sets up standard library
\item Calls main() function
\item When main returns, calls exit()

\subsection{Exercises}

\subsubsection{Common String Operations}

\textbf{String Length}

\begin{lstlisting}[language=c]
int strlen(char *s) {
    int len = 0;
    while (s[len] != '\\0')
        len++;
    return len;
}
\end{verbatim}

\textbf{String Reverse}

\begin{lstlisting}[language=c]
void strrev(char *s) {
    int len = strlen(s);
    for (int i = 0; i < len/2; i++) {
        char temp = s[i];
        s[i] = s[len-1-i];
        s[len-1-i] = temp;
    }
}
\end{verbatim}

\subsubsection{Integer I/O}

\textbf{Read Two Integers, Print Sum}

\begin{lstlisting}[language=assembly]
; Read x and y
; Print x + y
\end{verbatim}

\textbf{Read n, Print 1 to n}

\begin{lstlisting}[language=assembly]
; Read n
; Loop from 1 to n, print each
\end{verbatim}

\subsubsection{Skills Required}

\begin{itemize}
\item Character data handling (LDRB/STRB)
\item String manipulation
\item scanf for input
\item printf for output
\item Stack management
\item Function calling conventions
\item Loop implementation
\item Array indexing

\subsection{Key Takeaways}

\begin{enumerate}
\item \textbf{ASCII (7-bit), Latin-1 (8-bit), Unicode (32-bit)} represent character data with increasing capacity.

\begin{enumerate}
\item \textbf{LDRB/STRB for byte operations}, LDRH/STRH for half-words - smaller than word operations.

\begin{enumerate}
\item \textbf{Byte operations don't require alignment} unlike word operations (LDR/STR).

\begin{enumerate}
\item \textbf{Sign extension (LDRSB/LDRSH)} replicates sign bit to preserve signed values.

\begin{enumerate}
\item \textbf{Strings in C are char arrays} terminated with null character ('\\0' = 0).

\begin{enumerate}
\item \textbf{scanf and printf are library functions} called via BL instruction.

\begin{enumerate}
\item \textbf{scanf needs addresses (where to store)}, printf needs values (what to print).

\begin{enumerate}
\item \textbf{Format strings stored in .data section} using .asciz directive.

\begin{enumerate}
\item \textbf{Arguments passed in R0-R3} following ARM calling convention.

10. \textbf{Compilation chain: Compile $\rightarrow$ Assemble $\rightarrow$ Link $\rightarrow$ Load $\rightarrow$ Execute}.

11. \textbf{Static linking includes libraries in executable}, dynamic linking loads at runtime.

12. \textbf{Loader sets up virtual memory, copies segments, initializes stack} with arguments.

\subsection{Summary}

Character data handling and library function usage bridge high-level programming concepts and assembly implementation. Understanding byte/half-word operations enables efficient string manipulation and compact data storage. The scanf/printf functions demonstrate how assembly code interfaces with system libraries, requiring careful attention to calling conventions and argument types. The compilation, linking, and loading process reveals how source code transforms into running programs, involving multiple stages with distinct responsibilities. Static and dynamic linking represent different trade-offs between self-containment and flexibility. These concepts are essential for systems programming, understanding program structure, and debugging low-level issues. This knowledge prepares us for advanced topics including operating systems, compilers, and system-level optimization.


% Chapter 3: Processor Design
% \chapter{Processor Design}

% \section{Lecture 9: Microarchitecture and Datapath Design}

\emph{By Dr. Isuru Nawinne}

\subsection{Introduction}

This lecture transitions from instruction set architecture (ISA) to microarchitecture---the hardware implementation of the ISA. We explore how to build a processor that executes MIPS instructions, covering instruction formats, digital logic fundamentals, datapath construction, and single-cycle processor design. Understanding microarchitecture reveals how software instructions translate to hardware operations and provides the foundation for studying advanced processor designs including pipelining and superscalar execution.

\subsection{Course Context and MIPS ISA}

\subsubsection{Transition to Hardware Implementation}

\textbf{Previous Focus}: ARM ISA

\begin{itemize}
\item Instruction set
\item Assembly programming
\item Software perspective
\end{itemize}

\textbf{Current Focus}: MIPS Microarchitecture

\begin{itemize}
\item Hardware implementation
\item Processor design
\item Hardware perspective
\end{itemize}

\textbf{Why MIPS for Hardware Study?}

\begin{itemize}
\item Simpler than ARM (educational clarity)
\item Clean RISC design
\item Well-documented architecture
\item Concepts apply to all processors
\end{itemize}

\subsubsection{MIPS Instruction Categories}

\textbf{Three Instruction Types} (based on encoding)

\textbf{I-Type (Immediate)}

\begin{itemize}
\item Contains one immediate operand
\item Covers data processing, data transfer, control flow
\item Examples: ADDI, LW, SW, BEQ
\item Most common type
\end{itemize}

\textbf{R-Type (Register)}

\begin{itemize}
\item All operands are registers
\item Primarily arithmetic and logic
\item Examples: ADD, SUB, AND, OR
\item Opcode always 0, funct field specifies operation
\end{itemize}

\textbf{J-Type (Jump)}

\begin{itemize}
\item Jump instructions
\item Examples: J, JAL
\item 26-bit address field
\end{itemize}

\textbf{Contrast with ARM}

\begin{itemize}
\item ARM: Data processing, data transfer, flow control
\item MIPS: I-type, R-type, J-type
\item Different classification philosophy
\end{itemize}

\subsubsection{MIPS Instruction Encoding}

\textbf{Fixed 32-Bit Length}

\begin{itemize}
\item Every instruction exactly 32 bits
\item Simplifies fetch and decode
\item Enables efficient pipelining
\end{itemize}

\textbf{R-Type Format}

\begin{verbatim}
[Opcode][RS][RT][RD][SHAMT][Funct]
 6 bits  5   5   5    5      6 bits
\end{verbatim}

Fields:

\begin{itemize}
\item \textbf{Opcode}: Always 0 for R-type
\item \textbf{RS}: Source register 1 (5 bits for 32 registers)
\item \textbf{RT}: Source register 2
\item \textbf{RD}: Destination register
\item \textbf{SHAMT}: Shift amount (for shift instructions)
\item \textbf{Funct}: Function code (actual operation)
\end{itemize}

\textbf{I-Type Format}

\begin{verbatim}
[Opcode][RS][RT][Immediate]
 6 bits  5   5   16 bits
\end{verbatim}

Fields:

\begin{itemize}
\item \textbf{Opcode}: Varies by instruction
\item \textbf{RS}: Source/base register
\item \textbf{RT}: Source/destination register
\item \textbf{Immediate}: 16-bit immediate value or offset
\end{itemize}

\textbf{J-Type Format}

\begin{verbatim}
[Opcode][Address]
 6 bits  26 bits
\end{verbatim}

Fields:

\begin{itemize}
\item \textbf{Opcode}: 2 for J, 3 for JAL
\item \textbf{Address}: 26-bit jump target (word address)
\end{itemize}

\subsection{Digital Logic Review}

\subsubsection{Information Encoding}

\textbf{Binary Representation}

\begin{itemize}
\item Low voltage = Logic 0
\item High voltage = Logic 1
\item Digital signals immune to analog noise
\end{itemize}

\textbf{Multi-Bit Signals}

\begin{itemize}
\item One wire per bit
\item 32-bit instruction needs 32 wires
\item Parallel transmission within CPU
\end{itemize}

\subsubsection{Combinational Elements}

\textbf{Definition}

\begin{itemize}
\item Output is function of inputs ONLY
\item No internal state or memory
\item Purely functional relationship
\end{itemize}

\textbf{Examples}

\begin{itemize}
\item AND, OR, NOT gates
\item Multiplexers: $Y = (S == 0) ? I0 : I1$
\item Adders: $Y = A + B$
\item ALU: $Y = \text{function}(A, B, operation)$
\end{itemize}

\textbf{Characteristics}

\begin{itemize}
\item Output changes immediately with input (plus propagation delay)
\item Can draw complete truth table
\item Asynchronous operation (no clock needed)
\end{itemize}

\subsubsection{Sequential Elements (State Elements)}

\textbf{Definition}

\begin{itemize}
\item Output is function of inputs AND internal state
\item Has memory---stores information over time
\item State persists between clock cycles
\end{itemize}

\textbf{Examples}

\begin{itemize}
\item Registers
\item Flip-flops
\item Register files
\item Memory units
\end{itemize}

\textbf{Characteristics}

\begin{itemize}
\item Store information
\item Synchronized to clock signal
\item Output depends on history
\end{itemize}

\subsubsection{Clocking and Timing}

\textbf{Clock Signal}

\begin{itemize}
\item Periodic alternating signal: Low $\rightarrow$ High $\rightarrow$ Low $\rightarrow$ High...
\item Synchronizes all sequential operations
\end{itemize}

\textbf{Edge-Triggered}

\begin{itemize}
\item Rising edge: Transition $0 \rightarrow 1$
\item Falling edge: Transition $1 \rightarrow 0$
\item Most processors use rising edge
\end{itemize}

\textbf{Clock Period and Frequency}

\begin{verbatim}
Clock Period (T): Duration of one cycle
Clock Rate (f): Cycles per second

Relationship: f = 1/T

Example:
T = 250 ps = 0.25 ns
f = 1/(250 x 10^-12) = 4 GHz
\end{verbatim}

\subsubsection{Register Operations}

\textbf{Basic Register}

\begin{itemize}
\item Stores multi-bit value (e.g., 32 bits)
\item Updates on clock edge: D (input) $\rightarrow$ Q (output state)
\end{itemize}

\textbf{Timing Example}

\begin{figure}[h]
\centering
\includegraphics[width=0.7\textwidth]{img/Chapter 9 Register.jpeg}
\caption{Register Timing Diagram}
\end{figure}

\textbf{Register with Write Control}

\begin{itemize}
\item Additional Write Enable signal
\item Updates ONLY when clock edge AND Write Enable = 1
\item Otherwise holds previous value
\end{itemize}

\textbf{Timing Example}

\begin{figure}[h]
\centering
\includegraphics[width=0.7\textwidth]{img/Chapter 9 Write EN Register.jpeg}
\caption{Register with Write Enable Timing Diagram}
\end{figure}

\subsubsection{Critical Path and Clock Period}

\textbf{Combinational Logic Delay}

\begin{itemize}
\item All combinational elements have propagation delay
\item Different elements, different delays
\end{itemize}

\textbf{Clock Period Constraint}

\begin{verbatim}
Clock Period >= Longest Path Delay

Path: Register -> Combinational Logic -> Register

Must allow time for:
1. Register output stabilization
2. Combinational logic computation
3. Result reaching next register input
4. Setup time before next clock edge
\end{verbatim}

\textbf{Critical Path}

\begin{itemize}
\item Longest delay path from register to register
\item Determines minimum clock period
\item Limits maximum clock frequency
\end{itemize}

\textbf{Single-Cycle Constraint}

\begin{itemize}
\item Complete one instruction per clock cycle
\item Clock period must accommodate slowest instruction
\item All instructions take same time (inefficient!)
\end{itemize}

\subsection{CPU Execution Stages}

\begin{figure}[h]
\centering
\includegraphics[width=0.7\textwidth]{img/Chapter 9 CPU Overview.jpeg}
\caption{CPU Execution Stages Overview}
\end{figure}

\subsubsection{Instruction Fetch (IF)}

\textbf{Purpose}: Retrieve next instruction from memory

\textbf{Steps}:

\begin{enumerate}
\item Use Program Counter (PC) for instruction address
\item Access Instruction Memory with PC
\item Retrieve 32-bit instruction word
\item Instruction now in CPU for processing
\end{enumerate}

\textbf{Hardware}:

\begin{itemize}
\item Program Counter (32-bit register)
\item Instruction Memory (read-only during execution)
\item Address bus from PC to memory
\item Data bus from memory to CPU
\end{itemize}

\subsubsection{Instruction Decode (ID)}

\textbf{Purpose}: Interpret instruction and extract fields

\textbf{Decode Operations}:

\begin{enumerate}
\item \textbf{Examine Opcode} (bits 26-31):
  \begin{itemize}
  \item If opcode = 0: R-type
  \item If opcode = 2 or 3: J-type
  \item Otherwise: I-type
  \end{itemize}

\item \textbf{Extract Register Numbers}:
  \begin{itemize}
  \item R-type: RS, RT, RD (three 5-bit fields)
  \item I-type: RS, RT (two 5-bit fields)
  \item J-type: No registers
  \end{itemize}

\item \textbf{Extract Immediate/Address}:
  \begin{itemize}
  \item I-type: 16-bit immediate
  \item J-type: 26-bit address
  \end{itemize}

\item \textbf{Extract Function/Shift} (R-type only):
  \begin{itemize}
  \item Funct: bits 0-5 (ALU operation)
  \item SHAMT: bits 6-10 (shift amount)
  \end{itemize}
\end{enumerate}

\textbf{Control Unit Role}:

\begin{itemize}
\item Decodes opcode
\item Generates control signals
\item Determines datapath activation
\end{itemize}

\subsubsection{Execute (EX)}

\textbf{Purpose}: Perform operation or calculate address

\textbf{Operations by Type}:

\textbf{Arithmetic/Logic (R-type, I-type arithmetic)}:

\begin{itemize}
\item Send operands to ALU
\item ALU performs operation
\item Operation from funct field (R-type) or opcode (I-type)
\end{itemize}

\textbf{Memory Access (Load/Store)}:

\begin{itemize}
\item ALU calculates address: Base + Offset
\item Always performs addition
\item Result is memory address
\end{itemize}

\textbf{Branch}:

\begin{itemize}
\item ALU compares registers: RS - RT
\item Zero flag indicates equality
\item Result determines branch decision
\end{itemize}

\subsubsection{Memory Access (MEM)}

\textbf{Purpose}: Read or write data memory

\textbf{Applies To}:

\begin{itemize}
\item Load instructions: Read from memory
\item Store instructions: Write to memory
\item NOT arithmetic/logic (skip this stage)
\end{itemize}

\textbf{Load Operation}:

\begin{enumerate}
\item Use address from ALU
\item Read data from memory
\item Data will be written to register
\end{enumerate}

\textbf{Store Operation}:

\begin{enumerate}
\item Use address from ALU
\item Get data from RT register
\item Write data to memory
\end{enumerate}

\subsubsection{Register Write-Back (WB)}

\textbf{Purpose}: Write result to destination register

\textbf{Applies To}:

\begin{itemize}
\item Arithmetic/Logic: Write ALU result
\item Load: Write memory data
\item NOT store or branch
\end{itemize}

\textbf{Source Selection}:

\begin{itemize}
\item Arithmetic/Logic: Data from ALU
\item Load: Data from memory
\item Multiplexer selects appropriate source
\end{itemize}

\subsubsection{PC Update}

\textbf{Purpose}: Determine next instruction address

\textbf{Default}: PC = PC + 4 (sequential)

\textbf{Branch/Jump}: PC = calculated target address

\textbf{Control Flow}:

\begin{itemize}
\item Multiplexer selects next PC value
\item Sequential or branch/jump target
\item Update happens at clock edge
\end{itemize}

\subsection{R-Type Instruction Datapath}

\subsubsection{Register File}

\textbf{Structure}:

\begin{itemize}
\item 32 registers (R0-R31), 32 bits each
\item Three ports: 2 read, 1 write
\end{itemize}

\textbf{Read Ports}:

\begin{itemize}
\item Read Address 1: RS (5 bits)
\item Read Address 2: RT (5 bits)
\item Read Data 1: 32-bit output
\item Read Data 2: 32-bit output
\item Combinational (no clock)
\end{itemize}

\textbf{Write Port}:

\begin{itemize}
\item Write Address: RD (5 bits)
\item Write Data: 32-bit input
\item Write Enable: Control signal
\item Synchronized (clock edge)
\end{itemize}

\subsubsection{R-Type Execution Flow}

\textbf{Instruction}: \texttt{ADD \$t0, \$t1, \$t2} (R0 = R1 + R2)

\textbf{Step 1: Register Read}

\begin{itemize}
\item Extract RS (R1) and RT (R2) fields
\item Register file outputs two 32-bit values
\end{itemize}

\textbf{Step 2: ALU Operation}

\begin{itemize}
\item Inputs: Two register values
\item Funct field (6 bits) $\rightarrow$ ALU control (4 bits)
\item ALU performs specified operation
\item Examples: ADD, SUB, AND, OR, SLT
\end{itemize}

\textbf{Step 3: Write-Back}

\begin{itemize}
\item ALU result $\rightarrow$ Register file write data
\item RD field specifies destination
\item Write Enable = 1
\item At clock edge: Result written
\end{itemize}

\subsubsection{ALU Control}

\textbf{Function Field Encoding}:

\begin{verbatim}
Funct     | Operation | ALU Control
----------|-----------|-------------
0x20      | ADD       | 0010
0x22      | SUB       | 0110
0x24      | AND       | 0000
0x25      | OR        | 0001
0x2A      | SLT       | 0111
\end{verbatim}

\textbf{ALU Control Logic}:

\begin{itemize}
\item Input: 6-bit funct field
\item Output: 4-bit ALU operation
\item Combinational logic (lookup table)
\end{itemize}

\subsection{I-Type Instruction Datapath}

\subsubsection{Differences from R-Type}

\textbf{Operand Sources}:

\begin{itemize}
\item R-type: Both from registers
\item I-type: One register, one immediate
\end{itemize}

\textbf{Register Usage}:

\begin{itemize}
\item RS: Source register
\item RT: Destination register (NOT source!)
\item Immediate: 16-bit operand
\end{itemize}

\subsubsection{Sign Extension}

\textbf{Problem}: 16-bit immediate, 32-bit ALU

\textbf{Process}:

\begin{enumerate}
\item Take 16-bit immediate
\item Examine bit 15 (sign bit)
\item Replicate sign bit to bits 16-31
\item Result: 32-bit signed value
\end{enumerate}

\textbf{Examples}:

\begin{verbatim}
16-bit: 0x0005 -> 32-bit: 0x00000005 (+5)
16-bit: 0xFFFB -> 32-bit: 0xFFFFFFFB (-5)
\end{verbatim}

\textbf{Hardware}: Simple wire replication (fast)

\subsubsection{Multiplexer for ALU Input}

\textbf{ALU Input B Selection}:

\begin{itemize}
\item Input 0: Register data (RT) for R-type
\item Input 1: Sign-extended immediate for I-type
\item Select: ALUSrc control signal
\end{itemize}

\textbf{ALUSrc Signal}:

\begin{verbatim}
ALUSrc = 0: Use register (R-type, branch)
ALUSrc = 1: Use immediate (I-type)
\end{verbatim}

\subsection{Load/Store Instruction Datapath}

\subsubsection{Address Calculation}

\textbf{Formula}: Address = Base + Offset

\textbf{Components}:

\begin{itemize}
\item Base: RS register (32-bit pointer)
\item Offset: 16-bit signed immediate (sign-extended)
\item ALU: Always performs addition
\end{itemize}

\textbf{Examples}:

\begin{verbatim}
LW $t1, 8($t0)    # Load from $t0 + 8
SW $t2, -4($sp)   # Store to $sp - 4
\end{verbatim}

\subsubsection{Load Word (LW)}

\textbf{Instruction Format}:

\begin{itemize}
\item RS: Base register
\item RT: Destination register
\item Immediate: Offset
\end{itemize}

\textbf{Execution}:

\begin{enumerate}
\item Read RS (base address)
\item Sign-extend immediate (offset)
\item ALU adds: Address = RS + offset
\item Read data from memory at address
\item Write data to RT register
\end{enumerate}

\textbf{Critical Path}: Longest in single-cycle design

\begin{itemize}
\item Fetch $\rightarrow$ Register Read $\rightarrow$ ALU $\rightarrow$ Memory $\rightarrow$ Register Write
\end{itemize}

\subsubsection{Store Word (SW)}

\textbf{Instruction Format}:

\begin{itemize}
\item RS: Base register
\item RT: Source register (data to store)
\item Immediate: Offset
\end{itemize}

\textbf{Execution}:

\begin{enumerate}
\item Read RS (base) and RT (data)
\item ALU calculates address
\item Write RT data to memory at address
\item NO register write-back
\end{enumerate}

\textbf{Key Difference}:

\begin{itemize}
\item Reads TWO registers (RS and RT)
\item Memory write instead of read
\item No register write stage
\end{itemize}

\subsubsection{Data Memory}

\textbf{Interface}:

\begin{itemize}
\item Address: From ALU (32 bits)
\item Write Data: From RT register
\item Read Data: To register file (for loads)
\end{itemize}

\textbf{Control Signals}:

\begin{itemize}
\item MemRead: Enable read (LW)
\item MemWrite: Enable write (SW)
\end{itemize}

\textbf{Multiplexer for Write-Back}:

\begin{itemize}
\item Input 0: ALU result (arithmetic/logic)
\item Input 1: Memory data (load)
\item Select: MemtoReg signal
\end{itemize}

\subsection{Branch Instruction Datapath}

\subsubsection{Branch Types}

\textbf{BEQ (Branch if Equal)}:

\begin{itemize}
\item Compare RS and RT
\item Branch if RS == RT
\end{itemize}

\textbf{BNE (Branch if Not Equal)}:

\begin{itemize}
\item Compare RS and RT
\item Branch if RS != RT
\end{itemize}

\subsubsection{Branch Target Calculation}

\textbf{Components}:

\begin{enumerate}
\item PC + 4 (next sequential instruction)
\item Offset from immediate (in instructions)
\item Target = (PC + 4) + (Offset $\times$ 4)
\end{enumerate}

\textbf{Why PC + 4?}

\begin{itemize}
\item Offset relative to NEXT instruction
\item PC already incremented
\end{itemize}

\textbf{Word to Byte Conversion}:

\begin{itemize}
\item Immediate: Number of instructions
\item Multiply by 4: Byte offset
\item Shift left 2 (wire routing, no hardware!)
\end{itemize}

\subsubsection{Branch Execution}

\textbf{Step 1: Register Comparison}

\begin{itemize}
\item Read RS and RT
\item ALU subtracts: RS - RT
\item Generate Zero flag
\end{itemize}

\textbf{Step 2: Zero Flag Evaluation}

\begin{itemize}
\item Zero = 1: Values equal
\item Zero = 0: Values different
\end{itemize}

\textbf{Step 3: Target Calculation} (parallel)

\begin{itemize}
\item Sign-extend immediate
\item Shift left 2
\item Add to PC + 4
\end{itemize}

\textbf{Step 4: PC Update Decision}

\begin{verbatim}
BEQ: PCSrc = Branch AND Zero
BNE: PCSrc = Branch AND NOT(Zero)
\end{verbatim}

\textbf{Multiplexer}:

\begin{itemize}
\item Input 0: PC + 4 (sequential)
\item Input 1: Branch target
\item Select: PCSrc
\end{itemize}

\subsubsection{Sign Extension and Shifting}

\textbf{Sign Extension}: Preserves signed offset

\begin{itemize}
\item Forward branch: Positive offset
\item Backward branch: Negative offset
\end{itemize}

\textbf{Shift Left 2}: Wire routing trick

\begin{itemize}
\item Take bits 0-29 of sign-extended value
\item Connect to bits 2-31 of result
\item Append two zero wires at bits 0-1
\item NO actual shifter hardware!
\end{itemize}

\subsection{Complete Single-Cycle Datapath}

\begin{figure}[h]
\centering
\includegraphics[width=0.9\textwidth]{img/Chapter 9 CPU Control and Datapath.jpeg}
\caption{Complete Single-Cycle CPU Control and Datapath}
\end{figure}

\subsubsection{Integrated Components}

\textbf{Instruction Fetch}:

\begin{itemize}
\item PC register
\item Instruction memory
\item PC + 4 adder
\end{itemize}

\textbf{Register File}:

\begin{itemize}
\item 32 registers with 3 ports
\item Two read, one write
\end{itemize}

\textbf{ALU}:

\begin{itemize}
\item Two 32-bit inputs
\item Operation control
\item Result output
\item Zero flag
\end{itemize}

\textbf{Data Memory}:

\begin{itemize}
\item Address from ALU
\item Write data from register
\item Read data to register
\end{itemize}

\textbf{Sign Extender}:

\begin{itemize}
\item 16-bit input
\item 32-bit output
\end{itemize}

\textbf{Branch Logic}:

\begin{itemize}
\item Target adder
\item PC multiplexer
\end{itemize}

\textbf{Multiplexers}:

\begin{itemize}
\item ALU input B (register vs immediate)
\item Register write data (ALU vs memory)
\item Next PC (PC+4 vs branch target)
\end{itemize}

\subsubsection{Control Signals}

\textbf{Generated by Control Unit}:

\begin{enumerate}
\item RegDst: Register destination select
\item Branch: Branch instruction indicator
\item MemRead: Memory read enable
\item MemtoReg: Memory to register select
\item MemWrite: Memory write enable
\item ALUSrc: ALU source select
\item RegWrite: Register write enable
\item ALUOp: ALU operation type
\end{enumerate}

\subsubsection{Parallel Operations}

\textbf{Key Insight}: Hardware operates in PARALLEL

\begin{itemize}
\item All datapath elements active simultaneously
\item Some produce meaningless results
\item Control signals select valid paths
\end{itemize}

\textbf{Example}: R-type instruction

\begin{itemize}
\item Sign extender operates on bits 0-15
\item Produces meaningless output (no immediate in R-type)
\item Multiplexer doesn't select it (ALUSrc = 0)
\end{itemize}

\subsubsection{Critical Path Analysis}

\textbf{Path for Load Word} (longest):

\begin{verbatim}
1. Instruction fetch:     200 ps
2. Register read:         150 ps
3. Sign extend:           50 ps
4. Multiplexer:           25 ps
5. ALU address calc:      200 ps
6. Data memory access:    200 ps
7. Multiplexer:           25 ps
8. Register write setup:  100 ps
Total:                    950 ps
\end{verbatim}

\textbf{Clock Period}: Must be $\geq$ 950 ps

\textbf{Max Frequency}: 1/950 ps $\approx$ 1.05 GHz

\textbf{Inefficiency}:

\begin{itemize}
\item ALL instructions take 950 ps
\item Fast R-type (650 ps) waits
\item Wasted time per fast instruction
\end{itemize}

\subsubsection{Single-Cycle Disadvantages}

\textbf{Inefficiency}:

\begin{itemize}
\item Fast instructions wait for slow ones
\item Clock period by worst case
\item Cannot optimize common case
\end{itemize}

\textbf{Hardware Duplication}:

\begin{itemize}
\item Separate instruction/data memories
\item Multiple adders
\item Cannot reuse hardware in same cycle
\end{itemize}

\textbf{No Parallelism}:

\begin{itemize}
\item One instruction at a time
\item Hardware mostly idle
\item Poor resource utilization
\end{itemize}

\textbf{Advantages}:

\begin{itemize}
\item Simple design
\item Simple control
\item One instruction per cycle (conceptually)
\item Good for learning
\end{itemize}

\subsection{Key Takeaways}

\begin{enumerate}
\item \textbf{Microarchitecture is hardware implementation of ISA} - translating instruction semantics to hardware operations.

\item \textbf{MIPS uses three instruction types}: R-type (registers), I-type (immediate), J-type (jump).

\item \textbf{Fixed 32-bit instructions} simplify fetch/decode and enable efficient pipelining.

\item \textbf{Combinational elements} have output as function of inputs only; sequential elements have state.

\item \textbf{Clock period must exceed longest combinational path} between sequential elements.

\item \textbf{Six execution stages}: Fetch, Decode, Execute, Memory, Write-back, PC Update.

\item \textbf{Register file has three ports}: two read (combinational), one write (clocked).

\item \textbf{Sign extension} converts 16-bit immediate to 32-bit preserving signed value.

\item \textbf{Multiplexers select between data sources} based on control signals.

\item \textbf{ALU operations vary by instruction}: addition (load/store), subtraction (branch), varies (R-type).

\item \textbf{Critical path determines clock period} - load word is longest in single-cycle design.

\item \textbf{Single-cycle processor completes one instruction per cycle} but inefficiently (all take same time).

\item \textbf{Separate instruction and data memories} required for single-cycle (both accessed same cycle).

\item \textbf{Control signals orchestrate datapath} - generated by control unit from opcode.

\item \textbf{All hardware operates in parallel} - control signals select valid results, ignore others.
\end{enumerate}

\subsection{Summary}

Microarchitecture bridges the gap between software instructions and hardware implementation, revealing how processors execute programs. Building a single-cycle MIPS processor requires understanding digital logic fundamentals, datapath component design, and control signal generation. While conceptually simple (one instruction per cycle), the single-cycle design is inefficient because all instructions must complete within the time required by the slowest instruction. The critical path---typically the load word instruction---determines the maximum clock frequency. Understanding this foundation prepares us for more sophisticated designs including multi-cycle processors (which break execution into multiple stages) and pipelined processors (which overlap instruction execution for higher throughput). These microarchitecture concepts apply broadly across processor design, from embedded systems to high-performance superscalar processors.

% \section{Lecture 10: Processor Control}

\emph{By Dr. Isuru Nawinne}

\subsection{Introduction}

This lecture completes the single-cycle MIPS processor design by exploring the control unit—the component that generates control signals based on instruction opcodes. We examine ALU control generation using a two-stage approach, design the main control unit, analyze control signal purposes, and create truth tables mapping instructions to control patterns. Understanding control unit design reveals how hardware interprets instructions and orchestrates datapath operations, completing our understanding of processor implementation.

\subsection{Control Unit Overview}

\subsubsection{Recap of Datapath Components}

\textbf{Previously Covered}:

\begin{itemize}
\item Register File (32 registers, 3 ports)
\item ALU (arithmetic/logic operations)
\item Instruction Memory (stores program)
\item Data Memory (stores data)
\item Adders (PC+4, branch target)
\item Multiplexers (data source selection)
\item Sign Extender (16-bit to 32-bit)
\item Shifter (branch offset left 2)

\subsubsection{Control Unit Purpose}

\textbf{Function}: Generate control signals based on instruction

\textbf{Inputs}:

\begin{itemize}
\item Opcode (bits 26-31, 6 bits)
\item Funct field (bits 0-5, 6 bits) for R-type

\textbf{Outputs}: Control signals for datapath

\begin{itemize}
\item Multiplexer selections
\item Register write enable
\item Memory read/write
\item ALU operation
\item Branch decision

\subsubsection{Instruction Subset for Study}

\textbf{Selected Instructions}:

\begin{itemize}
\item \textbf{Load Word (LW)}: Memory read
\item \textbf{Store Word (SW)}: Memory write
\item \textbf{Branch if Equal (BEQ)}: Conditional branch
\item \textbf{R-type}: Arithmetic, logic, shift

\textbf{Coverage}:

\begin{itemize}
\item Uses almost all datapath hardware
\item Representative of most control signals
\item Excludes: Jump instructions, I-type arithmetic

\subsection{ALU Operations for Different Instructions}

\subsubsection{Load/Store Instructions}

\textbf{Address Calculation}:

\begin{verbatim}
Address = Base Register + Immediate Offset
        = RS + Sign_Extend(Immediate)
\end{verbatim}

\textbf{ALU Function}: ADDITION (always)

\begin{itemize}
\item Input A: RS register value
\item Input B: Sign-extended immediate
\item Operation: ADD
\item ALU Control: 0010 (binary)
\item Result: Memory address

\textbf{Example}:

\begin{verbatim}
LW $t1, 8($t0)    # Address = $t0 + 8
SW $t2, -4($sp)   # Address = $sp + (-4)
\end{verbatim}

\subsubsection{Branch Instructions}

\textbf{Comparison Operation}:

\begin{verbatim}
Compare RS and RT for equality
Method: Subtract RT from RS
\end{verbatim}

\textbf{ALU Function}: SUBTRACTION

\begin{itemize}
\item Input A: RS register value
\item Input B: RT register value
\item Operation: SUB
\item ALU Control: 0110 (binary)
\item Result: RS - RT
\item Zero Flag: Indicates if result is zero (equal)

\textbf{Branch Decision}:

\begin{verbatim}
Zero = 1: RS == RT, take branch
Zero = 0: RS != RT, don't take branch
\end{verbatim}

\subsubsection{R-Type Instructions}

\textbf{Variable Operations}: Determined by funct field

\textbf{ALU Function}: DEPENDS ON FUNCT

\begin{itemize}
\item Input A: RS register value
\item Input B: RT register value
\item Operation: From funct field
\item ALU Control: Varies
\item Result: Written to RD register

\textbf{Funct Field Mapping}:

\begin{verbatim}
Funct    | Operation | ALU Control
---------|-----------|-------------
0x20     | ADD       | 0010
0x22     | SUB       | 0110
0x24     | AND       | 0000
0x25     | OR        | 0001
0x2A     | SLT       | 0111
\end{verbatim}

\subsection{ALU Control Signal}

\subsubsection{Signal Format}

\textbf{4-Bit Signal}: Specifies ALU operation

\textbf{Possible Operations} (2⁴ = 16):

\begin{verbatim}
0000: AND
0001: OR
0010: ADD
0110: SUBTRACT
0111: Set on Less Than (SLT)
1100: NOR
\end{verbatim}

\textbf{Usage}:

\begin{itemize}
\item Not all 16 combinations used
\item Could use 3 bits for 8 operations
\item 4-bit standard allows expansion

\subsubsection{Control Signal Usage by Instruction}

\textbf{Load/Store}:

\begin{itemize}
\item ALU Control = 0010 (ADD)
\item Fixed operation
\item Independent of instruction specifics

\textbf{Branch}:

\begin{itemize}
\item ALU Control = 0110 (SUBTRACT)
\item Fixed operation
\item Zero flag is critical output

\textbf{R-Type}:

\begin{itemize}
\item ALU Control = Varies
\item Must decode funct field
\item Different operations need different controls

\subsection{Two-Stage ALU Control Generation}

\subsubsection{Design Rationale}

\textbf{Why Two Stages?}

\textbf{Efficiency}:

\begin{itemize}
\item Some instructions don't need funct field
\item Separates opcode-level from operation-level
\item Faster for non-R-type instructions

\textbf{Timing Optimization}:

\begin{itemize}
\item Other control signals needed faster
\item Examples: Register addressing, immediate routing
\item ALU control can afford slight delay

\textbf{Modularity}:

\begin{itemize}
\item Stage 1: Main control (opcode-based)
\item Stage 2: ALU control (operation-specific)
\item Cleaner design separation

\subsubsection{Stage 1: Generate ALUOp}

\textbf{Input}: Opcode (6 bits)

\textbf{Output}: ALUOp (2 bits)

\textbf{Encoding}:

\begin{verbatim}
Instruction    | Opcode   | ALUOp
---------------|----------|-------
Load Word      | 100011   | 00
Store Word     | 101011   | 00
Branch Equal   | 000100   | 01
R-type         | 000000   | 10
\end{verbatim}

\textbf{ALUOp Meaning}:

\begin{itemize}
\item \textbf{00}: Perform ADD (address calculation)
\item \textbf{01}: Perform SUBTRACT (comparison)
\item \textbf{10}: Operation from funct field

\textbf{Logic}: Purely combinational based on opcode

\subsubsection{Stage 2: Generate ALU Control}

\textbf{Inputs}:

\begin{itemize}
\item ALUOp (2 bits from Stage 1)
\item Funct field (6 bits from instruction)
\item Total: 8 input bits

\textbf{Output}: ALU Control (4 bits)

\textbf{Truth Table}:

\begin{verbatim}
ALUOp | Funct   | ALU Control | Operation
------|---------|-------------|----------
00    | XXXXXX  | 0010        | ADD (LW/SW)
01    | XXXXXX  | 0110        | SUB (BEQ)
10    | 100000  | 0010        | ADD (R-type)
10    | 100010  | 0110        | SUB (R-type)
10    | 100100  | 0000        | AND
10    | 100101  | 0001        | OR
10    | 101010  | 0111        | SLT
\end{verbatim}

\textbf{"X" Notation}: Don't Care

\begin{itemize}
\item For ALUOp = 00 or 01, funct irrelevant
\item Simplifies logic design
\item Reduces gate count

\subsubsection{Complete ALU Control Path}

\textbf{Flow Diagram}:

\begin{verbatim}
Instruction Opcode (6 bits)
         $\downarrow$
   [Main Control Unit]
         $\downarrow$
    ALUOp (2 bits)  +  Funct Field (6 bits)
         $\downarrow$                    $\downarrow$
              [ALU Control Unit]
                     $\downarrow$
            ALU Control (4 bits)
                     $\downarrow$
                   [ALU]
\end{verbatim}

\textbf{Advantages}:

\begin{itemize}
\item Modular design
\item Simplified main control
\item Localized R-type complexity
\item Easier to verify

\subsection{Main Control Signals}

\subsubsection{Complete Signal List}

\textbf{Signals Generated}:

\begin{enumerate}
\item \textbf{RegDst} (1 bit): Register destination select
\item \textbf{Branch} (1 bit): Branch instruction indicator
\item \textbf{MemRead} (1 bit): Memory read enable
\item \textbf{MemtoReg} (1 bit): Memory to register select
\item \textbf{MemWrite} (1 bit): Memory write enable
\item \textbf{ALUSrc} (1 bit): ALU source select
\item \textbf{RegWrite} (1 bit): Register write enable
\item \textbf{ALUOp} (2 bits): To ALU control unit

\textbf{Total}: 9 control bits from main control

\subsubsection{RegDst (Register Destination)}

\textbf{Purpose}: Select which field specifies write destination

\textbf{Multiplexer Control}:

\begin{itemize}
\item Input 0: RT field (bits 16-20)
\item Input 1: RD field (bits 11-15)
\item Output: Register write address (5 bits)

\textbf{Settings}:

\begin{verbatim}
RegDst = 0: Write to RT (Load Word)
RegDst = 1: Write to RD (R-type)
\end{verbatim}

\textbf{Rationale}:

\begin{itemize}
\item Load Word: RT is destination (I-type format)
\item R-type: RD is destination (R-type format)
\item Store/Branch: Don't care (no write)

\textbf{Examples}:

\begin{verbatim}
LW $t1, 8($t0)     # Write to $t1 (RT) $\rightarrow$ RegDst = 0
ADD $t2, $t3, $t4  # Write to $t2 (RD) $\rightarrow$ RegDst = 1
\end{verbatim}

\subsubsection{Branch}

\textbf{Purpose}: Indicate if instruction is branch

\textbf{Usage}: Combined with Zero flag for PC selection

\textbf{Settings}:

\begin{verbatim}
Branch = 0: Not a branch (LW, SW, R-type)
Branch = 1: Branch instruction (BEQ, BNE)
\end{verbatim}

\textbf{PC Selection Logic}:

\begin{verbatim}
For BEQ:
  PCSrc = Branch AND Zero
  (Take branch if instruction is branch AND comparison equal)

For BNE:
  PCSrc = Branch AND NOT(Zero)
  (Take branch if instruction is branch AND comparison not equal)
\end{verbatim}

\subsubsection{MemRead}

\textbf{Purpose}: Enable reading from data memory

\textbf{Settings}:

\begin{verbatim}
MemRead = 0: No memory read (R-type, SW, BEQ)
MemRead = 1: Read from memory (LW)
\end{verbatim}

\textbf{Function}:

\begin{itemize}
\item Controls data memory read enable
\item When high: Memory outputs data
\item When low: Memory read inactive

\subsubsection{MemtoReg (Memory to Register)}

\textbf{Purpose}: Select source of register write data

\textbf{Multiplexer Control}:

\begin{itemize}
\item Input 0: ALU result
\item Input 1: Data memory read data
\item Output: Register write data (32 bits)

\textbf{Settings}:

\begin{verbatim}
MemtoReg = 0: Write ALU result (R-type)
MemtoReg = 1: Write memory data (LW)
\end{verbatim}

\textbf{Examples}:

\begin{verbatim}
ADD $t1, $t2, $t3  # $t1 = ALU result $\rightarrow$ MemtoReg = 0
LW $t1, 8($t0)     # $t1 = memory data $\rightarrow$ MemtoReg = 1
\end{verbatim}

\subsubsection{MemWrite}

\textbf{Purpose}: Enable writing to data memory

\textbf{Settings}:

\begin{verbatim}
MemWrite = 0: No memory write (R-type, LW, BEQ)
MemWrite = 1: Write to memory (SW)
\end{verbatim}

\textbf{Function}:

\begin{itemize}
\item Controls data memory write enable
\item When high: Data written (on clock edge)
\item When low: Memory write disabled

\subsubsection{ALUSrc (ALU Source)}

\textbf{Purpose}: Select second ALU operand source

\textbf{Multiplexer Control}:

\begin{itemize}
\item Input 0: Register file Read Data 2 (RT value)
\item Input 1: Sign-extended immediate
\item Output: ALU Input B (32 bits)

\textbf{Settings}:

\begin{verbatim}
ALUSrc = 0: Use register (R-type, BEQ)
ALUSrc = 1: Use immediate (LW, SW)
\end{verbatim}

\textbf{Examples}:

\begin{verbatim}
ADD $t1, $t2, $t3  # Use $t3 $\rightarrow$ ALUSrc = 0
LW $t1, 8($t0)     # Use imm 8 $\rightarrow$ ALUSrc = 1
\end{verbatim}

\subsubsection{RegWrite}

\textbf{Purpose}: Enable writing to register file

\textbf{Settings}:

\begin{verbatim}
RegWrite = 0: No register write (SW, BEQ)
RegWrite = 1: Write to register (R-type, LW)
\end{verbatim}

\textbf{Usage by Instruction}:

\begin{verbatim}
R-type:    RegWrite = 1 (write ALU result)
Load Word: RegWrite = 1 (write memory data)
Store Word: RegWrite = 0 (no write)
Branch:    RegWrite = 0 (no write)
\end{verbatim}

\subsection{Control Signal Truth Table}

\subsubsection{Complete Table}

\begin{verbatim}
Instruction | RegDst | ALUSrc | MemtoReg | RegWrite | MemRead | MemWrite | Branch | ALUOp
------------|--------|--------|----------|----------|---------|----------|--------|-------
R-type      |   1    |   0    |    0     |    1     |    0    |    0     |   0    |  10
Load Word   |   0    |   1    |    1     |    1     |    1    |    0     |   0    |  00
Store Word  |   X    |   1    |    X     |    0     |    0    |    1     |   0    |  00
Branch Eq   |   X    |   0    |    X     |    0     |    0    |    0     |   1    |  01
\end{verbatim}

\textbf{Legend}:

\begin{itemize}
\item \textbf{0}: Signal low/false/select input 0
\item \textbf{1}: Signal high/true/select input 1
\item \textbf{X}: Don't Care (not used, can be anything)

\subsubsection{R-Type Control}

\textbf{Settings}:

\begin{verbatim}
RegDst = 1:     Write to RD field
ALUSrc = 0:     Second operand from register (RT)
MemtoReg = 0:   Write ALU result
RegWrite = 1:   Enable register write
MemRead = 0:    No memory read
MemWrite = 0:   No memory write
Branch = 0:     Not a branch
ALUOp = 10:     Consult funct field
\end{verbatim}

\textbf{Active Elements}:

\begin{itemize}
\item Instruction fetch
\item Register file (read RS, RT; write RD)
\item ALU (operation from funct)
\item Register write from ALU
\item PC updated to PC + 4

\textbf{Inactive Elements}:

\begin{itemize}
\item Data memory (not accessed)
\item Branch target (computed but not used)
\item Sign extender (operates but ignored)

\subsubsection{Load Word Control}

\textbf{Settings}:

\begin{verbatim}
RegDst = 0:     Write to RT field
ALUSrc = 1:     Second operand from immediate
MemtoReg = 1:   Write memory data
RegWrite = 1:   Enable register write
MemRead = 1:    Enable memory read
MemWrite = 0:   No memory write
Branch = 0:     Not a branch
ALUOp = 00:     ALU performs ADD
\end{verbatim}

\textbf{Active Elements}:

\begin{itemize}
\item Instruction fetch
\item Register file (read RS; write RT)
\item Sign extender
\item ALU (ADD for address)
\item Data memory (read)
\item Register write from memory
\item PC updated to PC + 4

\textbf{Critical Path}: Longest delay

\begin{itemize}
\item Fetch $\rightarrow$ Reg Read $\rightarrow$ Sign Extend $\rightarrow$ ALU $\rightarrow$ Memory $\rightarrow$ Reg Write

\subsubsection{Store Word Control}

\textbf{Settings}:

\begin{verbatim}
RegDst = X:     Don't care (no register write)
ALUSrc = 1:     Second operand from immediate
MemtoReg = X:   Don't care (no register write)
RegWrite = 0:   No register write
MemRead = 0:    No memory read
MemWrite = 1:   Enable memory write
Branch = 0:     Not a branch
ALUOp = 00:     ALU performs ADD
\end{verbatim}

\textbf{Key Difference from Load}:

\begin{itemize}
\item Read TWO registers (RS for base, RT for data)
\item Memory write instead of read
\item No register write stage

\subsubsection{Branch if Equal Control}

\textbf{Settings}:

\begin{verbatim}
RegDst = X:     Don't care (no register write)
ALUSrc = 0:     Second operand from register (RT)
MemtoReg = X:   Don't care (no register write)
RegWrite = 0:   No register write
MemRead = 0:    No memory read
MemWrite = 0:   No memory write
Branch = 1:     This is a branch
ALUOp = 01:     ALU performs SUBTRACT
\end{verbatim}

\textbf{Active Elements}:

\begin{itemize}
\item Instruction fetch
\item Register file (read RS, RT)
\item ALU (SUBTRACT for comparison, Zero flag)
\item Sign extender + shift (branch target)
\item Branch target adder (PC + 4 + offset)
\item PC multiplexer (select based on Branch AND Zero)

\textbf{Branch Decision Logic}:

\begin{verbatim}
Zero = (RS - RT == 0)
PCSrc = Branch AND Zero
If PCSrc:
  Next PC = PC + 4 + (SignExtend(Imm) << 2)
Else:
  Next PC = PC + 4
\end{verbatim}

\subsection{Control Unit Implementation}

\subsubsection{Input to Control Unit}

\textbf{Primary Input}: Opcode (bits 26-31, 6 bits)

\begin{itemize}
\item Identifies instruction type
\item Determines all control signal values

\textbf{Secondary Input}: Funct field (bits 0-5, 6 bits)

\begin{itemize}
\item Only for R-type (opcode = 000000)
\item Specifies ALU operation

\subsubsection{Combinational Logic Design}

\textbf{Method}: Standard digital logic techniques

\textbf{Steps}:

\begin{enumerate}
\item Create truth table (opcode $\rightarrow$ control signals)
\item List all control signals as outputs
\item Fill in values for each instruction
\item Use Karnaugh maps or Boolean algebra to minimize
\item Implement with logic gates

\textbf{Example for RegWrite}:

\begin{verbatim}
RegWrite = (R-type) OR (Load Word)
RegWrite = (opcode == 000000) OR (opcode == 100011)
\end{verbatim}

\subsubsection{Control Unit Structure}

\textbf{ROM-Based Implementation}:

\begin{itemize}
\item Opcode as ROM address
\item ROM location stores control pattern
\item Simple but inflexible

\textbf{PLA (Programmable Logic Array)}:

\begin{itemize}
\item Implements minimized logic equations
\item More efficient than ROM
\item Standard for simple processors

\textbf{Hardwired Logic}:

\begin{itemize}
\item Custom logic gates
\item Fastest implementation
\item Most common for high-performance

\textbf{Microcode} (not typical for RISC):

\begin{itemize}
\item Control signals stored in memory
\item More flexible but slower
\item Used in CISC (e.g., x86)

\subsubsection{Timing Considerations}

\textbf{Signal Generation Time}:

\begin{itemize}
\item Must complete early in clock cycle
\item Before datapath elements need signals
\item Critical for clock frequency

\textbf{Signal Stability}:

\begin{itemize}
\item Must remain stable throughout cycle
\item Changes only between instructions
\item Combinational logic ensures this

\textbf{Clock Period Impact}:

\begin{itemize}
\item Control logic adds delay
\item Typically small vs. ALU/memory
\item Well-designed control has minimal impact

\subsection{Why Separate MemRead and MemWrite?}

\subsubsection{Initial Observation}

\textbf{Question}: Seem mutually exclusive—why not one signal?

\begin{itemize}
\item Could use: 0 = Read, 1 = Write
\item Appears redundant

\subsubsection{Answer: Yes, Separate Signals Needed}

\textbf{Timing Control}:

\begin{itemize}
\item Write Enable: Specifies WHEN to write
\item Read Enable: Specifies WHEN valid data available
\item Different timing requirements

\textbf{No Operation State}:

\begin{itemize}
\item Both = 0: No memory access
\item Common for R-type and branch
\item Single signal couldn't represent this

\textbf{Three States Required}:

\begin{verbatim}
MemRead=1, MemWrite=0: Read
MemRead=0, MemWrite=1: Write
MemRead=0, MemWrite=0: No access
(MemRead=1, MemWrite=1: Invalid)
\end{verbatim}

\subsubsection{Future: Pipelined Processors}

\textbf{Concurrent Access}:

\begin{itemize}
\item Different pipeline stages access memory
\item One stage reading, another writing
\item Separate signals essential

\textbf{Memory Banking}:

\begin{itemize}
\item Separate read/write ports
\item Enables simultaneous access
\item Separate signals control independent ports

\subsubsection{Design Philosophy}

\textbf{Orthogonality}:

\begin{itemize}
\item Each signal controls independent function
\item Easier to understand and verify
\item Reduces design errors

\textbf{Flexibility}:

\begin{itemize}
\item Supports future enhancements
\item Allows memory optimization
\item Standard practice

\subsection{Complete Datapath with Control}

\subsubsection{Integrated System}

\textbf{Components Connected}:

\begin{itemize}
\item Control Unit (generates signals)
\item Datapath (executes operations)
\item Blue lines: Control signals
\item Black lines: Data paths

\textbf{Control Unit Connections}:

\begin{itemize}
\item Input: Instruction opcode
\item Outputs: All control signals
\item Fan out to datapath elements

\textbf{ALU Control Unit}:

\begin{itemize}
\item Separate box near ALU
\item Inputs: ALUOp, Funct
\item Output: ALU Control (4 bits)

\subsubsection{Example: Load Word Execution}

\textbf{Instruction}: \texttt{LW $t1, 8($t0)}

\textbf{Step 1: Fetch}

\begin{verbatim}
PC $\rightarrow$ Instruction Memory
Opcode = 100011 (LW)
\end{verbatim}

\textbf{Step 2: Control Signals}

\begin{verbatim}
RegDst=0, ALUSrc=1, MemtoReg=1, RegWrite=1,
MemRead=1, MemWrite=0, Branch=0, ALUOp=00
\end{verbatim}

\textbf{Step 3: Register Read}

RS field ($t0) $\rightarrow$ Register file
Read Data 1 = $t0 value

\textbf{Step 4: ALU}

Immediate = 8
Sign-extended to 32 bits
ALUSrc=1: Selects immediate
ALU performs ADD: $t0 + 8 = address

\textbf{Step 5: Memory}

MemRead=1: Memory reads at address
Data output from memory

\textbf{Step 6: Write-Back}

MemtoReg=1: Selects memory data
RegDst=0: Selects RT ($t1)
RegWrite=1: Enables write
At clock edge: Memory data $\rightarrow$ $t1

\textbf{Step 7: PC Update}

Branch=0: PCSrc=0
PC updated to PC + 4

\subsection{Key Takeaways}

\begin{enumerate}
\item \textbf{Control unit generates signals based on instruction opcode}, orchestrating datapath operations.

\begin{enumerate}
\item \textbf{ALU control uses two-stage generation}: Opcode $\rightarrow$ ALUOp (2 bits) $\rightarrow$ ALU Control (4 bits).

\begin{enumerate}
\item \textbf{Stage 1 (Main Control)}: Opcode to ALUOp - identifies operation category.

\begin{enumerate}
\item \textbf{Stage 2 (ALU Control)}: ALUOp + Funct to ALU Control - specifies exact operation.

\begin{enumerate}
\item \textbf{Two-stage design optimizes timing and modularity}, separating concerns.

\begin{enumerate}
\item \textbf{Main control signals}: RegDst, Branch, MemRead, MemtoReg, MemWrite, ALUSrc, RegWrite, ALUOp.

\begin{enumerate}
\item \textbf{Load/Store always use ADD} for address calculation, regardless of other details.

\begin{enumerate}
\item \textbf{Branch uses SUBTRACT} for comparison, with Zero flag indicating equality.

\begin{enumerate}
\item \textbf{R-type ALU operation from funct field}, providing operation flexibility.

10. \textbf{Instruction format regularity simplifies control}, with consistent field positions.

11. \textbf{Register roles vary by instruction type}, especially RT (destination vs. source).

12. \textbf{Control signals mutually exclusive} for proper operation - only valid combinations used.

13. \textbf{Separate MemRead/MemWrite needed} for no-op state and future pipelining.

14. \textbf{Control logic is combinational} (no state), generating signals each cycle.

15. \textbf{Truth tables map opcode to control patterns}, enabling systematic design.

16. \textbf{"Don't care" values simplify logic minimization}, reducing gate count.

17. \textbf{Control unit design uses standard digital logic techniques}, including K-maps and Boolean algebra.

18. \textbf{Datapath elements may operate but outputs ignored} if not selected by control signals.

19. \textbf{Complete processor integrates datapath and control}, with control signals orchestrating all operations.

20. \textbf{Single-cycle design simple but inefficient} - foundation for advanced multi-cycle and pipelined designs.

\subsection{Summary}

The control unit completes the single-cycle MIPS processor, generating control signals that orchestrate datapath operations based on instruction opcodes. The two-stage ALU control generation (opcode $\rightarrow$ ALUOp $\rightarrow$ ALU Control) elegantly separates concerns, with the main control handling instruction-level decisions and the ALU control handling operation-specific details. Each control signal serves a specific purpose, from selecting multiplexer inputs (RegDst, ALUSrc, MemtoReg) to enabling register and memory operations (RegWrite, MemRead, MemWrite) to handling branches (Branch). Truth tables systematically map instructions to control patterns, with "don't care" values simplifying logic design. While the single-cycle processor provides conceptual clarity and simplicity, its inefficiency (all instructions taking the same time as the slowest) motivates more sophisticated designs. Understanding this foundation prepares us for multi-cycle processors (which break execution into variable-length stages) and pipelined processors (which overlap instruction execution for higher throughput), both building on the control principles established here.

% \section{Lecture 11: Complete Single-Cycle MIPS Processor and Performance Analysis}

\emph{By Dr. Isuru Nawinne}

\subsection{Introduction}

This lecture completes the single-cycle MIPS processor design by providing comprehensive analysis of control signals for all instruction types (R-type, Branch, Load, Store, Jump), introducing detailed timing analysis with concrete delay values, and demonstrating the fundamental performance limitations that motivate the evolution toward multi-cycle and pipelined implementations. We build upon previous datapath and control unit knowledge to create a functioning processor while understanding why single-cycle design, though conceptually simple, proves inefficient in practice.

\subsection{Lecture Overview and Context}

\subsubsection{Recap from Previous Lectures}

The foundational work completed in previous lectures includes:

\textbf{Completed Topics:}

\begin{itemize}
\item Datapath components: Register file, ALU, memories, adders, multiplexers
\item Sign extension and shifting for immediate operands
\item Control unit concept and ALU control generation
\item Control signal purposes and functions
\end{itemize}

\textbf{Current Focus:}

\begin{itemize}
\item Complete control signal analysis for all instructions
\item Detailed walkthrough of instruction execution
\item Jump instruction integration
\item Timing analysis with concrete delay values
\item Performance limitations of single-cycle design
\end{itemize}

\subsubsection{Instruction Subset Review}

\textbf{Selected Instructions for Study:}

\begin{itemize}
\item \textbf{R-type}: ADD, SUB, AND, OR (arithmetic/logic operations)
\item \textbf{Load Word (LW)}: Memory read
\item \textbf{Store Word (SW)}: Memory write
\item \textbf{Branch if Equal (BEQ)}: Conditional branch
\item \textbf{Jump (J)}: Unconditional jump
\end{itemize}

\textbf{Coverage:}

\begin{itemize}
\item Represents 95\% of MIPS microarchitecture hardware
\item Comprehensive enough for understanding design principles
\item Omits some I-type arithmetic (covered conceptually)
\item Foundation for complete processor understanding
\end{itemize}

\subsection{Control Unit Inputs and Outputs}

\subsubsection{Control Unit Inputs}

\textbf{Total Input Bits:} 12 bits

\paragraph{Primary Input - Opcode (6 bits):}

\begin{itemize}
\item Bits 26-31 of instruction
\item Identifies instruction type
\item Used for almost all control signal generation
\item Most significant determinant of control behavior
\end{itemize}

\paragraph{Secondary Input - Funct Field (6 bits):}

\begin{itemize}
\item Bits 0-5 of instruction
\item Only relevant for R-type instructions (opcode = 000000)
\item Specifies ALU operation for R-type
\item Ignored for I-type and J-type instructions
\end{itemize}

\textbf{Usage Pattern:}

\begin{itemize}
\item Opcode always examined
\item Funct field examined only when opcode = 0 (R-type)
\item Combined with ALUOp for final ALU control signal
\end{itemize}

\subsubsection{Control Unit Outputs}

\textbf{Total Output Bits:} 9 bits (8 signals, one is 2-bit)

\textbf{Control Signals Generated:}

\begin{enumerate}
\item \textbf{RegDst} (1 bit): Select register write address
\item \textbf{Branch} (1 bit): Instruction is branch type
\item \textbf{MemRead} (1 bit): Enable memory read
\item \textbf{MemtoReg} (1 bit): Select register write data source
\item \textbf{MemWrite} (1 bit): Enable memory write
\item \textbf{ALUSrc} (1 bit): Select ALU second operand source
\item \textbf{RegWrite} (1 bit): Enable register file write
\item \textbf{ALUOp} (2 bits): ALU operation category
\end{enumerate}

\textbf{Additional Signal for Jump:}

\begin{enumerate}
\item \textbf{Jump} (1 bit): Select jump target for PC
\end{enumerate}

\textbf{Implementation:}

\begin{itemize}
\item Combinational logic circuit
\item Inputs: Opcode and Funct field
\item Outputs: Control signals
\item Design method: Truth tables, Karnaugh maps, Boolean minimization
\item To be implemented in Lab 5
\end{itemize}

\subsection{R-Type Instruction Detailed Analysis}

\subsubsection{Instruction Format}

\textbf{Encoding Structure (32 bits):}

\begin{itemize}
\item \textbf{Bits 26-31}: Opcode = 000000 (0) - ALL R-type instructions
\item \textbf{Bits 21-25}: RS (5 bits) - First source register
\item \textbf{Bits 16-20}: RT (5 bits) - Second source register
\item \textbf{Bits 11-15}: RD (5 bits) - Destination register
\item \textbf{Bits 6-10}: SHAMT (5 bits) - Shift amount
\item \textbf{Bits 0-5}: Funct (6 bits) - Function code (specifies operation)

\textbf{Example: ADD \$1, \$2, \$3}

Encoding: 000000 00010 00011 00001 00000 100000
         |Opcode| RS  | RT  | RD  |SHAMT| Funct |
         |  0   |  2  |  3  |  1  |  0  |  32   |

\textbf{Operation:} \texttt{$1 = $2 + $3}

\subsubsection{Datapath Elements Used}

\textbf{Active Elements (shown in black):}

\begin{itemize}
\item Instruction Memory: Fetch instruction
\item Program Counter: Current instruction address
\item PC + 4 Adder: Calculate next sequential address
\item Register File: Read RS, RT; Write RD
\item Multiplexer (RegDst): Select RD as write address
\item Multiplexer (ALUSrc): Select RT value (not immediate)
\item ALU: Perform operation specified by funct
\item Multiplexer (MemtoReg): Select ALU result (not memory)
\item Multiplexer (PC source): Select PC+4 (not branch)

\textbf{Inactive Elements (grayed out):}

\begin{itemize}
\item Data Memory: Not accessed
\item Sign Extender: Not used (no immediate value)
\item Branch Target Adder: Calculated but not used
\item Shift Left 2: Not used

\subsubsection{Control Signal Values for R-Type}

\textbf{Exercise Example: ADD \$1, \$2, \$3}

\begin{table}[h]
\centering
\begin{tabular}{|l|l|p{7cm}|}
\hline
\textbf{Signal} & \textbf{Value} & \textbf{Reason} \\
\hline
RegDst & 1 & Write to RD (bits 11-15), not RT \\
Branch & 0 & Not a branch instruction \\
MemRead & 0 & Not reading from memory \\
MemtoReg & 0 & Write ALU result (not memory data) \\
ALUOp & 10 & R-type: Consult funct field \\
MemWrite & 0 & Not writing to memory \\
ALUSrc & 0 & Second operand from register RT (not immediate) \\
RegWrite & 1 & Write result to destination register \\
\hline
\end{tabular}
\end{table}

\textbf{Detailed Explanations:}

\textbf{RegDst = 1:}

\begin{itemize}
\item Multiplexer selects input 1
\item Input 1: Bits 11-15 (RD field)
\item Input 0: Bits 16-20 (RT field)
\item R-type destination always in RD

\textbf{Branch = 0:}

\begin{itemize}
\item Not a branch instruction
\item Branch control AND Zero $\rightarrow$ 0 AND X = 0
\item PC source multiplexer selects PC+4

\textbf{MemRead = 0, MemWrite = 0:}

\begin{itemize}
\item R-type doesn't access data memory
\item Memory control signals disabled
\item Data memory outputs ignored (don't care)

\textbf{MemtoReg = 0:}

\begin{itemize}
\item Multiplexer selects ALU result
\item Not memory data (memory not accessed)
\item ALU result goes to register write data

\textbf{ALUOp = 10 (binary):}

\begin{itemize}
\item Indicates R-type instruction
\item ALU Control Unit examines funct field
\item For ADD: funct = 100000 $\rightarrow$ ALU Control = 0010 (ADD)

\textbf{ALUSrc = 0:}

\begin{itemize}
\item Multiplexer selects register value
\item Register file Read Data 2 (RT value)
\item Not sign-extended immediate

\textbf{RegWrite = 1:}

\begin{itemize}
\item Enable register file write
\item Result written to RD at clock edge
\item Essential for saving computation result

\subsubsection{Execution Steps for R-Type}

\textbf{Step 1: Instruction Fetch}

\begin{itemize}
\item PC value $\rightarrow$ Instruction Memory address
\end{itemize}
\item Instruction word retrieved
\item Opcode (000000) sent to Control Unit

\textbf{Step 2: Control Signal Generation}

\begin{itemize}
\item Control Unit decodes opcode = 0
\item Identifies R-type instruction
\item Generates all control signals
\item Sends funct field to ALU Control

\textbf{Step 3: Register Read}

\begin{itemize}
\item RS field (00010 = 2) $\rightarrow$ Read Address 1
\item RT field (00011 = 3) $\rightarrow$ Read Address 2
\item Read Data 1 = $2 value
\item Read Data 2 = $3 value

\textbf{Step 4: ALU Operation}

\begin{itemize}
\item ALUSrc = 0: Select RT value for Input B
\item Input A = $2 value, Input B = $3 value
\item ALU Control = 0010 (ADD operation)
\item ALU Result = $2 + $3

\textbf{Step 5: Register Write Preparation}

\begin{itemize}
\item MemtoReg = 0: Select ALU result
\item RegDst = 1: Select RD (00001 = 1)
\item Write Data = ALU result
\item Write Address = $1

\textbf{Step 6: Clock Edge Actions}

\begin{itemize}
\item RegWrite = 1 enabled
\item ALU result written to $1
\item PC updated to PC + 4
\item Next instruction fetch begins

\subsection{Branch If Equal Instruction Detailed Analysis}

\subsubsection{Instruction Format}

\textbf{Encoding Structure (32 bits):}

\begin{itemize}
\item \textbf{Bits 26-31}: Opcode = 000100 (4) - BEQ
\item \textbf{Bits 21-25}: RS (5 bits) - First comparison register
\item \textbf{Bits 16-20}: RT (5 bits) - Second comparison register
\item \textbf{Bits 0-15}: Immediate (16 bits) - Branch offset (in instructions)

\textbf{Example: BEQ \$1, \$2, 100}

Encoding: 000100 00001 00010 0000000001100100
         |Opcode|  RS |  RT |    Immediate      |
         |  4   |  1  |  2  |       100         |

\textbf{Operation:} \texttt{If ($1 == $2) then PC = PC + 4 + (100 $\times$ 4)}

\subsubsection{Datapath Elements Used}

\textbf{Active Elements:}

\begin{itemize}
\item Instruction Memory: Fetch instruction
\item Program Counter & PC+4 Adder
\item Register File: Read RS, RT (no write)
\item ALU: Subtract RT from RS
\item Zero Flag: Compare result to zero
\item Sign Extender: Extend 16-bit offset to 32-bit
\item Shift Left 2: Convert word offset to byte offset
\item Branch Target Adder: Calculate PC + 4 + (offset $\times$ 4)
\item AND Gate: Combine Branch signal and Zero flag
\item PC Source Multiplexer: Select next PC value

\textbf{Inactive Elements:}

\begin{itemize}
\item Data Memory: Not accessed
\item Register Write: Not writing to registers
\item ALU Result (except Zero flag): Not used

\subsubsection{Control Signal Values for BEQ}

\textbf{Exercise Example: BEQ \$1, \$2, 100}

\begin{table}[h]
\centering
\begin{tabular}{|l|l|p{7cm}|}
\hline
\textbf{Signal} & \textbf{Value} & \textbf{Reason} \\
\hline
RegDst & X & Don't care (not writing to register) \\
Branch & 1 & This IS a branch instruction \\
MemRead & 0 & Not reading from memory \\
MemtoReg & X & Don't care (not writing to register) \\
ALUOp & 01 & Perform SUBTRACT for comparison \\
MemWrite & 0 & Not writing to memory \\
ALUSrc & 0 & Compare two register values (not immediate) \\
RegWrite & 0 & Not writing to register file \\
\hline
\end{tabular}
\end{table}

\textbf{Detailed Explanations:}

\textbf{RegDst = X (Don't Care):}

\begin{itemize}
\item RegWrite = 0, so write address irrelevant
\item No register write operation
\item Multiplexer output ignored
\item Using X simplifies Boolean logic

\textbf{Branch = 1:}

\begin{itemize}
\item Identifies instruction as branch type
\item Feeds into AND gate with Zero flag
\item PCSrc = Branch AND Zero
\item If Zero = 1 (values equal): Take branch
\item If Zero = 0 (values differ): Don't take branch

\textbf{MemRead = 0, MemWrite = 0:}

\begin{itemize}
\item Branch doesn't access memory
\item Memory control signals disabled

\textbf{MemtoReg = X (Don't Care):}

\begin{itemize}
\item RegWrite = 0, so write data source irrelevant
\item Multiplexer output ignored

\textbf{ALUOp = 01:}

\begin{itemize}
\item Specifies SUBTRACT operation
\item ALU Control receives 01
\item Generates ALU Control = 0110 (SUB)
\item Independent of funct field

\textbf{ALUSrc = 0:}

\begin{itemize}
\item Need RT value from register (not immediate)
\item Immediate used for branch target (not ALU input)
\item ALU compares RS and RT register values

\textbf{RegWrite = 0:}

\begin{itemize}
\item Branch doesn't modify registers
\item Essential to prevent accidental writes
\item If =1, would corrupt register file

\subsubsection{Branch Target Calculation}

\textbf{Word Offset to Byte Offset:}

\begin{itemize}
\item Immediate field: 100 (in instructions/words)
\item Sign extend to 32 bits: 0x00000064
\item Shift left by 2: 0x00000190 (multiply by 4)
\item Result: 400 bytes (100 instructions $\times$ 4 bytes/instruction)
\end{itemize}

\textbf{Branch Target Address:}

\begin{itemize}
\item Current PC + 4: Address of next sequential instruction
\item Offset: 400 bytes
\item Branch Target = (PC + 4) + 400

\textbf{Example:}

\begin{itemize}
\item Current instruction at address 1000
\item PC + 4 = 1004
\item Branch Target = 1004 + 400 = 1404
\item If branch taken: Next instruction at 1404
\item If branch not taken: Next instruction at 1004

\textbf{PCSrc Selection:}

PCSrc = Branch AND Zero
      = 1 AND (RS == RT ? 1 : 0)

If PCSrc = 1: PC $\leftarrow$ Branch Target (1404)
If PCSrc = 0: PC $\leftarrow$ PC + 4 (1004)

\subsection{Load Word Instruction Detailed Analysis}

\subsubsection{Instruction Format}

\textbf{Encoding Structure (32 bits):}

\begin{itemize}
\item \textbf{Bits 26-31}: Opcode = 100011 (35) - LW
\item \textbf{Bits 21-25}: RS (5 bits) - Base address register
\item \textbf{Bits 16-20}: RT (5 bits) - Destination register
\item \textbf{Bits 0-15}: Immediate (16 bits) - Address offset

\textbf{Example: LW \$8, 32(\$9)}

Encoding: 100011 01001 01000 0000000000100000
         |Opcode|  RS |  RT |    Immediate      |
         | 35   |  9  |  8  |        32         |

\textbf{Operation:} \texttt{$8 = Memory[$9 + 32]}

\subsubsection{Datapath Elements Used}

\textbf{Active Elements:}

\begin{itemize}
\item Instruction Memory: Fetch instruction
\item Program Counter & PC+4 Adder
\item Register File: Read RS (base); Write RT (destination)
\item Sign Extender: Extend offset to 32 bits
\item Multiplexer (ALUSrc): Select immediate
\item ALU: Add base + offset
\item Data Memory: Read at calculated address
\item Multiplexer (MemtoReg): Select memory data
\item Multiplexer (RegDst): Select RT for write

\textbf{Inactive Elements:}

\begin{itemize}
\item Second register read (RT as source): Not used
\item Branch circuitry: Not used

\subsubsection{Control Signal Values for LW}

\textbf{Exercise Example: LW \$8, 32(\$9)}

\begin{table}[h]
\centering
\begin{tabular}{|l|l|p{7cm}|}
\hline
\textbf{Signal} & \textbf{Value} & \textbf{Reason} \\
\hline
RegDst & 0 & Write to RT (bits 16-20), not RD \\
Branch & 0 & Not a branch instruction \\
MemRead & 1 & Reading from data memory \\
MemtoReg & 1 & Write memory data (not ALU result) \\
ALUOp & 00 & Perform ADD for address calculation \\
MemWrite & 0 & Not writing to memory (reading only) \\
ALUSrc & 1 & Add immediate offset (not register) \\
RegWrite & 1 & Write loaded data to destination register \\
\hline
\end{tabular}
\end{table}

\textbf{Detailed Explanations:}

\textbf{RegDst = 0:}

\begin{itemize}
\item I-type format: Destination in RT field
\item Multiplexer selects bits 16-20
\item RT = 01000 (register 8)
\item Different from R-type (RD field)

\textbf{Branch = 0:}

\begin{itemize}
\item Sequential execution
\item PC updated to PC + 4

\textbf{MemRead = 1:}

\begin{itemize}
\item Enable data memory read
\item Essential for memory timing
\item Memory outputs data at calculated address
\item If 0: Memory output undefined (ignored anyway)

\textbf{MemtoReg = 1:}

\begin{itemize}
\item Multiplexer selects memory data
\item Input 1: Data memory read output
\item Input 0: ALU result (address, not data!)
\item Must select memory data for load

\textbf{ALUOp = 00:}

\begin{itemize}
\item Address calculation requires ADD
\item Base address + offset
\item ALU Control = 0010 (ADD)

\textbf{MemWrite = 0:}

\begin{itemize}
\item Reading, not writing
\item Critical: Prevents memory corruption
\item If 1: Would write garbage to memory

\textbf{ALUSrc = 1:}

\begin{itemize}
\item Need immediate offset for address calculation
\item Multiplexer selects sign-extended immediate
\item Input 1: Sign-extended offset
\item Input 0: RT value (not used for address calc)

\textbf{RegWrite = 1:}

\begin{itemize}
\item Must write loaded data to RT
\item Data from memory $\rightarrow$ Register $8
\item If 0: Data lost, load ineffective

\subsubsection{Critical Path for Load Word}

\textbf{Longest Delay in Single-Cycle:}

\begin{enumerate}
\item Instruction Memory read
\item Register File read (base address)
\item Sign Extension
\item ALU address calculation
\item Data Memory read
\item Register write setup

\textbf{Load Word is the slowest instruction!}

\begin{itemize}
\item Determines minimum clock period
\item All other instructions must wait for this worst case
\item Major performance bottleneck

\subsection{Store Word Instruction Detailed Analysis}

\subsubsection{Instruction Format}

\textbf{Encoding Structure (32 bits):}

\begin{itemize}
\item \textbf{Bits 26-31}: Opcode = 101011 (43) - SW
\item \textbf{Bits 21-25}: RS (5 bits) - Base address register
\item \textbf{Bits 16-20}: RT (5 bits) - Source data register
\item \textbf{Bits 0-15}: Immediate (16 bits) - Address offset

\textbf{Example: SW \$8, 32(\$9)}

Encoding: 101011 01001 01000 0000000000100000
         |Opcode|  RS |  RT |    Immediate      |
         | 43   |  9  |  8  |        32         |

\textbf{Operation:} \texttt{Memory[$9 + 32] = $8}

_Note: Fixed error in lecture (was "$32", should be "32")_

\subsubsection{Datapath Elements Used}

\textbf{Active Elements:}

\begin{itemize}
\item Instruction Memory
\item Program Counter & PC+4 Adder
\item Register File: Read RS (base) AND RT (data source)
\item Sign Extender
\item Multiplexer (ALUSrc): Select immediate
\item ALU: Add base + offset
\item Data Memory: Write RT data at calculated address

\textbf{Inactive Elements:}

\begin{itemize}
\item Register Write: No register write
\item Memory Read: Writing, not reading
\item MemtoReg multiplexer: Output not used

\textbf{Key Difference from Load:}

\begin{itemize}
\item TWO register reads: RS for base, RT for data
\item Memory write instead of read
\item NO register write operation

\subsubsection{Control Signal Values for SW}

\textbf{Exercise Example: SW \$8, 32(\$9)}

\begin{table}[h]
\centering
\begin{tabular}{|l|l|p{7cm}|}
\hline
\textbf{Signal} & \textbf{Value} & \textbf{Reason} \\
\hline
RegDst & X & Don't care (not writing to register) \\
Branch & 0 & Not a branch instruction \\
MemRead & 0 & Not reading from memory (writing) \\
MemtoReg & X & Don't care (not writing to register) \\
ALUOp & 00 & Perform ADD for address calculation \\
MemWrite & 1 & Writing to data memory \\
ALUSrc & 1 & Add immediate offset \\
RegWrite & 0 & Not writing to register file \\
\hline
\end{tabular}
\end{table}

\textbf{Detailed Explanations:}

\textbf{RegDst = X (Don't Care):}

\begin{itemize}
\item RegWrite = 0: No register write
\item Write address irrelevant
\item Could be 0 or 1, doesn't matter
\item Using X simplifies logic design

\textbf{CRITICAL: RegWrite = 0:}

\begin{itemize}
\item Must prevent register file write
\item \textbf{If RegWrite = 1:} Disaster!
\item Some register address fed to write port
\item Either ALU result (address) or memory data (garbage, MemRead=0)
\item Would corrupt random register
\item Data integrity violated

\textbf{Why It Matters:}

\begin{itemize}
\item Hardware operates in parallel
\item Multiplexers produce outputs even if not used
\item Without RegWrite = 0:
\item RegDst mux outputs some address
\item MemtoReg mux outputs some data
\item If RegWrite = 1: This garbage written to register!
\item Control signal correctness essential

\textbf{MemRead = 0, MemWrite = 1:}

\begin{itemize}
\item Writing to memory, not reading
\item MemRead = 0: Memory read output undefined
\item MemWrite = 1: Memory accepts write data
\item Opposite of Load Word

\textbf{MemtoReg = X (Don't Care):}

\begin{itemize}
\item RegWrite = 0: Write data source irrelevant
\item Output not used
\item Even if wrong data selected, RegWrite prevents write

\textbf{ALUOp = 00:}

\begin{itemize}
\item Same as Load Word
\item Address calculation: ADD operation

\textbf{ALUSrc = 1:}

\begin{itemize}
\item Need immediate offset
\item Same as Load Word

\subsubsection{Important Lesson: Don't Care vs Zero}

\textbf{Student Confusion:}
_"RegDst = 0 is not wrong, but best answer is X"_

\textbf{Clarification:}

\begin{itemize}
\item \textbf{Functionally:} 0 works (doesn't cause error)
\item \textbf{Logically:} X is correct (truly doesn't matter)
\item \textbf{Design perspective:} X simplifies Boolean expressions
\item \textbf{Karnaugh map minimization:} X allows more groupings

\textbf{However:}

\begin{itemize}
\item RegWrite MUST be 0 (not X!)
\item MemWrite MUST be correct (not X!)
\item Read/Write enables are critical for data integrity

\subsection{Jump Instruction Integration}

\subsubsection{Instruction Format}

\textbf{Encoding Structure (32 bits):}

\begin{itemize}
\item \textbf{Bits 26-31}: Opcode = 000010 (2) - J
\item \textbf{Bits 0-25}: Address (26 bits) - Jump target (word address)

\textbf{Alternative: JAL (Jump and Link)}

\begin{itemize}
\item Opcode = 000011 (3)
\item Used for function calls
\item Saves return address in register $31

\textbf{Example: J 100}

Encoding: 000010 00000000000000000001100100
         |Opcode|        Target Address        |
         |  2   |            100               |

\textbf{Operation:} \texttt{PC = {PC+4[31:28], Address, 2'b00}}

\subsubsection{Jump Target Address Calculation}

\textbf{Word Address to Byte Address:}

\begin{itemize}
\item Target field: 26 bits (word address)
\item Shift left by 2: Append 2 zero bits
\item Result: 28-bit byte address

\textbf{Upper 4 Bits:}

\begin{itemize}
\item Take from PC+4 current value
\item Bits 31:28 of next sequential instruction
\item Preserves region (256 MB regions)
\item Jump within same region as current PC

\textbf{Concatenation:}

\begin{verbatim}
PC+4:         [31:28] [27:2] [1:0]
                |       (ignored)
Jump Target:  [31:28] [Target×4] [00]
                |       |          |
              From    From       Append
              PC+4    instruction zeros
\end{verbatim}

\textbf{Example:}

\begin{itemize}
\item PC = 0x10000000
\item PC+4 = 0x10000004
\item Target = 100 = 0x000064
\item Shift left 2: 0x000190
\item Upper 4 bits: 0x1
\item Jump Address: 0x10000190

\textbf{Limitation:}

\begin{itemize}
\item Can only jump within 28-bit range (256 MB)
\item Upper 4 bits fixed by current PC region
\item For larger jumps: Use jump register (JR) instruction

\subsubsection{Additional Datapath Hardware}

\textbf{New Components:}

\textbf{Shift Left 2 (for jump):}

\begin{itemize}
\item Input: 26-bit target field
\item Output: 28-bit byte offset
\item Implementation: Wire routing (no actual shifter!)

\textbf{Concatenation Logic:}

\begin{itemize}
\item Input 1: PC+4 bits [31:28] (4 bits)
\item Input 2: Shifted target (28 bits)
\item Output: 32-bit jump address
\item Implementation: Wire concatenation

\textbf{New Multiplexer:}

\begin{itemize}
\item \textbf{Input 0}: Output from branch/sequential mux
\item Could be PC+4 or branch target
\item \textbf{Input 1}: Jump target address (32 bits)
\item \textbf{Select}: Jump control signal
\item \textbf{Output}: Next PC value

\textbf{Original PC Source Mux:}

\begin{itemize}
\item Input 0: PC + 4
\item Input 1: Branch target
\item Select: PCSrc (Branch AND Zero)

\textbf{New Jump Mux (outer):}

\begin{itemize}
\item Input 0: Original mux output (PC+4 or branch target)
\item Input 1: Jump target
\item Select: Jump signal
\item Output: Final next PC value

\subsubsection{Jump Control Signal}

\textbf{Jump Signal:}

\begin{itemize}
\item 10th control output bit
\item Generated by Control Unit
\item Based on opcode = 2 (J) or 3 (JAL)

\textbf{Values:}

\begin{itemize}
\item Jump = 1: Select jump target
\item Jump = 0: Select sequential/branch

\textbf{Other Control Signals for Jump:}

\begin{table}[h]
\centering
\begin{tabular}{|l|l|p{7cm}|}
\hline
\textbf{Signal} & \textbf{Value} & \textbf{Reason} \\
\hline
RegDst & X & Don't care \\
Branch & 0 & Not a branch (different mechanism) \\
MemRead & 0 & Not accessing memory \\
MemtoReg & X & Don't care \\
ALUOp & XX & Don't care (ALU not used) \\
MemWrite & 0 & Not writing memory \\
ALUSrc & X & Don't care \\
RegWrite & 0 & Not writing register (J instruction) \\
Jump & 1 & This IS a jump instruction \\
\hline
\end{tabular}
\end{table}

\textbf{Note: JAL (Jump and Link) different:}

\begin{itemize}
\item RegWrite = 1 (saves return address)
\item RegDst = ? (special: write to $31)
\item Additional logic needed for return address

\subsubsection{Complete Datapath with Jump}

\textbf{All Instruction Types Supported:}

\begin{itemize}
\item \textbf{R-type}: Arithmetic, logic, shift
\item \textbf{I-type}: Load, Store, Branch, Immediate arithmetic
\item \textbf{J-type}: Jump, Jump and Link

\textbf{Coverage:}

\begin{itemize}
\item 95%+ of MIPS ISA hardware
\item Complete single-cycle implementation
\item Additional variants (BNE, shifts, etc.) need minor additions

\textbf{Datapath Completeness:}

\begin{itemize}
\item Two memories: Instruction and Data
\item One ALU for computation
\item Multiple adders: PC+4, Branch target
\item Many multiplexers for data routing
\item Sign extender
\item Shift left 2 circuits (wire routing)
\item Control unit with 10 control signal bits

\subsection{Timing Analysis with Concrete Delays}

\subsubsection{Assumed Component Delays}

\textbf{Delay Values (in nanoseconds):}

\begin{table}[h]
\centering
\begin{tabular}{|l|l|p{6cm}|}
\hline
\textbf{Component} & \textbf{Delay} & \textbf{Notes} \\
\hline
Instruction Memory & 2 ns & Read instruction at PC address \\
Register File (Read) & 1 ns & Output data after address change \\
Register File (Write) & 1 ns & At clock edge (next cycle) \\
Sign Extender & $\sim$0 ns & Negligible (wire replication) \\
Multiplexers & $\sim$0 ns & Negligible compared to other delays \\
ALU Operation & 2 ns & Arithmetic/logic/comparison \\
Data Memory (Read) & 2 ns & Output data after address provided \\
Data Memory (Write) & 2 ns & At clock edge (next cycle) \\
PC+4 Adder & 2 ns & Simple addition \\
Branch Target Adder & 2 ns & Addition with offset \\
\hline
\end{tabular}
\end{table}

\textbf{Assumptions:}

\begin{itemize}
\item Simplified for analysis
\item Real delays depend on technology, circuit design
\item Memory accesses typically slowest
\item Combinational logic relatively fast

\subsubsection{Critical Path Analysis}

\textbf{Definition:}

\begin{itemize}
\item Longest delay path from clock edge to clock edge
\item Determines minimum clock period
\item All combinational logic between sequential elements

\textbf{Single-Cycle Constraint:}

\begin{itemize}
\item Entire instruction must complete in one clock cycle
\item Clock period $\geq$ Critical path delay
\item All instructions take same time (worst case)

\subsubsection{Load Word Instruction Timing}

\textbf{Step-by-Step Delay Calculation:}

\textbf{Step 1: Instruction Fetch (2 ns)}

\begin{itemize}
\item Clock edge: PC updated
\item PC $\rightarrow$ Instruction Memory
\item Instruction Memory reads and outputs instruction
\item Delay: 2 ns
\item Running total: 2 ns

\textbf{Step 2: Register Read (1 ns)}

\begin{itemize}
\item Instruction decoded
\item RS field extracted
\item RS $\rightarrow$ Register File Read Address 1
\item Register File outputs base address
\item Delay: 1 ns
\item Running total: 2 + 1 = 3 ns

\textbf{Step 3: Sign Extension (\textasciitilde{}0 ns)}

\begin{itemize}
\item Immediate field extracted
\item Sign extended to 32 bits
\item Delay: Negligible
\item Running total: ~3 ns

\textbf{Step 4: ALU Address Calculation (2 ns)}

\begin{itemize}
\item Base address + offset
\item ALU performs addition
\item Output: Memory address
\item Delay: 2 ns
\item Running total: 3 + 2 = 5 ns

\textbf{Step 5: Memory Read (2 ns)}

\begin{itemize}
\item ALU result $\rightarrow$ Data Memory address
\item MemRead = 1 asserted
\item Data Memory reads and outputs data
\item Delay: 2 ns
\item Running total: 5 + 2 = 7 ns

\textbf{Step 6: Register Write Setup (\textasciitilde{}0 ns)}

\begin{itemize}
\item Memory data $\rightarrow$ Register Write Data input
\item RT $\rightarrow$ Register Write Address
\item Ready for clock edge
\item Delay: Setup time negligible
\item Running total: ~7 ns

\textbf{Clock Edge: Register Write (next cycle)}

\begin{itemize}
\item At next positive clock edge
\item Data written to register
\item Takes 1 ns but in next cycle

\textbf{Minimum Clock Period:} 7 nanoseconds  
\textbf{Maximum Clock Frequency:} 1/7 ns ≈ 143 MHz

\textbf{Load Word is Critical Path!}

\begin{itemize}
\item Longest instruction in single-cycle design
\item Determines clock period for ALL instructions

\subsubsection{Store Word Instruction Timing}

\textbf{Step-by-Step Delay:}

\begin{enumerate}
\item \textbf{Instruction Fetch:} 2 ns (total: 2 ns)
\item \textbf{Register Read:} 1 ns (total: 3 ns)
\item Read RS (base) AND RT (data)
\item \textbf{Sign Extension:} ~0 ns (total: 3 ns)
\item \textbf{ALU Address Calculation:} 2 ns (total: 5 ns)
\item \textbf{Memory Write Setup:} ~0 ns (total: 5 ns)
\item Address and data ready at memory inputs

\textbf{Clock Edge: Memory Write (end of cycle)}

\begin{itemize}
\item Data written to memory at clock edge
\item Takes 2 ns but next instruction fetch also 2 ns
\item Next instruction register read starts after 3 ns total
\item Memory write completes before register read needs data
\item No conflict

\textbf{Minimum Time Required:} 5 nanoseconds

\textbf{Note:}

\begin{itemize}
\item Faster than Load Word (no memory read delay)
\item But must use 7 ns clock period anyway (single-cycle)
\item Wastes 2 ns per Store instruction

\subsubsection{Arithmetic Instruction Timing (ADD, SUB, AND, OR)}

\textbf{Step-by-Step Delay:}

\begin{enumerate}
\item \textbf{Instruction Fetch:} 2 ns (total: 2 ns)
\item \textbf{Register Read:} 1 ns (total: 3 ns)
\item Read RS and RT
\item \textbf{ALU Operation:} 2 ns (total: 5 ns)
\item Perform arithmetic/logic operation
\item \textbf{Register Write Setup:} ~0 ns (total: 5 ns)
\item ALU result ready at register write data input

\textbf{Clock Edge: Register Write}

\begin{itemize}
\item Result written to RD

\textbf{Minimum Time Required:} 5 nanoseconds

\textbf{Efficiency Loss:}

\begin{itemize}
\item Could run at 5 ns clock period
\item Forced to wait 7 ns (Load Word limitation)
\item Wastes 2 ns = 28.6% time wasted per R-type instruction

\subsubsection{Branch Instruction Timing}

\textbf{Step-by-Step Delay:}

\begin{enumerate}
\item \textbf{Instruction Fetch:} 2 ns (total: 2 ns)
\item \textbf{Register Read:} 1 ns (total: 3 ns)
\item Read RS and RT for comparison
\item \textbf{ALU Comparison:} 2 ns (total: 5 ns)
\item Subtract RS - RT
\item Generate Zero flag
\item \textbf{Branch Target Calculation:} 2 ns (parallel with ALU)
\item Sign extend offset: ~0 ns
\item Shift left 2: ~0 ns (wire routing)
\item Add to PC+4: 2 ns
\item Can happen in parallel with ALU operation!
\item \textbf{PC Update Setup:} ~0 ns (total: 5 ns)
\item Zero flag + Branch $\rightarrow$ PCSrc
\item Multiplexer selects next PC
\item Ready for clock edge

\textbf{Minimum Time Required:} 5 nanoseconds

\textbf{Key Insight:}

\begin{itemize}
\item Branch target calculation parallel to ALU
\item PC+4 already available from fetch stage
\item No memory access needed
\item Fast like R-type

\subsubsection{Jump Instruction Timing}

\textbf{Step-by-Step Delay:}

\begin{enumerate}
\item \textbf{Instruction Fetch:} 2 ns (total: 2 ns)
\item Also calculates PC+4 in parallel
\item \textbf{Jump Target Calculation:} ~0 ns
\item Extract 26-bit target
\item Shift left 2: Wire routing, ~0 ns
\item Concatenate with PC+4[31:28]: Wire connection, ~0 ns
\item No ALU, no memory, no registers!
\item \textbf{PC Update Setup:} ~0 ns (total: 2 ns)

\textbf{Minimum Time Required:} 2 nanoseconds

\textbf{Fastest Instruction:}

\begin{itemize}
\item Only instruction fetch needed
\item Jump target calculation: Wire operations only
\item No sequential dependencies
\item Wastes 5 ns waiting for clock period!

\subsubsection{Timing Summary Table}

\begin{table}[h]
\centering
\begin{tabular}{|l|l|l|l|}
\hline
\textbf{Instruction Type} & \textbf{Time Required} & \textbf{Wasted Time} & \textbf{Efficiency} \\
\hline
Load Word (LW) & 7 ns & 0 ns & 100\% \\
Store Word (SW) & 5 ns & 2 ns & 71.4\% \\
R-type (ADD, etc.) & 5 ns & 2 ns & 71.4\% \\
Branch (BEQ) & 5 ns & 2 ns & 71.4\% \\
Jump (J) & 2 ns & 5 ns & 28.6\% \\
\hline
\end{tabular}
\end{table}

\textbf{Clock Period (Single-Cycle):} 7 ns (determined by LW)  
\textbf{Clock Frequency:} ~143 MHz

\textbf{Performance Impact:}

\begin{itemize}
\item Most instructions waste time
\item Only Load Word fully utilizes clock cycle
\item Tremendous inefficiency

\subsection{Performance Analysis}

\subsubsection{Program Composition Example}

\textbf{Typical MIPS Program Profile:}

\begin{table}[h]
\centering
\begin{tabular}{|l|l|l|l|}
\hline
\textbf{Instruction Type} & \textbf{Percentage} & \textbf{Time if Variable} & \textbf{Time (Fixed 7ns)} \\
\hline
Arithmetic & 48\% & 5 ns & 7 ns \\
Load Word & 22\% & 7 ns & 7 ns \\
Store Word & 11\% & 5 ns & 7 ns \\
Branch & 19\% & 5 ns & 7 ns \\
\hline
\end{tabular}
\end{table}

\subsubsection{Average Time Calculation}

\textbf{Variable Time (Ideal):}

Average = (0.48 $\times$ 5) + (0.22 $\times$ 7) + (0.11 $\times$ 5) + (0.19 $\times$ 5)
        = 2.40 + 1.54 + 0.55 + 0.95
        = 5.44 ns per instruction

\textbf{Single-Cycle (Actual):}

Average = 7 ns per instruction (all instructions)

\textbf{Performance Loss:}

Overhead = 7 - 5.44 = 1.56 ns per instruction
Efficiency = 5.44 / 7 = 77.7\%
Waste = 22.3\% of time

\subsubsection{Critical Path Problem}

\textbf{Critical Path Determination:}

\begin{itemize}
\item Load Word uses most datapath elements
\item Sequential dependencies:
\end{itemize}
\begin{enumerate}
\item Instruction Memory
\item Register File
\item ALU
\item Data Memory
\item (Register Write in next cycle)

\textbf{Design Principle Violation:}

\begin{itemize}
\item \textbf{"Make the common case fast"}
\item Common case: Arithmetic instructions (48%)
\item Slow case (Load Word) determines speed
\item Common case forced to slow down
\item Design is inefficient

\subsubsection{Clock Period Inflexibility}

\textbf{Single-Cycle Constraint:}

\begin{itemize}
\item Clock period MUST be constant
\item Cannot vary by instruction
\item Must accommodate worst case (slowest instruction)
\item All faster instructions penalized

\textbf{Implications:}

\begin{itemize}
\item Arithmetic: Could run at 143 MHz, forced to 143 MHz ✓
\item Load: Needs 143 MHz, gets 143 MHz ✓
\item Jump: Could run at 500 MHz, forced to 143 MHz ✗

\textbf{Efficiency by Instruction:}

\begin{table}[h]
\centering
\begin{tabular}{|l|l|l|}
\hline
\textbf{Instruction} & \textbf{Efficiency} & \textbf{Waste} \\
\hline
Jump & 28.6\% & 71.4\% \\
Arithmetic & 71.4\% & 28.6\% \\
Store & 71.4\% & 28.6\% \\
Branch & 71.4\% & 28.6\% \\
Load & 100.0\% & 0\% \\
\hline
\end{tabular}
\end{table}

\subsection{Path to Better Performance: Multi-Cycle Design}

\subsubsection{Multi-Cycle Concept}

\textbf{Basic Idea:}

\begin{itemize}
\item Break instruction execution into multiple stages
\item Each stage completes in one (shorter) clock cycle
\item Different instructions use different number of cycles
\item Only use stages actually needed

\textbf{Advantages:}

\begin{itemize}
\item Shorter clock period (faster clock)
\item Instructions take only time they need
\item Better average performance
\item More efficient resource utilization

\subsubsection{Stage Division}

\textbf{Typical Stages:}

\textbf{Stage 1: Instruction Fetch (IF)}

\begin{itemize}
\item Read from instruction memory
\item Update PC to PC+4
\item Store instruction in register

\textbf{Stage 2: Instruction Decode (ID)}

\begin{itemize}
\item Decode opcode
\item Read registers
\item Generate control signals
\item Sign extend immediate

\textbf{Stage 3: Execute (EX)}

\begin{itemize}
\item ALU operation
\item Or address calculation
\item Or branch comparison

\textbf{Stage 4: Memory Access (MEM)}

\begin{itemize}
\item Read from data memory (if load)
\item Write to data memory (if store)
\item Or skip this stage

\textbf{Stage 5: Write-Back (WB)}

\begin{itemize}
\item Write result to register file
\item Or skip if no write needed

\textbf{Not All Instructions Use All Stages:}

\begin{itemize}
\item \textbf{R-type}: IF, ID, EX, WB (skip MEM) = 4 cycles
\item \textbf{Load}: IF, ID, EX, MEM, WB (all stages) = 5 cycles
\item \textbf{Store}: IF, ID, EX, MEM (skip WB) = 4 cycles
\item \textbf{Branch}: IF, ID, EX (skip MEM, WB) = 3 cycles
\item \textbf{Jump}: IF, ID (skip EX, MEM, WB) = 2 cycles

\subsubsection{Clock Period in Multi-Cycle}

\textbf{Determining Clock Period:}

\begin{itemize}
\item Clock period = Longest stage delay
\item NOT longest instruction delay
\item Much shorter than single-cycle

\textbf{Example Stage Delays:}

\begin{table}[h]
\centering
\begin{tabular}{|l|l|}
\hline
\textbf{Stage} & \textbf{Delay} \\
\hline
IF (Instr Memory) & 2 ns \\
ID (Register Read) & 1 ns \\
EX (ALU) & 2 ns \\
MEM (Data Memory) & 2 ns \\
WB (Register Write) & 1 ns \\
\hline
\end{tabular}
\end{table}

\textbf{Longest Stage:} 2 ns  
\textbf{Clock Period:} 2 ns (vs 7 ns single-cycle)  
\textbf{Clock Frequency:} 500 MHz (vs 143 MHz single-cycle)

\subsubsection{Performance Comparison}

\textbf{Single-Cycle:}

All instructions: 1 cycle $\times$ 7 ns = 7 ns

\textbf{Multi-Cycle (with 2 ns clock):}

\begin{table}[h]
\centering
\begin{tabular}{|l|l|l|}
\hline
\textbf{Instruction} & \textbf{Cycles} & \textbf{Time} \\
\hline
Arithmetic & 4 & 8 ns \\
Load & 5 & 10 ns \\
Store & 4 & 8 ns \\
Branch & 3 & 6 ns \\
Jump & 2 & 4 ns \\
\hline
\end{tabular}
\end{table}

\textbf{Weighted Average (same program profile):}

Average = (0.48 $\times$ 8) + (0.22 $\times$ 10) + (0.11 $\times$ 8) + (0.19 $\times$ 6)
        = 3.84 + 2.20 + 0.88 + 1.14
        = 8.06 ns per instruction

\textbf{Wait, That's Worse!}

\begin{itemize}
\item Multi-cycle: 8.06 ns average
\item Single-cycle: 7 ns always
\item Multi-cycle slower?!

\textbf{Resolution:}

\begin{itemize}
\item Example delays assumed equal stage times
\item In reality, stages have different delays
\item Need to balance stage delays
\item Goal: Make all stages approximately equal
\item Then multi-cycle becomes efficient

\textbf{Ideal Multi-Cycle (balanced 1.4 ns stages):}

\begin{table}[h]
\centering
\begin{tabular}{|l|l|l|}
\hline
\textbf{Instruction} & \textbf{Cycles} & \textbf{Time} \\
\hline
Arithmetic & 4 & 5.6 ns \\
Load & 5 & 7.0 ns \\
Store & 4 & 5.6 ns \\
Branch & 3 & 4.2 ns \\
\hline
\end{tabular}
\end{table}

Average = (0.48 $\times$ 5.6) + (0.22 $\times$ 7.0) + (0.11 $\times$ 5.6) + (0.19 $\times$ 4.2)
        = 2.69 + 1.54 + 0.62 + 0.80
        = 5.65 ns per instruction

Speedup = 7 / 5.65 = 1.24$\times$ faster

\subsubsection{Design Challenge}

\textbf{Stage Balancing:}

\begin{itemize}
\item Goal: Roughly equal delay per stage
\item Challenge: Memory slower than ALU
\item Memory stage limits clock period
\item Need techniques:
\item Faster memory
\item Cache memory (next topic)
\item Pipeline (next lecture)

\textbf{Resource Reuse:}

\begin{itemize}
\item Single ALU used across multiple cycles
\item Single memory port can be reused
\item Fewer hardware resources needed
\item More control complexity (FSM needed)

\subsection{Preview: Pipelining}

\subsubsection{Next Step Beyond Multi-Cycle}

\textbf{Pipelining Concept:}

\begin{itemize}
\item Multiple instructions in flight simultaneously
\item Each instruction at different stage
\item Like assembly line
\item \textbf{Stage 1:} Fetch instruction A
\item \textbf{Stage 2:} Decode A, Fetch B
\item \textbf{Stage 3:} Execute A, Decode B, Fetch C
\item \textbf{Stage 4:} Memory A, Execute B, Decode C, Fetch D
\item \textbf{Stage 5:} Write A, Memory B, Execute C, Decode D, Fetch E

\textbf{Benefits:}

\begin{itemize}
\item One instruction completes per cycle (like single-cycle)
\item But clock period short (like multi-cycle)
\item Best of both worlds
\item Dramatic performance improvement

\textbf{Challenges (Covered Next Lecture):}

\begin{itemize}
\item Hazards: Data dependencies between instructions
\item Control hazards: Branches affect pipeline
\item Structural hazards: Resource conflicts
\item Need forwarding and stall logic
\item More complex control

\subsubsection{Coming Next}

\textbf{Topics:}

\begin{itemize}
\item Pipelined datapath design
\item Hazard detection and resolution
\item Forwarding (bypassing)
\item Branch prediction
\item Performance analysis
\item MIPS pipeline implementation

\subsection{Key Takeaways}

\begin{enumerate}
\item \textbf{Single-cycle design executes each instruction in one clock cycle}, with clock period determined by the slowest instruction (Load Word at 7 ns).

\begin{enumerate}
\item \textbf{Control unit generates signals based on opcode}, orchestrating datapath operations for R-type, Load, Store, Branch, and Jump instructions.

\begin{enumerate}
\item \textbf{Load Word is the critical path} (Instruction Fetch $\rightarrow$ Register Read $\rightarrow$ ALU $\rightarrow$ Memory Read $\rightarrow$ Register Write), determining minimum clock period.

\begin{enumerate}
\item \textbf{Jump instruction uses PC[31:28]} concatenated with shifted immediate to form 32-bit target address, enabling 256 MB jump range.

\begin{enumerate}
\item \textbf{Control signals must prevent data corruption}, with RegWrite=0 for Store and Branch to avoid unintended register modifications.

\begin{enumerate}
\item \textbf{"Don't care" values (X) simplify control logic}, allowing optimization when signals don't affect instruction outcome.

\begin{enumerate}
\item \textbf{Hardware operates concurrently}, not sequentially—multiple operations happen simultaneously within each clock cycle.

\begin{enumerate}
\item \textbf{Performance inefficiency drives design evolution}, as most instructions finish early but must wait for full clock period.

\begin{enumerate}
\item \textbf{Resource utilization varies dramatically}, with arithmetic instructions using ~43% of clock period while Load uses 100%.

10. \textbf{Timing analysis reveals optimization opportunities}, showing that memory access dominates critical path (4 ns of 7 ns total).

11. \textbf{Write operations occur at clock edge}, ensuring data stability and preventing race conditions in sequential logic.

12. \textbf{Branch target calculation happens in parallel} with ALU comparison, optimizing branch instruction timing.

13. \textbf{Sign extension is effectively instantaneous} (combinational logic), adding negligible delay to critical path.

14. \textbf{Clock period sets maximum frequency} (~143 MHz for 7 ns period), directly impacting overall processor performance.

15. \textbf{Common case (arithmetic) runs slowly}, violating fundamental design principle of making common case fast.

16. \textbf{Stage division concept emerges from timing analysis}, suggesting multi-cycle implementation could improve efficiency.

17. \textbf{Control signal truth tables systematically define behavior}, mapping each instruction to specific control patterns.

18. \textbf{PC update mechanisms vary by instruction type}, using PC+4, branch target, or jump target based on control signals.

19. \textbf{Data memory access only for Load/Store}, with MemRead and MemWrite controlling when memory participates in execution.

20. \textbf{Performance analysis quantifies inefficiency}, providing concrete motivation for pipelined processor designs in subsequent lectures.

\subsection{Summary}

The single-cycle MIPS processor represents a complete, functioning implementation where each instruction executes in exactly one clock cycle. While conceptually straightforward and easy to understand, the design reveals fundamental performance limitations that drive modern processor architecture evolution. The critical path analysis shows Load Word requiring 7 nanoseconds while simpler instructions like arithmetic operations complete in just 3 nanoseconds, forcing all instructions to wait for the slowest operation. This inefficiency—with most instructions utilizing less than half the available clock period—violates the crucial design principle of "making the common case fast." The systematic control signal analysis demonstrates how the control unit orchestrates datapath operations for different instruction types (R-type, Load, Store, Branch, Jump), with careful attention to preventing data corruption through proper RegWrite and MemWrite signals. The jump instruction introduces pseudo-direct addressing, concatenating PC upper bits with shifted immediate for 256 MB addressability. While the single-cycle design provides essential conceptual foundation for understanding processor operation, the detailed timing analysis and resource utilization metrics clearly motivate the need for more sophisticated approaches—multi-cycle processors that divide execution into variable-length stages, and pipelined processors that overlap instruction execution for dramatically improved throughput. These performance limitations aren't flaws but rather inevitable consequences of the single-cycle constraint, establishing why modern processors universally adopt pipelining despite the additional complexity it introduces.
% % Lecture 12: Pipelined Processors
% CO224 - Computer Architecture

\chapter{Pipelined Processors}

\section{Introduction}

% TODO: Add content from markdown

\section{1. Recap: Single-Cycle Performance Limitations}

% TODO: Add content from markdown

\section{2. Pipelining Concept: The Laundry Shop Analogy}

% TODO: Add content from markdown

\section{3. MIPS Five-Stage Pipeline}

% TODO: Add content from markdown

\section{4. MIPS ISA Design for Pipelining}

% TODO: Add content from markdown

\section{5. Instruction-Level Parallelism (ILP)}

% TODO: Add content from markdown

\section{6. Pipeline Hazards: Structural Hazards}

% TODO: Add content from markdown

\section{7. Data Hazards}

% TODO: Add content from markdown

\section{8. Control Hazards}

% TODO: Add content from markdown

\section{9. Summary and Key Concepts}

% TODO: Add content from markdown

\section{10. Important Formulas and Metrics}

% TODO: Add content from markdown

\section{Key Takeaways}

% TODO: Add content from markdown

\section{Summary}

% TODO: Add content from markdown

% Note: Convert markdown content manually for best results
% Remember to:
% - Replace markdown images with \includegraphics
% - Convert code blocks to \begin{lstlisting}
% - Convert lists to \begin{itemize} or \begin{enumerate}
% - Convert bold/italic with \textbf{} and \textit{}

% % Lecture 13: Pipeline Operation and Timing
% CO224 - Computer Architecture

\chapter{Pipeline Operation and Timing}

\section{Introduction}

% TODO: Add content from markdown

\section{1. Lecture Introduction and Recap}

% TODO: Add content from markdown

\section{2. Five-Stage MIPS Pipeline Review}

% TODO: Add content from markdown

\section{3. Pipeline Registers: Necessity and Function}

% TODO: Add content from markdown

\section{4. Load Word Instruction: Detailed Cycle-by-Cycle Analysis}

% TODO: Add content from markdown

\section{5. Store Word Instruction: Key Differences}

% TODO: Add content from markdown

\section{6. Common Pipeline Diagram Errors}

% TODO: Add content from markdown

\section{7. Multi-Clock-Cycle Pipeline Diagrams}

% TODO: Add content from markdown

\section{8. Timing and Clock Frequency Analysis}

% TODO: Add content from markdown

\section{9. Practical Exercises and Solutions}

% TODO: Add content from markdown

\section{10. Summary and Key Takeaways}

% TODO: Add content from markdown

\section{11. Important Formulas}

% TODO: Add content from markdown

\section{Key Takeaways}

% TODO: Add content from markdown

\section{Summary}

% TODO: Add content from markdown

% Note: Convert markdown content manually for best results
% Remember to:
% - Replace markdown images with \includegraphics
% - Convert code blocks to \begin{lstlisting}
% - Convert lists to \begin{itemize} or \begin{enumerate}
% - Convert bold/italic with \textbf{} and \textit{}


% Chapter 4: Memory Hierarchy
% \chapter{Memory Hierarchy}

% \section{Lecture 14: Introduction to Memory Systems and Cache Memory}

\emph{By Dr. Isuru Nawinne}

\subsection{Introduction}

This lecture marks a crucial transition from CPU-centric topics to memory systems, introducing cache memory as the elegant solution to the fundamental processor-memory speed gap. We begin with historical context, tracing how stored-program concept revolutionized computing, then explore the memory hierarchy that creates the illusion of large, fast memory through careful exploitation of temporal and spatial locality. The direct-mapped cache organization receives detailed treatment, establishing foundational concepts of blocks, tags, indices, and valid bits that underpin all cache designs. Understanding cache memory proves as essential as understanding processor architecture, as memory system performance often determines overall computer system speed in practice.

\subsection{Lecture Introduction and Historical Context}

\subsubsection{Lecture Transition}

\textbf{Previous Topics:}

\begin{itemize}
\item CPU datapath and control (ARM, MIPS, pipelining)

\textbf{New Focus:}

\begin{itemize}
\item Memory systems (equally important as CPU)

\textbf{Motivation:}

\begin{itemize}
\item Memory plays as significant a role as CPU in modern computer architecture

\subsubsection{Historical Background}

\paragraph{Early Computing Machines (1940s)}

\textbf{Examples:}

\begin{itemize}
\item ENIAC (University of Pennsylvania)
\item Harvard Mark I (ASTC)

\textbf{Characteristics:}

\begin{itemize}
\item Filled entire rooms
\item Built using vacuum tubes and electrical circuitry
\item Developed for war efforts (World War II)
\item Used for artillery planning, nuclear weapon calculations
\item No concept of software or memory as we know today

\textbf{Programming Method:}

\begin{itemize}
\item Rewiring the entire machine for each algorithm
\item Engineers spent days/weeks reconfiguring machines
\item No stored program concept

\subsubsection{Key Historical Figures}

\paragraph{Alan Turing (1936)}

\begin{itemize}
\item British mathematician, brilliant mind
\item First conceived the stored program computer concept
\item Designed the Universal Turing Machine (hypothetical machine)
\item First notion of memory, programs stored in computers, data read/write operations
\item Later involved in World War II cryptography (Enigma Machine, "The Imitation Game")

\paragraph{John von Neumann (1940s)}

\begin{itemize}
\item Hungarian mathematician, regarded as "last of the brilliant mathematicians"
\item Prodigy: Solving calculus problems by age 8
\item Contributed across many fields
\item Got involved with EDVAC computer project
\item Implemented stored program concept based on Turing's ideas

\subsubsection{First Stored Program Computers}

\paragraph{EDVAC (1948)}

\begin{itemize}
\item Commissioned by U.S. Army
\item John von Neumann involved as consultant
\item Memory: Initially 1044 words, upgraded to 1024 words (power of 2)
\item First machine with stored program concept
\item Memory stored program electrically (not in wiring)
\item Engineers created the first "memory" device
\item First test programs: Nuclear weapon detonation calculations, hydrogen bomb calculations

\paragraph{Von Neumann Architecture}

\textbf{Key Concept:}

\begin{itemize}
\item Data AND instructions both in SAME memory
\item Access data and programs through SAME connection pathways
\item Unified memory for instructions and data
\item This concept became foundation of modern computers

\paragraph{EDSAC (Cambridge University)}

\begin{itemize}
\item Built about a year after EDVAC
\item First machine fully implementing Von Neumann architecture
\item Memory: 512 words of 18 bits each
\item Also built for war effort

\paragraph{Harvard Architecture (Contrasted)}

\begin{itemize}
\item Separate storages for instructions and data
\item Separate connections to instruction memory and data memory
\item Used in MIPS datapath design (separate instruction memory and data memory)

\textbf{Modern Computers:}

\begin{itemize}
\item Use a MIX of both Von Neumann and Harvard architectures
\item Features from both types incorporated

\subsection{Memory Technologies: Types and Characteristics}

\subsubsection{Commonly Used Memory Technologies Today}

\begin{itemize}
\item SRAM (Static RAM)
\item DRAM (Dynamic RAM)
\item Flash Memory
\item Magnetic Disk
\item Magnetic Tape

\subsubsection{SRAM (Static RAM)}

| Property            | Value/Description                          |
| ------------------- | ------------------------------------------ |
| \textbf{Technology}      | Built using flip-flops                     |
| \textbf{Volatility}      | Volatile (loses content when power lost)   |
| \textbf{Access Time}     | Less than 1 nanosecond (< 1 ns)            |
| \textbf{Clock Frequency} | More than 1 GHz                            |
| \textbf{Cycle Time}      | Less than 1 nanosecond (< 1 ns)            |
| \textbf{Capacity}        | Kilobytes to Megabytes range               |
| \textbf{Cost}            | ~$2000 per gigabyte (VERY EXPENSIVE)       |
| \textbf{Speed}           | Extremely fast                             |
| \textbf{Usage}           | Cache memories (small amounts due to cost) |

\textbf{Note on Cycle Time:}

\begin{itemize}
\item Cycle time = minimum time between two consecutive memory accesses
\item Access time ≈ Cycle time for SRAM

\subsubsection{DRAM (Dynamic RAM)}

| Property        | Value/Description                              |
| --------------- | ---------------------------------------------- |
| \textbf{Technology}  | Transistors + Capacitors                       |
| \textbf{Volatility}  | Volatile (requires power AND periodic refresh) |
| \textbf{Access Time} | ~25 nanoseconds (50 ns in some contexts)       |
| \textbf{Cycle Time}  | ~50 nanoseconds (double the access time)       |
| \textbf{Capacity}    | Gigabytes (8 GB, 16 GB, or more)               |
| \textbf{Cost}        | ~$10 per gigabyte                              |
| \textbf{Usage}       | Main memory in computers                       |

\textbf{Key Characteristics:}

\begin{itemize}
\item Capacitor charge must be maintained
\item "Destructive read": Reading loses the charge, requires rewrite/refresh
\item Longer cycle time due to refresh requirement
\item After reading, must rewrite data to same cell
\item Significantly slower than SRAM (25-50 ns vs < 1 ns)

\subsubsection{Flash Memory}

| Property        | Value/Description                                     |
| --------------- | ----------------------------------------------------- |
| \textbf{Technology}  | NAND MOSFET (NAND gate with two gates)                |
| \textbf{Volatility}  | Non-volatile (retains data without power)             |
| \textbf{Access Time} | ~70 nanoseconds                                       |
| \textbf{Cycle Time}  | ~70 nanoseconds                                       |
| \textbf{Capacity}    | Gigabytes range                                       |
| \textbf{Cost}        | Less than $1 per gigabyte                             |
| \textbf{Usage}       | Secondary storage (SSDs - Solid State Devices/Drives) |

\textbf{Limitation:}

\begin{itemize}
\item Limited read/write cycles
\item After several thousand cycles, memory cells may degrade
\item Integrity decreases, capacity effectively decreases
\item Slightly slower than DRAM, but non-volatile

\subsubsection{Magnetic Disk}

| Property        | Value/Description                                          |
| --------------- | ---------------------------------------------------------- |
| \textbf{Technology}  | Magnetic (mechanical device)                               |
| \textbf{Access Time} | 5 to 10 milliseconds (MUCH slower than electronic memory!) |
| \textbf{Cycle Time}  | Similar to access time (~5-10 ms)                          |
| \textbf{Capacity}    | Several terabytes                                          |
| \textbf{Cost}        | Fraction of a dollar per gigabyte (very cheap)             |

\textbf{Usage:}

\begin{itemize}
\item Previously: Main secondary storage
\item Currently: Being replaced by flash/SSDs for secondary storage
\item Now used primarily for tertiary storage, backups
\item Good for long-term data retention, low cost
\item Slowness acceptable for infrequent backup operations

\textbf{Note:} Average numbers; varies by data location on disk. Mechanical: spinning platters, moving read/write heads.

\subsection{The Memory Performance Problem}

\subsubsection{The CPU-Memory Speed Gap}

\textbf{CPU Clock Cycle:}

\begin{itemize}
\item Modern CPUs: > 1 GHz clock frequency
\item Clock cycle: < 1 nanosecond (1 ns corresponds to 1 GHz)

\textbf{Main Memory (DRAM):}

\begin{itemize}
\item Cycle time: ~50 nanoseconds
\item Time between starts of two consecutive memory accesses: 50 ns

\subsubsection{The Problem}

\textbf{Speed Discrepancy:}

\begin{itemize}
\item CPU cycle: < 1 ns
\item Memory cycle: 50 ns
\item \textbf{Memory is 50$\times$ SLOWER than CPU!}

\subsubsection{Impact on Pipelining}

\textbf{The Challenge:}

\begin{itemize}
\item In MIPS pipeline, MEM stage must finish in ONE clock cycle
\item Every pipeline stage must take same time
\item How can MEM stage complete in 1 ns when memory takes 50 ns?
\item Pipeline performance would be severely degraded

\textbf{The Contradiction:}

\begin{itemize}
\item CPU expects 1 ns memory access
\item Actual DRAM takes 50 ns
\item "Something is not right" - how can this work?

\subsection{Memory Hierarchy Concept}

\subsubsection{The Solution: Memory Hierarchy}

\textbf{Core Idea:}

\begin{itemize}
\item Trick the CPU into thinking memory is BOTH fast AND large
\item Desired characteristics:
\item Fast access times (like SRAM: < 1 ns)
\item Large capacity (like Disk: terabytes)
\item These characteristics don't exist in single technology
\item Solution: Implement memory as a HIERARCHY

\subsubsection{Memory Hierarchy Structure}

Level 1 (Top): SRAM (Cache)
\begin{itemize}
\item Smallest capacity
\item Fastest speed
\item Closest to CPU physically

Level 2: DRAM (Main Memory)
\begin{itemize}
\item Medium capacity
\item Medium speed

Level 3 (Bottom): Disk
\begin{itemize}
\item Largest capacity
\item Slowest speed

\subsubsection{Key Principles}

\paragraph{1. CPU Access Restriction}

\begin{itemize}
\item CPU can ONLY access top level (SRAM cache)
\item CPU thinks cache is the actual memory
\item CPU cannot directly access DRAM or Disk

\paragraph{2. CPU's Perception}

\begin{itemize}
\item Experiences the SPEED of SRAM
\item Feels the CAPACITY of DRAM and Disk combined
\item Illusion: Memory is as fast as SRAM AND as big as lowest level

\paragraph{3. Data Organization}

\begin{itemize}
\item Upper levels contain SUBSET of data from lower levels
\item SRAM (few MB) contains subset of DRAM (several GB)
\item DRAM contains subset of Disk (several TB)
\item At any given time, each level holds only a fraction of lower level's data

\paragraph{4. Hierarchy Characteristics}

\begin{itemize}
\item Devices up the hierarchy: Smaller and faster
\item Devices down the hierarchy: Larger but slower

\subsubsection{The Challenge}

\textbf{What if CPU asks for data NOT in the cache (top level)?}

\begin{itemize}
\item Need mechanism to copy data from lower levels
\item This leads to the concepts of hits, misses, and cache management

\subsection{Analogy: Music Library}

\subsubsection{Understanding Memory Hierarchy Through Music}

\paragraph{Three-Level Music System}

\textbf{1. Mobile Phone (analogous to SRAM/Cache):}

\begin{itemize}
\item Carries a subset of your favorite songs
\item Always with you
\item Listen to music directly from phone
\item Limited storage (like cache has limited capacity)

\textbf{2. Computer Hard Disk (analogous to DRAM/Main Memory):}

\begin{itemize}
\item Main music collection stored here
\item Larger collection than phone
\item Not always accessible (not in pocket)
\item Copy songs from here to phone when needed

\textbf{3. Internet (analogous to Disk/Mass Storage):}

\begin{itemize}
\item All songs available (massive storage)
\item Download/buy songs from here
\item Copy to computer, then to phone

\subsubsection{Usage Scenarios}

\paragraph{Scenario 1 (Hit)}

\begin{itemize}
\item Want to listen to a song
\item Song is already on phone
\item Just play it directly
\item Similar to cache hit: Data already in cache

\paragraph{Scenario 2 (Miss to Level 2)}

\begin{itemize}
\item Want to listen to a song
\item Song NOT on phone
\item Must go to computer and copy to phone
\item Then listen on phone
\item Similar to cache miss: Must fetch from main memory

\paragraph{Scenario 3 (Miss to Level 3)}

\begin{itemize}
\item Want to listen to a song
\item Song NOT on phone AND NOT on computer
\item Download from internet to computer
\item Copy to phone
\item Then listen
\item Similar to cache miss to disk: Must fetch from lowest level

\subsubsection{Key Parallels}

\begin{itemize}
\item Always listen from phone (CPU always accesses cache)
\item Main collection in computer (main memory holds primary data)
\item All data available on internet (disk holds everything)
\item Copy operations when data not available at higher levels

\subsection{Memory Hierarchy Terminology}

\subsubsection{Essential Terms for Memory Access}

\paragraph{HIT}

\textbf{Definition:} Requested data IS available at the accessed level

\begin{itemize}
\item CPU requests data $\rightarrow$ Data found in cache
\item Like wanting to listen to song already on your phone
\item Can be served immediately from that level

\paragraph{MISS}

\textbf{Definition:} Requested data is NOT available at the accessed level

\begin{itemize}
\item CPU requests data $\rightarrow$ Data NOT found in cache
\item Like wanting to listen to song not on your phone
\item Must fetch from lower level in hierarchy

\paragraph{HIT RATE}

\textbf{Definition:} Ratio/percentage of accesses that result in hits

\textbf{Formula:}

Hit Rate = (Number of Hits) / (Total Accesses)

\textbf{Example:} 100 accesses, 90 hits $\rightarrow$ Hit Rate = 90% or 0.9

Indicates how often data is found at the accessed level. Higher hit rate = better performance.

\paragraph{MISS RATE}

\textbf{Definition:} Ratio/percentage of accesses that result in misses

\textbf{Formula:}

Miss Rate = (Number of Misses) / (Total Accesses)
Miss Rate = 1 - Hit Rate

\textbf{Example:} 100 accesses, 10 misses $\rightarrow$ Miss Rate = 10% or 0.1

Lower miss rate = better performance.

\paragraph{HIT LATENCY}

\textbf{Definition:} Time taken to determine if access is a hit AND serve the data

\begin{itemize}
\item Time to check if data is in cache and deliver it to CPU
\item For SRAM cache: < 1 nanosecond

\textbf{Components:}

\begin{itemize}
\item Time to search cache
\item Time to verify data presence
\item Time to extract and send data to CPU

\paragraph{MISS PENALTY}

\textbf{Definition:} EXTRA time required when access is a miss

\textbf{Process:}

\begin{enumerate}
\item Determine it's a miss (hit latency spent)
\item Go to next level (DRAM)
\item Find the data
\item Copy to cache
\item Put in appropriate place
\item Deliver to CPU

\textbf{Key Points:}

\begin{itemize}
\item Total time on miss = Hit Latency + Miss Penalty
\item Miss penalty for DRAM access can be 100$\times$ hit latency
\item Very expensive in terms of time!

\subsection{Performance Impact and Requirements}

\subsubsection{Average Memory Access Time}

\textbf{Formula:}

Average Access Time = Hit Latency + (Miss Rate $\times$ Miss Penalty)

\textbf{Explanation:}

\begin{itemize}
\item ALL accesses consume hit latency (must check cache)
\item Only misses consume additional miss penalty
\item Miss Rate determines portion of accesses incurring penalty

\subsubsection{Example Analysis}

\textbf{Given:}

\begin{itemize}
\item Hit Latency (SRAM): < 1 nanosecond
\item Miss Penalty (DRAM access): ~100 nanoseconds (100$\times$ slower)
\item CPU clock cycle: < 1 nanosecond

\paragraph{For Pipeline to Work}

\begin{itemize}
\item MEM stage must complete in 1 clock cycle
\item Memory access must complete in < 1 ns most of the time

\paragraph{Required Hit Rate Calculation}

\textbf{If Hit Rate = 99.9\% (Miss Rate = 0.1\%):}

Average Time = 1 ns + (0.001 $\times$ 100 ns)
             = 1 ns + 0.1 ns
             = 1.1 ns

Still close to 1 clock cycle!

\textbf{If Hit Rate = 90\% (Miss Rate = 10\%):}

Average Time = 1 ns + (0.10 $\times$ 100 ns)
             = 1 ns + 10 ns
             = 11 ns

Unacceptable! 11$\times$ slower than CPU clock!

\subsubsection{Critical Requirement}

\begin{itemize}
\item Need VERY HIGH hit rate at cache level
\item Not just high, but VERY, VERY high
\item Target: \textbf{99.9\% or better}
\item Only 0.1% of accesses should go to memory

\subsubsection{Performance Implications}

\paragraph{With 99.9\% Hit Rate}

\begin{itemize}
\item 99.9% of time: CPU works fine, memory appears fast
\item 0.1% of time: CPU must STALL, wait for data from DRAM
\item Stall is unavoidable for misses
\item Overall: CPU maintains illusion of fast, large memory

\paragraph{With Lower Hit Rate}

\begin{itemize}
\item More frequent stalls
\item Pipeline performance degrades significantly
\item Average memory access time increases
\item CPU slows down dramatically

\textbf{Conclusion:}

\begin{itemize}
\item Must ensure VERY high hit rate at SRAM level
\item Memory hierarchy only works if locality principles hold
\item Like having most songs you want to listen to already on phone
\item Don't want to copy from computer frequently (time-consuming)

\subsection{Principles of Locality}

\subsubsection{Foundation for Memory Hierarchy Success}

\textbf{Nature of Computer Programs:}

\begin{itemize}
\item Programs access only SMALL portion of entire address space at any given time
\item Address space: Entire memory range (address 0 to maximum address)
\item At any time window, program uses only small fraction of total data
\item True by nature of how programs are written, compiled, and executed
\item True for instruction sets like ARM, MIPS

\subsubsection{Temporal Locality (Locality in Time)}

\paragraph{Definition}

\textbf{"Recently accessed data are likely to be accessed again soon"}

\textbf{Explanation:}

\begin{itemize}
\item If you access memory address A at time T
\item High probability of accessing address A again at time T+ΔT (soon after)
\item Same data accessed multiple times in short time window
\item "Locality in time" - data clustered temporally

\paragraph{Common Examples in Programs}

\textbf{a) Loop Index Variables:}

\begin{lstlisting}[language=c]
for (int i = 0; i < 100; i++) {
    // i is accessed every iteration
    // Same memory location for 'i' accessed repeatedly
}
\end{verbatim}

\textbf{b) Loop-Invariant Data:}

\begin{lstlisting}[language=c]
for (int i = 0; i < n; i++) {
    result = result + array[i] * constant;
    // 'result' and 'constant' accessed every iteration
}
\end{verbatim}

\textbf{c) Function/Procedure Calls:}

\begin{itemize}
\item Local variables accessed multiple times during function execution
\item Same stack frame locations accessed repeatedly

\textbf{d) Instructions:}

\begin{itemize}
\item Loop body instructions executed many times
\item Same instruction addresses accessed repeatedly

\paragraph{Music Analogy}

\begin{itemize}
\item If you listen to a song, you're likely to listen to it again soon
\item Sometimes listen to same song 10 times in a row
\item Want to replay favorite songs

\paragraph{Degree of Temporal Locality}

\begin{itemize}
\item Varies from program to program
\item But present in nearly ALL programs
\item Stronger in some (tight loops) than others

\subsubsection{Spatial Locality (Locality in Space)}

\paragraph{Definition}

\textbf{"Data located close to recently accessed data are likely to be accessed soon"}

\textbf{Explanation:}

\begin{itemize}
\item If you access memory address A at time T
\item High probability of accessing addresses A+1, A+2, A+3, ... soon after
\item Sequential or nearby addresses accessed together
\item "Locality in space" - data clustered spatially in memory

\paragraph{Common Examples in Programs}

\textbf{a) Array Traversal:}

\begin{lstlisting}[language=c]
for (int i = 0; i < 100; i++) {
    sum += array[i];
    // Access array[0], then array[1], then array[2], ...
    // Sequential memory addresses
}
\end{verbatim}

\textbf{b) Sequential Instruction Execution:}

\begin{itemize}
\item Instructions stored sequentially in memory
\item PC increments: fetch instruction at PC, then PC+4, then PC+8, ...
\item Except for branches, mostly sequential

\textbf{c) Data Structures:}

\begin{lstlisting}[language=c]
struct Student {
    int id;
    char name[50];
    float gpa;
};
Student s;
// Accessing s.id, then s.name, then s.gpa
// Nearby memory locations
\end{verbatim}

\textbf{d) String Processing:}

\begin{lstlisting}[language=c]
char str[] = "Hello";
for (int i = 0; str[i] != '\0'; i++) {
    // Access str[0], str[1], str[2], ...
    // Consecutive bytes in memory
}
\end{verbatim}

\paragraph{Music Analogy}

\begin{itemize}
\item If you listen to song by artist X, likely to listen to another song by artist X
\item If you listen to song from album Y, likely to listen to next song in album Y
\item Related/nearby songs accessed together

\paragraph{Degree of Spatial Locality}

\begin{itemize}
\item Varies by data access patterns
\item Strong in array-based algorithms
\item Present in most structured programs

\subsubsection{Universal Applicability}

\begin{itemize}
\item Both principles hold true for NEARLY ALL programs
\item Degree varies, but principles universally applicable
\item Foundation assumptions for cache design

\subsection{Cache Memory Concept and Block-Based Operation}

\subsubsection{Cache Memory Overview}

\textbf{Purpose:}

\begin{itemize}
\item Memory device at top level of hierarchy
\item Based on two principles of locality
\item Decides what data to keep based on locality principles

\subsubsection{Data Organization: BLOCKS}

\textbf{Key Concepts:}

\begin{itemize}
\item CPU requests individual WORDS from memory
\item Between cache and memory: Handle BLOCKS of data
\item \textbf{Block} = multiple consecutive words
\item Block size example: 8 bytes = 2 words (with 4-byte words)
\item Hidden from CPU (CPU still thinks in words)

\subsubsection{Why Blocks? (Spatial Locality)}

\paragraph{Instead of Words}

\begin{itemize}
\item Fetch single word CPU requested
\item Next access likely nearby address
\item Would require another fetch

\paragraph{Using Blocks}

\begin{itemize}
\item Fetch requested word AND nearby words together
\item Bring entire block (e.g., 8 consecutive bytes)
\item Subsequent accesses likely in same block (spatial locality)
\item Reduces future misses

\paragraph{Music Library Analogy}

\begin{itemize}
\item Want to listen to one song $\rightarrow$ Copy entire album to phone
\item Not just the single song you want right now
\item Because you'll likely want other songs from same album soon
\item Saves future copy operations

\paragraph{Block Benefits}

\begin{itemize}
\item Exploits spatial locality
\item Reduces miss rate
\item Amortizes fetch cost over multiple words
\item More efficient use of memory bandwidth

\subsubsection{Cache Management Decisions}

\paragraph{1. What to Keep in Cache}

\begin{itemize}
\item Based on BOTH locality principles
\item Recently accessed data (temporal locality)
\item Blocks containing nearby data (spatial locality)

\paragraph{2. What to Evict from Cache}

\begin{itemize}
\item Based on TEMPORAL locality
\item When cache full and need space for new block
\item Must throw out existing data

\subsubsection{Eviction Strategy (Ideal)}

\textbf{Least Recently Used (LRU):}

\begin{itemize}
\item Throw out LEAST RECENTLY USED (LRU) data
\item If cache has 10 blocks, need to evict 1
\item Choose the block that was used longest time ago
\item Keep more recently used blocks
\item Temporal locality suggests LRU block least likely to be accessed soon

\textbf{Example:}

\begin{itemize}
\item Cache has blocks A, B, C, D, E
\item Last access times: A(10 cycles ago), B(2 cycles ago), C(50 cycles ago), D(5 cycles ago), E(1 cycle ago)
\item Need to evict one block
\item Evict C (least recently used, 50 cycles ago)
\item Keep E, B, D, A (more recently used)

\subsection{Memory Addressing: Bytes, Words, and Blocks}

\subsubsection{Byte Address}

\textbf{Definition:} Address referring to individual byte in memory

\textbf{Characteristics:}

\begin{itemize}
\item Each byte-sized location has unique address
\item Standard memory addressing
\item Address Space: With 32-bit address, can access 2³² individual bytes

\textbf{Example Address:}

Address: 00000000000000000000000000001010 (binary)
       = 10 (decimal)
Points to: Byte at memory location 10

\textbf{Memory Structure:}

Address 0:  [byte 0]
Address 1:  [byte 1]
Address 2:  [byte 2]
...
Address 10: [byte 10]  $\leftarrow$ This byte addressed by example
...

\subsubsection{Word Address}

\textbf{Definition:} Address referring to a word (multiple bytes) in memory

\textbf{Typical Word Size:} 4 bytes (32 bits)

\paragraph{Word Alignment}

\begin{itemize}
\item Words start at addresses that are multiples of 4
\item Word 0: Addresses 0, 1, 2, 3
\item Word 1: Addresses 4, 5, 6, 7
\item Word 2: Addresses 8, 9, 10, 11
\item Word 3: Addresses 12, 13, 14, 15

\paragraph{Word Address Format (32-bit)}

[30-bit word identifier][2-bit byte offset]
                        └── Always "00" for word-aligned addresses

\textbf{Example:}

Address: ...00001000 (binary)
\begin{itemize}
\item Last 2 bits: 00 $\rightarrow$ Word-aligned
\item Remaining bits: Identify which word
\item This is address 8, start of word 2

\paragraph{Byte Within Word}

Last 2 bits select byte within word:

\begin{itemize}
\item \texttt{00} $\rightarrow$ First byte (address 8)
\item \texttt{01} $\rightarrow$ Second byte (address 9)
\item \texttt{10} $\rightarrow$ Third byte (address 10)
\item \texttt{11} $\rightarrow$ Fourth byte (address 11)

\textbf{Key Points:}

\begin{itemize}
\item Word addresses are multiples of 4
\item Can divide by 4 without remainder
\item Last 2 bits = 00 for word addresses
\item NOT all addresses ending in 00 are word addresses, but word addresses end in 00
\item Only portion of address except last 2 bits identifies the word

\subsubsection{Block Address}

\textbf{Definition:} Address referring to a block (multiple words) in memory

\textbf{Example Block Size:} 8 bytes = 2 words

\paragraph{Block Alignment}

\begin{itemize}
\item Blocks start at addresses that are multiples of 8
\item Block 0: Addresses 0-7
\item Block 1: Addresses 8-15
\item Block 2: Addresses 16-23
\item Block 3: Addresses 24-31

\paragraph{Block Address Format (32-bit)}

[Block Identifier][3-bit offset]
                   └── Last 3 bits for 8-byte blocks

\textbf{Example:}

Address: 00000000000000000000000000101101 (binary)
         = 45 (decimal)

Block Address Portion:
\begin{itemize}
\item Ignore last 3 bits: 00101 (offset part)
\item Block address: 00000000000000000000000000101 (identifies block)
\item This identifies the block containing address 45

\paragraph{Offset Within Block (3 bits for 8-byte blocks)}

\textbf{BYTE OFFSET (all 3 bits):}

\begin{itemize}
\item Used to identify individual BYTE within block
\item \texttt{000} $\rightarrow$ Byte 0
\item \texttt{001} $\rightarrow$ Byte 1
\item ...
\item \texttt{111} $\rightarrow$ Byte 7

\textbf{WORD OFFSET (most significant bit of offset):}

\begin{itemize}
\item Used to identify WORD within block (when block has 2 words)
\item \texttt{0XX} $\rightarrow$ First word (bytes 0-3)
\item \texttt{1XX} $\rightarrow$ Second word (bytes 4-7)
\item Only need 1 bit to select between 2 words

\subsubsection{Address Components Summary}

\textbf{For address with 8-byte blocks, 4-byte words:}

[Block Address][Word Offset][Byte in Word]
     ^              ^              ^
     |              |              └── 2 bits: Select byte within word
     |              └── 1 bit: Select word within block
     └── Remaining bits: Identify which block

\textbf{Example Breakdown:}

Address: ...00101101
\begin{itemize}
\item Last 2 bits (01): Byte offset within word $\rightarrow$ Byte 1 of word
\item 3rd bit from right (1): Word offset $\rightarrow$ Second word of block
\item Remaining bits (...00101): Block address $\rightarrow$ Block 5

\textbf{All Bytes in Same Block:}

\begin{itemize}
\item Share same block address
\item Differ only in offset bits

\textbf{Important Distinctions:}

\begin{itemize}
\item \textbf{Byte address:} Full 32 bits
\item \textbf{Word address:} Term refers to full address of word-aligned location
\item \textbf{Block address:} Term refers to portion of address identifying block (excluding offset)

\subsection{The Cache Addressing Problem}

\subsubsection{Problem Statement}

\paragraph{In Main Memory}

\begin{itemize}
\item Direct addressing: Address 10 $\rightarrow$ Direct access to location 10
\item Like array indexing: array[10] directly accesses index 10
\item Straightforward: Address uniquely identifies memory location
\item No search required: Hardware directly decodes address

\paragraph{In Cache}

\begin{itemize}
\item Cache is MUCH smaller than memory
\item Memory: Gigabytes (millions/billions of addresses)
\item Cache: Kilobytes or Megabytes (thousands/few million bytes)
\item Example: Memory has 1 million addresses, cache has only 8 slots

\subsubsection{The Challenge}

\begin{itemize}
\item CPU generates address from full address space (e.g., address 10)
\item Cache has only 8 slots (indices 0-7)
\item Cannot directly use memory address as cache index
\item Address 10 doesn't directly map to cache location
\item \textbf{How to find data in cache with memory address?}

\subsubsection{Initial Solution Idea: Store Addresses with Data}

\textbf{Approach:}

\begin{itemize}
\item Store memory address alongside data in cache
\item Each cache entry: [Address | Data]
\item When CPU requests address, search cache for matching address

\textbf{Problems with This Approach:}

\textbf{1. Space Overhead:}

\begin{itemize}
\item Must store full address (e.g., 32 bits) with each data block
\item Significant storage overhead
\item Example: 32-bit address + 256-bit data block = ~13% overhead

\textbf{2. Search Time:}

\begin{itemize}
\item Must search through ALL cache entries
\item Sequential or parallel search required
\item Example: 8 cache slots $\rightarrow$ Check all 8 tags
\item Time-consuming, degrades hit latency
\item Cannot directly access cache entry

\subsubsection{Need for Better Solution}

\textbf{Requirements:}

\begin{itemize}
\item Require MAPPING between memory addresses and cache locations
\item Want DIRECT access (no search) if possible
\item Must be efficient in both space and time

\textbf{Requirements for Practical Cache:}

\begin{enumerate}
\item Fast access (< 1 ns hit latency)
\item Minimal storage overhead
\item Direct or near-direct cache indexing
\item Efficient tag comparison (if needed)

\textbf{Solution Preview:} Address Mapping Functions

\begin{itemize}
\item Need function: Memory Address $\rightarrow$ Cache Location
\item Different mapping strategies possible
\item Simplest: Direct Mapping (discussed next)

\subsection{Direct-Mapped Cache}

\subsubsection{Direct Mapping Concept}

\textbf{Definition:}

\begin{itemize}
\item Each memory address maps to EXACTLY ONE cache location
\item One-to-one deterministic mapping
\item No choice in cache placement

\textbf{Mapping Rule:}

Cache Index = Block Address MOD (Number of Blocks in Cache)

\textbf{Formula:}

Cache Index = (Block Address) mod (Cache Size in Blocks)

\textbf{Example:}

\begin{itemize}
\item Cache has 8 blocks $\rightarrow$ Indices 0-7
\item Block address = 13
\item Cache index = 13 mod 8 = 5
\item Block 13 maps ONLY to cache index 5

\subsubsection{Mathematical Properties}

\paragraph{Mod Operation with Powers of 2}

\begin{itemize}
\item Cache sizes typically powers of 2 (1, 2, 4, 8, 16, 32, ...)
\item Mod by power of 2 = take least significant bits
\item Example: N mod 8 = N mod 2³ = last 3 bits of N

\paragraph{Hardware Implementation}

\begin{itemize}
\item No division circuit needed!
\item Simply extract least significant bits
\item Very fast, pure combinational logic

\subsubsection{Direct Mapping Example}

\textbf{Given:}

\begin{itemize}
\item Block size: 8 bytes
\item Cache size: 8 blocks
\item Cache indices: 0, 1, 2, 3, 4, 5, 6, 7

\textbf{Cache Structure (Initial View):}

| Index | Data Block |
| ----- | ---------- |
| 0     | [64 bits]  |
| 1     | [64 bits]  |
| 2     | [64 bits]  |
| 3     | [64 bits]  |
| 4     | [64 bits]  |
| 5     | [64 bits]  |
| 6     | [64 bits]  |
| 7     | [64 bits]  |

\textbf{Example Addresses:}

\textbf{Address 1:}

Binary: ...00000001[011]
         └─ Block address = 0
         └─ Offset = 3 bytes
Cache index = 0 mod 8 = 0
Maps to cache index 0

\textbf{Address 2 (block address in focus):}

Binary: ...00000101[000]
         └─ Block address = 5
         └─ Offset = 0
Cache index = 5 mod 8 = 5
Maps to cache index 5

\subsubsection{Address Structure for Direct-Mapped Cache}

[Tag][Index][Offset]
  ^     ^       ^
  |     |       └── Identifies byte/word within block
  |     └── Identifies cache location (index)
  └── Remaining bits to differentiate blocks mapping to same index

\paragraph{Bit Allocation (for 8-block cache, 8-byte blocks, 32-bit address)}

\begin{itemize}
\item \textbf{Offset:} 3 bits (for 8-byte blocks: 2³ = 8)
\item \textbf{Index:} 3 bits (for 8 cache blocks: 2³ = 8)
\item \textbf{Tag:} 26 bits (remaining: 32 - 3 - 3 = 26)

\paragraph{Index Bits}

\begin{itemize}
\item Least significant bits of block address
\item Directly select cache location
\item Number of bits = log₂(cache blocks)
\item 8 blocks $\rightarrow$ 3 index bits
\item 16 blocks $\rightarrow$ 4 index bits
\item 32 blocks $\rightarrow$ 5 index bits

\subsection{The Tag Problem in Direct-Mapped Cache}

\subsubsection{Conflict Issue}

\textbf{Multiple Blocks $\rightarrow$ Same Index:}

\begin{itemize}
\item Many memory blocks map to same cache index
\item Example: Blocks 5, 13, 21, 29, ... all map to index 5 (mod 8)
\item Only ONE can occupy cache index 5 at a time

\textbf{Example Addresses Mapping to Index 5:}

\textbf{Address A:}

Block address: ...00000101
Index bits (last 3): 101 $\rightarrow$ Index 5

\textbf{Address B:}

Block address: ...00001101
Index bits (last 3): 101 $\rightarrow$ Index 5

Both map to index 5, but different blocks!

\subsubsection{The Problem}

\begin{itemize}
\item When CPU requests address with index 5
\item Is data at index 5 for Address A or Address B?
\item Need way to differentiate between conflicting blocks

\subsubsection{Solution: TAG FIELD}

\textbf{Tag Definition:}

\begin{itemize}
\item Remaining bits of block address (excluding index and offset)
\item Stored WITH data in cache
\item Used to verify correct block is present

Tag = Block Address (excluding index bits)

\paragraph{Example Address Breakdown}

\textbf{Full Address:}

[26-bit Tag][3-bit Index][3-bit Offset]

\textbf{Address A:}

[00000000000000000000000000][101][000]
 └── Tag = 0                └─ Index=5 └─ Offset

\textbf{Address B:}

[00000000000000000000000001][101][000]
 └── Tag = 1                └─ Index=5 └─ Offset

Both have index 5, but DIFFERENT tags!

\subsubsection{Cache Structure with Tags}

| Index | Valid | Tag  | Data Block |
| ----- | ----- | ---- | ---------- |
| 0     | V     | Tag0 | [64 bits]  |
| 1     | V     | Tag1 | [64 bits]  |
| 2     | V     | Tag2 | [64 bits]  |
| 3     | V     | Tag3 | [64 bits]  |
| 4     | V     | Tag4 | [64 bits]  |
| 5     | V     | Tag5 | [64 bits]  |
| 6     | V     | Tag6 | [64 bits]  |
| 7     | V     | Tag7 | [64 bits]  |

\textbf{Storage Requirements Per Cache Entry:}

\begin{itemize}
\item Tag: 26 bits (in this example)
\item Valid bit: 1 bit
\item Data: 64 bits (8 bytes)
\item Total: 91 bits per entry

\textbf{Storage Overhead:}

Overhead = (Tag + Valid) / Total
         = (26 + 1) / (26 + 1 + 64)
         = 27 / 91
         ≈ 30% overhead in this small example

\paragraph{Note on Overhead}

\begin{itemize}
\item Example uses VERY small cache (8 blocks)
\item Real caches are much larger (thousands of blocks)
\item Larger caches $\rightarrow$ More index bits
\item More index bits $\rightarrow$ Fewer tag bits
\item Overhead percentage decreases with larger caches

\textbf{Example with Larger Cache:}

\begin{itemize}
\item 1024 blocks (2¹⁰)
\item Index: 10 bits
\item Tag: 32 - 10 - 3 = 19 bits
\item Overhead: (19+1)/84 ≈ 24% (better)

\subsubsection{Valid Bit}

\textbf{Purpose:}

\begin{itemize}
\item Indicates whether cache entry contains valid data
\item Prevents using uninitialized/stale data

\textbf{Initial State:}

\begin{itemize}
\item At program start, cache is empty
\item All entries contain garbage/random values
\item All valid bits set to 0 (invalid)

\textbf{After Data Loaded:}

\begin{itemize}
\item When block loaded into cache, valid bit set to 1
\item Indicates data is reliable

\textbf{Uses Beyond Initialization:}

\begin{itemize}
\item Cache coherence (multi-processor systems)
\item Invalidating stale data
\item Handling context switches

\subsection{Cache Read Access Operation}

\subsubsection{Read Access Process}

\textbf{CPU Provides:}

\begin{enumerate}
\item Address (word or byte address)
\item Control Signal: Read/Write indicator (from control unit)

\subsubsection{For Read Access}

\paragraph{Step 1: ADDRESS BREAKDOWN}

\begin{itemize}
\item Receive address from CPU
\item Parse into three fields:
\item Tag bits
\item Index bits
\item Offset bits

\textbf{Example Address (32-bit):}

[26-bit Tag][3-bit Index][3-bit Offset]

\paragraph{Step 2: INDEXING THE CACHE}

\begin{itemize}
\item Extract index bits from address
\item Use index to directly access cache entry
\item Combinational logic routes to correct entry
\item Like array indexing: index 5 $\rightarrow$ entry 5
\item No search needed!
\item Fast: Pure combinational delay

\textbf{Hardware:}

\begin{itemize}
\item Decoder circuit takes index bits
\item Selects one of N cache entries
\item Activates corresponding row

\paragraph{Step 3: TAG COMPARISON}

\begin{itemize}
\item Extract stored tag from selected cache entry
\item Extract tag bits from incoming address
\item Compare the two tags
\item Use comparator circuit

\textbf{Comparator Circuit:}

\begin{itemize}
\item For each bit position: XNOR gate
\item XNOR outputs 1 if bits match, 0 if different
\item AND all XNOR outputs together
\item Final output: 1 if all bits match (tags equal), 0 otherwise

\textbf{Example (4-bit tags):}

Stored tag:   1 0 1 1
Address tag:  1 0 1 1
XNOR:         1 1 1 1  $\rightarrow$ AND = 1 (MATCH!)

Stored tag:   1 0 1 1
Address tag:  1 0 0 1
XNOR:         1 1 0 1  $\rightarrow$ AND = 0 (NO MATCH)

\textbf{For N-bit tag:}

\begin{itemize}
\item N XNOR gates (parallel)
\item 1 N-input AND gate
\item Very fast combinational circuit

\paragraph{Step 4: VALID BIT CHECK}

\begin{itemize}
\item Extract valid bit from selected cache entry
\item Check if entry is valid
\item Valid bit = 1 $\rightarrow$ Entry contains valid data
\item Valid bit = 0 $\rightarrow$ Entry is invalid (ignore)

\paragraph{Step 5: HIT/MISS DETERMINATION}

\begin{itemize}
\item Combine tag comparison and valid bit
\item Hit = (Tag Match) AND (Valid Bit = 1)
\item Miss = (Tag Mismatch) OR (Valid Bit = 0)

\textbf{Logic Circuit:}

Tag Match Output ─┐
                  AND ─$\rightarrow$ Hit/Miss Signal
Valid Bit ────────┘

\textbf{Output:}

\begin{itemize}
\item 1 $\rightarrow$ HIT (data present and valid)
\item 0 $\rightarrow$ MISS (data not present or invalid)

\textbf{Hit Latency:}

\begin{itemize}
\item Time for steps 2-5
\item Dominated by:
\item Indexing combinational delay
\item Tag comparator delay
\item Valid bit access
\item Typically < 1 nanosecond for SRAM

\paragraph{Step 6: DATA EXTRACTION (Parallel with Tag Check)}

\begin{itemize}
\item Can happen in PARALLEL with tag comparison
\item Extract entire data block from selected cache entry
\item Put data block on internal wires

\textbf{Data Block:}

\begin{itemize}
\item Contains multiple words
\item Example: 8 bytes = 2 words (4 bytes each)

\paragraph{Step 7: WORD SELECTION (Using Offset)}

\begin{itemize}
\item CPU wants a single WORD, not entire block
\item Use offset bits to select correct word from block
\item Offset bits $\rightarrow$ Multiplexer select signal

\textbf{Multiplexer (MUX):}

\begin{itemize}
\item Inputs: All words in the data block
\item Select: Word offset bits from address
\item Output: Selected word

\textbf{Example (2 words per block):}

\begin{itemize}
\item Block contains: Word0 (bytes 0-3), Word1 (bytes 4-7)
\item Word offset = 0 $\rightarrow$ Select Word0
\item Word offset = 1 $\rightarrow$ Select Word1
\item Need 1-bit select for 2:1 MUX

\textbf{Example (4 words per block):}

\begin{itemize}
\item Block contains: Word0, Word1, Word2, Word3
\item Word offset = 2 bits $\rightarrow$ Select among 4 words
\item Need 4:1 MUX

\textbf{Timing:}

\begin{itemize}
\item Data extraction and word selection happen in parallel with tag check
\item Both combinational circuits
\item Similar delays
\item Can overlap operations

\paragraph{Step 8: DECISION BASED ON HIT/MISS}

\textbf{If HIT (signal = 1):}

\begin{itemize}
\item Selected word is correct data
\item Send word to CPU immediately
\item Access complete
\item Total time: Hit latency (< 1 ns)

\textbf{If MISS (signal = 0):}

\begin{itemize}
\item Selected word is WRONG data (different block or invalid)
\item CANNOT send to CPU
\item Must fetch correct block from main memory (DRAM)
\item CPU must STALL (wait)
\item Cache controller takes over
\item Total time: Hit latency + Miss penalty

\textbf{Miss Handling:}

\begin{itemize}
\item Will discuss in next lecture
\item Involves accessing main memory
\item Bringing block into cache
\item Potentially evicting old block
\item Then serving CPU request

\subsection{Cache Circuit Components Summary}

\subsubsection{Key Circuit Elements}

\paragraph{1. INDEXING CIRCUITRY}

\begin{itemize}
\item \textbf{Input:} Index bits from address
\item \textbf{Function:} Decoder to select cache entry
\item \textbf{Output:} Activates one cache row
\item \textbf{Type:} Combinational logic
\item \textbf{Delay:} Part of hit latency

\paragraph{2. TAG COMPARATOR}

\begin{itemize}
\item \textbf{Input:} Stored tag, Address tag
\item \textbf{Function:} Multi-bit equality check
\item \textbf{Components:}
\item N XNOR gates (N = tag bit width)
\item 1 N-input AND gate
\item \textbf{Output:} 1 if equal, 0 if not equal
\item \textbf{Type:} Combinational logic
\item \textbf{Delay:} Part of hit latency

\paragraph{3. VALID BIT CHECK}

\begin{itemize}
\item \textbf{Input:} Valid bit from cache entry
\item \textbf{Function:} Read and check validity
\item \textbf{Output:} 1 if valid, 0 if invalid
\item \textbf{Type:} Simple wire/buffer
\item \textbf{Delay:} Minimal

\paragraph{4. HIT/MISS LOGIC}

\begin{itemize}
\item \textbf{Input:} Tag match signal, Valid bit
\item \textbf{Function:} AND gate
\item \textbf{Output:} Hit/Miss signal
\item \textbf{Type:} Combinational logic
\item \textbf{Delay:} Single gate delay

\paragraph{5. DATA ARRAY ACCESS}

\begin{itemize}
\item \textbf{Input:} Index bits
\item \textbf{Function:} Read data block from cache
\item \textbf{Output:} Multi-word data block
\item \textbf{Type:} SRAM memory read
\item \textbf{Delay:} SRAM access time (parallel with tag check)

\paragraph{6. WORD SELECTOR (Multiplexer)}

\begin{itemize}
\item \textbf{Input:} Data block, Word offset bits
\item \textbf{Function:} Select one word from block
\item \textbf{Output:} Single word
\item \textbf{Type:} MUX (combinational)
\item \textbf{Delay:} MUX delay (parallel with tag check)

\paragraph{7. CONTROL LOGIC (Cache Controller)}

\begin{itemize}
\item \textbf{Input:} Hit/Miss signal, Read/Write control
\item \textbf{Function:} Decide next actions
\item \textbf{Output:} Control signals for CPU, memory
\item \textbf{On Hit:} Enable data to CPU
\item \textbf{On Miss:} Initiate memory fetch, stall CPU
\item \textbf{Type:} Sequential logic (state machine)

\subsubsection{Hit Latency Components}

\textbf{Contributing Factors:}

\begin{itemize}
\item Indexing delay
\item Tag comparison delay
\item Valid bit check delay
\item Hit/Miss determination delay
\item Word selection delay (parallel)
\item Wire delays

\textbf{Dominant Delays:}

\begin{itemize}
\item Indexing (decoder)
\item Tag comparator (XNOR + AND)
\item These determine critical path

\textbf{Parallelism:}

\begin{itemize}
\item Tag check and data extraction happen simultaneously
\item Reduces total hit latency
\item Only one path delay counts (whichever is longer)

\subsection{Next Lecture Preview}

\subsubsection{Topics to Cover}

\paragraph{1. Cache Miss Handling}

\begin{itemize}
\item What happens after miss is determined?
\item How to fetch block from main memory?
\item Where to place new block in cache?
\item What to do if cache location occupied?

\paragraph{2. Cache Controller State Machine}

\begin{itemize}
\item Not just combinational logic
\item Sequential control needed for misses
\item Multiple clock cycles to handle miss
\item States: Idle, Compare Tags, Allocate, Write Back, etc.

\paragraph{3. Write Operations}

\begin{itemize}
\item Read operation covered this lecture
\item Write more complex: Must update cache AND memory
\item Write policies: Write-through, Write-back
\item Dirty bits for modified blocks

\paragraph{4. Replacement Policies}

\begin{itemize}
\item When cache full, which block to evict?
\item Least Recently Used (LRU)
\item Other policies: FIFO, Random, LFU

\paragraph{5. Performance Analysis}

\begin{itemize}
\item Calculate average access time
\item Impact of hit rate, miss penalty
\item Cache size vs. performance tradeoffs

\paragraph{6. Advanced Cache Concepts}

\begin{itemize}
\item Set-associative caches (beyond direct-mapped)
\item Multi-level caches (L1, L2, L3)
\item Fully associative caches

\subsection{Key Takeaways and Summary}

\subsubsection{Historical Foundations}

\begin{itemize}
\item Early computers had no memory/software concept
\item Alan Turing conceived stored program computer (1936)
\item John von Neumann implemented it in EDVAC (1948)
\item Von Neumann architecture: Unified memory for instructions and data
\item Harvard architecture: Separate instruction and data memories

\subsubsection{Memory Technologies Hierarchy}

| Technology | Speed             | Size             | Cost                      |
| ---------- | ----------------- | ---------------- | ------------------------- |
| SRAM       | Fastest (< 1 ns)  | Smallest (KB-MB) | Most expensive ($2000/GB) |
| DRAM       | Medium (~50 ns)   | Medium (GB)      | Moderate ($10/GB)         |
| Flash      | Similar to DRAM   | Gigabytes        | Cheap (< $1/GB)           |
| Disk       | Slowest (5-10 ms) | Largest (TB)     | Cheapest (cents/GB)       |

\subsubsection{The Performance Problem}

\begin{itemize}
\item CPU cycle time: < 1 nanosecond
\item Main memory cycle time: ~50 nanoseconds
\item \textbf{Memory 50$\times$ slower than CPU!}
\item Pipeline requires memory access in 1 cycle
\item Cannot directly use DRAM for CPU memory accesses

\subsubsection{Memory Hierarchy Solution}

\begin{itemize}
\item Multiple levels: SRAM (cache) $\rightarrow$ DRAM $\rightarrow$ Disk
\item CPU accesses only top level (cache)
\item Upper levels hold subsets of lower levels
\item Trick CPU: Fast as SRAM, large as Disk
\item Requires very high hit rate (> 99.9%) at cache level

\subsubsection{Principles of Locality}

\textbf{1. Temporal Locality:} Recently accessed data likely accessed again soon

\begin{itemize}
\item Example: Loop variables, instructions in loops

\textbf{2. Spatial Locality:} Data near recently accessed data likely accessed soon

\begin{itemize}
\item Example: Array elements, sequential instructions

\begin{itemize}
\item Both principles present in virtually all programs
\item Foundation for cache effectiveness

\subsubsection{Memory Addressing}

\begin{itemize}
\item \textbf{Byte Address:} Individual byte reference (full address)
\item \textbf{Word Address:} 4-byte word reference (last 2 bits = 00 for alignment)
\item \textbf{Block Address:} Multiple-word block reference (excludes offset bits)
\item Address structure: [Block Address][Offset]
\item Offset subdivides: [Word Offset][Byte in Word]

\subsubsection{Cache Terminology}

\begin{itemize}
\item \textbf{Hit:} Data found in cache $\rightarrow$ Fast access (< 1 ns)
\item \textbf{Miss:} Data not in cache $\rightarrow$ Slow access (+ ~100 ns penalty)
\item \textbf{Hit Rate:} Fraction of accesses that hit (want > 99.9%)
\item \textbf{Miss Rate:} Fraction of accesses that miss (1 - Hit Rate)
\item \textbf{Hit Latency:} Time to determine hit and access data
\item \textbf{Miss Penalty:} EXTRA time to fetch from memory on miss

\subsubsection{Cache Organization (Direct-Mapped)}

\begin{itemize}
\item Each memory block maps to exactly ONE cache location
\item Mapping: Cache Index = Block Address mod (Cache Size)
\item Address fields: [Tag][Index][Offset]
\item \textbf{Index:} Selects cache entry directly (no search!)
\item \textbf{Tag:} Differentiates blocks mapping to same index
\item \textbf{Offset:} Selects word/byte within block
\item \textbf{Valid bit:} Indicates if entry contains valid data

\subsubsection{Direct-Mapped Cache Structure}

\begin{itemize}
\item \textbf{Tag array:} Stores tags for verification
\item \textbf{Valid bit array:} Validity indicators
\item \textbf{Data array:} Stores actual data blocks
\item Index not stored (implicit in position)

\subsubsection{Cache Read Access Process}

\begin{enumerate}
\item Extract index from address $\rightarrow$ Access cache entry
\item Extract tag from cache entry $\rightarrow$ Compare with address tag
\item Check valid bit from entry
\item Determine hit/miss: (Tag Match) AND (Valid)
\item In parallel: Extract data block, select word using offset
\item If HIT: Send word to CPU (done in < 1 ns)
\item If MISS: Must fetch from memory (will cover next lecture)

\subsubsection{Critical Requirements}

\begin{itemize}
\item Hit latency must be < 1 CPU clock cycle
\item Hit rate must be very high (> 99.9%)
\item Only way to achieve: Exploit locality principles
\item Direct mapping enables fast indexing (no search)
\item Parallel tag check and data extraction minimize latency

\subsubsection{Average Access Time Formula}

Average Access Time = Hit Latency + (Miss Rate $\times$ Miss Penalty)

\begin{itemize}
\item Must keep Miss Rate very low for performance
\item Even 1% miss rate catastrophic if penalty is 100$\times$
\item Example: 1% miss rate $\rightarrow$ 1 + (0.01 $\times$ 100) = 2 ns average
\item Example: 0.1% miss rate $\rightarrow$ 1 + (0.001 $\times$ 100) = 1.1 ns average
\item Target: 99.9% or better hit rate

\subsubsection{Pending Topics (Next Lectures)}

\begin{itemize}
\item Cache miss handling and memory fetch
\item Cache controller state machine
\item Write operations and write policies
\item Block replacement strategies (LRU, etc.)
\item Set-associative and fully associative caches
\item Multi-level cache hierarchies
\item Performance analysis and optimization

\subsubsection{Music Library Analogy Summary}

\begin{itemize}
\item \textbf{Phone (cache):} Small, fast, always accessible
\item \textbf{Computer (main memory):} Larger, slower, main collection
\item \textbf{Internet (disk):} Huge, slowest, everything available
\item Listen from phone (CPU accesses cache)
\item Copy from computer when song not on phone (fetch on miss)
\item Download from internet when not on computer (fetch from disk)
\item Keep favorite songs on phone (exploit temporal locality)
\item Copy whole album at once (exploit spatial locality)

\subsection{Key Takeaways}

\begin{enumerate}
\item \textbf{Stored-program concept} revolutionized computing—programs stored in memory like data, eliminating manual reconfiguration for each algorithm.

\begin{enumerate}
\item \textbf{Von Neumann architecture} established fundamental computer organization—CPU, memory, and I/O with instructions and data sharing same memory.

\begin{enumerate}
\item \textbf{Processor-memory speed gap} creates performance bottleneck—CPU operates at nanosecond scale while main memory requires tens of nanoseconds.

\begin{enumerate}
\item \textbf{Memory hierarchy} provides illusion of large, fast memory—small fast cache near CPU, larger slower DRAM main memory, massive slow disk storage.

\begin{enumerate}
\item \textbf{Temporal locality}: Recently accessed data likely accessed again soon—programs exhibit loops, function calls, and repeated variable access patterns.

\begin{enumerate}
\item \textbf{Spatial locality}: Nearby data likely accessed soon—programs access arrays sequentially and instructions execute in order.

\begin{enumerate}
\item \textbf{Cache exploits locality} to achieve high hit rates—keeping frequently accessed data in fast storage dramatically improves average access time.

\begin{enumerate}
\item \textbf{Cache organized in blocks}, not individual words—exploiting spatial locality by fetching multiple words together.

\begin{enumerate}
\item \textbf{Direct-mapped cache}: Each memory block maps to exactly one cache location—simplest cache organization using modulo arithmetic for mapping.

10. \textbf{Address breakdown}: Tag + Index + Offset—index selects cache entry, tag identifies specific block, offset selects word within block.

11. \textbf{Valid bit} indicates cache entry contains meaningful data—essential for distinguishing real data from uninitialized entries at startup.

12. \textbf{Cache hit} occurs when requested data found in cache—CPU receives data in ~1 nanosecond, avoiding slow main memory access.

13. \textbf{Cache miss} requires main memory fetch—takes ~100 nanoseconds, replacing cache entry with new block from memory.

14. \textbf{Hit rate determines cache effectiveness}—even 1% miss rate significantly impacts average memory access time with 100$\times$ penalty.

15. \textbf{Block size affects performance}—larger blocks exploit spatial locality better but reduce total number of blocks, potentially increasing conflicts.

16. \textbf{Cache size} represents total data storage capacity—typical L1 caches 32-64 KB, L2 caches 256 KB-1 MB.

17. \textbf{Tag comparison} happens in parallel with data access—enabling fast hit detection and maintaining single-cycle cache access.

18. \textbf{Music library analogy} clarifies cache concept—phone (cache) holds favorites, computer (DRAM) has main collection, internet (disk) contains everything.

19. \textbf{Cache transparent to programmer}—software sees uniform memory, hardware manages cache automatically for best performance.

20. \textbf{Memory hierarchy only works because programs exhibit locality}—without temporal and spatial locality, caching would fail catastrophically.

\subsection{Summary}

The introduction to memory systems and cache memory reveals how the fundamental processor-memory speed gap—with CPUs operating 100$\times$ faster than main memory—drives sophisticated cache hierarchy designs that create the illusion of large, fast memory. Historical context from Alan Turing's theoretical foundations through Von Neumann's stored-program architecture establishes how modern computers execute instructions fetched from memory rather than requiring manual reconfiguration. The memory hierarchy concept, with small fast SRAM caches near the CPU, larger slower DRAM main memory, and massive disk storage, exploits two fundamental program properties: temporal locality (recently accessed data likely accessed again soon) and spatial locality (nearby data likely accessed soon). Cache memory, organized in blocks rather than individual words, dramatically improves average access time by maintaining frequently accessed data in fast storage, achieving hit rates often exceeding 95% in practice. Direct-mapped cache organization, the simplest mapping scheme, uses modulo arithmetic to assign each memory block to exactly one cache location, with address bits divided into tag (identifying specific block), index (selecting cache entry), and offset (choosing word within block). The valid bit distinguishes real cached data from uninitialized entries, essential at system startup when cache contains random values. Cache hits deliver data in approximately 1 nanosecond while misses require ~100 nanosecond main memory access, making even small miss rates significant—a 1% miss rate doubles average access time from 1 ns to 2 ns. The music library analogy effectively clarifies concepts: phone storage represents cache (small, fast, always accessible), computer storage represents main memory (larger, slower, main collection), and internet streaming represents disk (unlimited, very slow, backup). This cache transparency—programmer sees uniform memory while hardware automatically manages caching—enables software compatibility across different cache configurations. The critical insight remains that memory hierarchy effectiveness depends entirely on programs exhibiting locality; without these natural access patterns inherent to how we write code, caching would provide no benefit. Understanding cache fundamentals proves essential for both hardware designers optimizing cache architectures and software developers writing cache-friendly code that maximizes hit rates.

% \section{Lecture 15: Cache Memory Operations – Read/Write Access and Write Policies}

\emph{By Dr. Isuru Nawinne}

\subsection{Introduction}

This lecture provides a comprehensive, step‑by‑step examination of how a direct‑mapped cache services read and write requests, differentiates hits from misses, and preserves data correctness. We finish the full read path (including stall + block fetch sequence), analyze write hits and misses, and introduce the write‑through policy as the simplest consistency mechanism between cache and main memory. Performance consequences of constant memory writes, the need for high hit rates, and the motivation for more advanced write‑back policies (next lecture) are emphasized. By the end you will understand exactly what the cache controller must do (state transitions, signals, data/tag/valid updates) for every access type and why write policies are a central architectural tradeoff.

\subsection{Lecture Introduction and Recap}

\subsubsection{Previous Lecture Review}

\paragraph{Memory Systems Foundation}

\begin{itemize}
\item Memory hierarchy concept (SRAM $\rightarrow$ DRAM $\rightarrow$ Disk)
\item Illusion of large and fast memory simultaneously
\item CPU accesses only cache (top level)

\paragraph{Locality Principles}

\begin{itemize}
\item \textbf{Temporal locality:} Recently accessed data likely accessed again soon
\item \textbf{Spatial locality:} Nearby data likely accessed soon
\item Foundation for cache effectiveness

\paragraph{Direct-Mapped Cache Introduction}

\begin{itemize}
\item Each memory block maps to exactly ONE cache location
\item Mapping function: Cache Index = Block Address MOD Cache Size
\item Read access process partially covered

\paragraph{Cache Structure (Recap)}

\begin{itemize}
\item \textbf{Data array:} Stores data blocks (not individual words)
\item \textbf{Tag array:} Stores tags for block identification
\item \textbf{Valid bit array:} Indicates valid/invalid entries
\item \textbf{Index:} Not stored, implicit in position (for convenience in diagrams)

\paragraph{Address Breakdown (Recap)}

[Tag][Index][Offset]
  ^      ^       ^
  |      |       └── Identifies word/byte within block
  |      └── Identifies cache entry (direct mapping)
  └── Remaining bits for block identification

\subsubsection{Today's Focus}

\begin{itemize}
\item Complete discussion of read miss handling
\item Write access operations (hit and miss)
\item Write policies and their implications
\item Data consistency issues
\item Performance considerations

\subsection{Cache Read Access - Complete Process}

\subsubsection{Read Access Input Signals}

\textbf{From CPU to Cache Controller:}

\begin{enumerate}
\item \textbf{Address} (word or byte address)
\item \textbf{Read Control Signal} (from CPU control unit)
\item Indicates this is a read operation (not write)
\item Part of memory control signals

\subsubsection{Cache Read Steps (Detailed)}

\paragraph{Step 1: Address Decomposition}

\begin{itemize}
\item Parse incoming address into three fields:
\item \textbf{Tag:} For verification
\item \textbf{Index:} For cache entry selection
\item \textbf{Offset:} For word/byte selection within block

\paragraph{Step 2: Cache Entry Selection (Indexing)}

\begin{itemize}
\item Extract index bits from address
\item Cache controller knows which bits are index (by design)
\item Use demultiplexer circuitry to access correct cache entry
\item Example: Index = 101 (binary) $\rightarrow$ Access cache entry 5
\item Direct access, no search needed
\item Combinational logic (fast)

\paragraph{Step 3: Tag Comparison}

\begin{itemize}
\item Extract stored tag from selected cache entry
\item Extract tag from incoming address
\item Use comparator circuit (XNOR gates + AND gate)
\item Output: 1 if tags match, 0 if tags differ

\paragraph{Step 4: Valid Bit Check}

\begin{itemize}
\item Extract valid bit from selected cache entry
\item Check if entry contains valid data
\item Output: 1 if valid, 0 if invalid

\paragraph{Step 5: Hit/Miss Determination}

\begin{itemize}
\item Logic: \textbf{Hit = (Tag Match) AND (Valid Bit)}
\item If both conditions true $\rightarrow$ HIT
\item If either condition false $\rightarrow$ MISS
\item Single AND gate combines both signals

\paragraph{Step 6: Data Extraction (Parallel Operation)}

\begin{itemize}
\item Happens simultaneously with tag comparison
\item Extract entire data block from cache entry
\item Place data block on internal wires
\item Example: 8-byte block = 2 words

\paragraph{Step 7: Word Selection (Using Offset)}

\begin{itemize}
\item CPU requests a single WORD
\item Use word offset bits as MUX select signal
\item Example: 2 words in block
\item Offset MSB = 0 $\rightarrow$ Select first word
\item Offset MSB = 1 $\rightarrow$ Select second word
\item Multiplexer extracts correct word from block

\subsubsection{Timing Optimization}

\textbf{Parallel Operations:}

\begin{itemize}
\item Tag comparison (Steps 3-5) and data extraction (Steps 6-7) happen in PARALLEL
\item Both are combinational circuits
\item Total delay = max(tag comparison delay, data extraction delay)
\item Reduces overall hit latency

\subsubsection{Read Hit Outcome}

\begin{itemize}
\item Selected word is correct data
\item Send word to CPU immediately
\item No stall required
\item Total time: Hit latency (< 1 nanosecond for SRAM)
\item Completes within one CPU clock cycle
\item Pipeline continues uninterrupted

\subsubsection{Pipeline Integration}

\begin{itemize}
\item In MIPS pipeline, MEM stage accesses memory
\item With cache hit: Memory access completes in 1 cycle
\item Pipeline maintains smooth operation
\item No bubbles inserted

\subsection{Cache Read Miss Handling}

\subsubsection{Read Miss Scenario}

\paragraph{Miss Conditions}

\begin{enumerate}
\item \textbf{Tag mismatch} (most common)
\item Requested block not in cache
\item Different block occupies that cache location
\item \textbf{Invalid entry}
\item Valid bit = 0
\item Entry contains no valid data (e.g., after initialization)
\item \textbf{Both conditions}
\item Tag mismatch AND invalid entry

\subsubsection{Read Miss Response Required Actions}

\paragraph{Action 1: STALL THE CPU}

\textbf{Process:}

\begin{itemize}
\item CPU cannot proceed without requested data
\item Data hazard would occur if CPU continues
\item Cache controller sends STALL signal to CPU
\item CPU must monitor stall signal continuously
\item When stall signal high $\rightarrow$ Freeze CPU operation
\item Stop fetching new instructions
\item Freeze all pipeline stages
\item Hold current state

\textbf{CPU's Perspective:}

\begin{itemize}
\item CPU doesn't know cache and memory are separate
\item CPU sees memory hierarchy as single "memory"
\item Must respond to stall signal from memory subsystem
\item In MEM stage: Check and respond to stall signal

\paragraph{Action 2: MAKE READ REQUEST TO MAIN MEMORY}

\textbf{Request Details:}

\begin{itemize}
\item Request the missing DATA BLOCK (not just word!)
\item Cache and memory trade in BLOCKS
\item CPU trades in words/bytes, but cache-memory interface uses blocks
\item Send block address to main memory
\item Memory fetches entire block

\textbf{Reason for Block Transfer:}

\begin{itemize}
\item Exploits spatial locality
\item Fetches requested word AND nearby words
\item Reduces future misses for nearby addresses
\item More efficient than fetching single words

\textbf{Memory Access Time:}

\begin{itemize}
\item DRAM access: Several CPU clock cycles
\item Range: 10 to 100+ CPU clock cycles
\item Much slower than cache (< 1 cycle)
\item This is the \textbf{MISS PENALTY}

\paragraph{Action 3: WAIT FOR MEMORY RESPONSE}

\begin{itemize}
\item Memory performs read operation
\item Data travels from memory to cache
\item Controller waits (CPU still stalled)
\item Multiple clock cycles elapse

\paragraph{Action 4: UPDATE CACHE ENTRY}

\textbf{Three components to update:}

\textbf{a) Update Data Block:}

\begin{itemize}
\item Write fetched block into cache entry
\item Replace old data at that index

\textbf{b) Update Tag:}

\begin{itemize}
\item Extract tag from block address
\item Write tag into tag array at that index
\item Ensures future tag comparisons work correctly

\textbf{c) Set Valid Bit:}

\begin{itemize}
\item Set valid bit to 1
\item Denotes entry now contains valid data

\paragraph{Action 5: SEND DATA TO CPU}

\begin{itemize}
\item Extract requested word from newly loaded block
\item Use offset to select correct word
\item Put data on bus to CPU
\item CPU receives requested data

\paragraph{Action 6: CLEAR STALL SIGNAL}

\begin{itemize}
\item Cache controller clears (lowers) stall signal
\item CPU detects stall signal going low
\item CPU resumes operation
\item Pipeline unfreezes and continues

\subsubsection{Total Read Miss Time}

\textbf{Formula:}

Read Miss Time = Hit Latency + Miss Penalty

\textbf{Where:}

\begin{itemize}
\item \textbf{Hit Latency:} Time to determine it's a miss (< 1 ns)
\item \textbf{Miss Penalty:} Time to fetch from memory (10-100+ CPU cycles)

\textbf{Example Calculation:}

\begin{itemize}
\item Hit latency: 1 ns (1 cycle at 1 GHz)
\item Miss penalty: 50 ns (50 cycles at 1 GHz)
\item Total: 1 + 50 = 51 cycles

\subsubsection{Performance Impact}

\begin{itemize}
\item Single miss causes 50+ cycle stall
\item Catastrophic for pipeline performance
\item Emphasizes need for high hit rate (> 99.9%)

\subsubsection{Question: What About the Old Block?}

\textbf{The Deferred Question:}

\begin{itemize}
\item When fetching new block on miss
\item Old block occupies that cache entry
\item What happens to old block?
\item Is it okay to discard it?

\textbf{Initial Answer:} "We'll discuss after introducing write policies"

\begin{itemize}
\item Answer depends on write policy
\item Need to understand writes first
\item Question will be revisited

\subsection{Cache Write Access - Introduction}

\subsubsection{Write Access Input Signals}

\textbf{From CPU to Cache Controller:}

\begin{enumerate}
\item \textbf{Address} (where to write)
\item \textbf{Data Word} (what to write)
\item \textbf{Write Control Signal} (indicates write operation)

Three inputs vs. two for read (no data input needed for read).

\subsubsection{Write Access Process}

\paragraph{Step 1: Address Decomposition}

\begin{itemize}
\item Same as read: [Tag][Index][Offset]

\paragraph{Step 2: Cache Entry Selection}

\begin{itemize}
\item Same as read: Use index bits
\item Demultiplexer accesses correct entry
\item Direct access based on index
\item Example: Index 101 $\rightarrow$ Entry 5

\paragraph{Step 3: Tag Comparison}

\begin{itemize}
\item Extract tag from cache entry
\item Compare with incoming address tag
\item Comparator circuit (same as read)
\item Output: Match or no match

\paragraph{Step 4: Valid Bit Check}

\begin{itemize}
\item Extract and check valid bit
\item Same as read operation
\item Ensures entry is valid

\paragraph{Step 5: Hit/Miss Determination}

\begin{itemize}
\item Hit = (Tag Match) AND (Valid Bit)
\item Same logic as read
\item Determines write hit or write miss

\paragraph{Step 6: Data Writing (The Difference)}

\textbf{This is where write differs from read:}

\begin{itemize}
\item Must write data word to correct location in block
\item Use offset to determine which word in block

\subsubsection{Writing Mechanism}

\textbf{Input:}

\begin{itemize}
\item Incoming data word (from CPU)
\item Offset bits from address

\textbf{Demultiplexer Selection:}

\begin{itemize}
\item Use word offset as demultiplexer select signal
\item Example with 2 words per block:
\item Word offset = 0 $\rightarrow$ Write to first word
\item Word offset = 1 $\rightarrow$ Write to second word
\item Demultiplexer directs data to correct word position

\textbf{Example:}

\begin{itemize}
\item Block has 2 words: Word0 (bytes 0-3), Word1 (bytes 4-7)
\item Incoming data word: 0x12345678
\item Offset MSB = 1 $\rightarrow$ Select Word1
\item Demux directs data to Word1 position in block

\textbf{Write Operation Control:}

\begin{itemize}
\item Writing controlled by Write control signal from CPU
\item Only write if signal indicates write operation
\item Demultiplexer enabled by write signal

\subsubsection{Critical Question: Can Write and Tag Compare Happen in Parallel?}

\paragraph{For Read (Previous Discussion)}

\begin{itemize}
\item \textbf{YES, both can happen in parallel}
\item If miss, discard extracted data (no harm done)
\item Reading doesn't change cache state

\paragraph{For Write (Current Question)}

\textbf{More problematic!}

\begin{itemize}
\item What if we write and then discover tag mismatch?

\textbf{Scenario:}

\begin{itemize}
\item Write to cache entry simultaneously with tag comparison
\item Tag comparison returns MISMATCH
\item We've now CORRUPTED data in cache!
\item Written to wrong block (different tag)
\item Data integrity violated

\textbf{Problem:}

\begin{itemize}
\item If invalid entry: Not too serious (data was garbage anyway)
\item If tag mismatch: \textbf{SERIOUS problem!}
\item Overwrote valid data for different block
\item That block's data now corrupted
\item Future accesses to that block get wrong data

\textbf{Initial Conclusion:}

\begin{itemize}
\item Cannot safely write and tag compare in parallel
\item Need mechanism to prevent corruption
\item Solution depends on write policy (discussed next)

\subsection{Write Policies - Introduction}

\subsubsection{The Data Consistency Problem}

\textbf{Scenario:}

\begin{itemize}
\item CPU writes to address A
\item Address A hits in cache
\item Cache controller writes new value to cache entry
\item Cache now has updated value
\item Main memory still has OLD value
\item Two versions exist: Cache version $\neq$ Memory version

\textbf{The Inconsistency:}

\begin{itemize}
\item Cache entry now INCONSISTENT with main memory
\item Same address has different values in different levels
\item Data coherence problem

\subsubsection{Why This Matters}

\begin{itemize}
\item Future access to same address: Which value is correct?
\item If cache entry replaced: New value lost
\item I/O devices may access memory directly (bypass cache)
\item Multi-processor systems: Other CPUs access memory
\item Must maintain data consistency across hierarchy

\subsubsection{Two Fundamental Write Policies}

\begin{enumerate}
\item \textbf{Write-Through} (discussed this lecture)
\item \textbf{Write-Back} (mentioned, detailed in next lecture)

\subsection{Write-Through Policy}

\subsubsection{Write-Through Definition}

\textbf{Policy Statement:}

> "Always write to BOTH cache AND memory"

\textbf{Mechanism:}

\begin{itemize}
\item On every write operation:
\end{itemize}
\begin{enumerate}
\item Write to cache (if hit)
\item Simultaneously write to main memory
\item Both levels updated together
\item Ensures cache and memory always consistent

\subsubsection{Write-Through Process}

\paragraph{Write Hit with Write-Through}

\begin{enumerate}
\item Determine it's a write hit (tag match + valid)
\item Write data word to cache block (using offset)
\item Also send write request to main memory
\item Update same address in memory
\item Wait for memory write to complete
\item Both cache and memory now have same value

\paragraph{Write Miss with Write-Through}

\begin{enumerate}
\item Determine it's a write miss
\item Stall CPU
\item Fetch missing block from memory (read operation)
\item Update cache entry with fetched block
\item Write the word to correct position in block
\item Also write to memory
\item Clear stall signal
\item Both levels updated

\subsubsection{Advantages of Write-Through}

\paragraph{Advantage 1: SIMPLICITY}

\begin{itemize}
\item Straightforward to implement
\item No complex consistency protocols
\item Cache controller logic simpler
\item Design principle: Keep cache simple

\paragraph{Advantage 2: CONSISTENCY GUARANTEED}

\begin{itemize}
\item Cache and memory ALWAYS have same values
\item No special handling for discarded blocks
\item Can replace any cache entry anytime
\item Memory always has correct, up-to-date data

\paragraph{Advantage 3: ANSWERS THE OLD BLOCK QUESTION}

\textbf{With write-through policy:}

\begin{itemize}
\item Old block can be safely discarded
\item All updates were written to memory
\item Memory has latest version
\item Future accesses can fetch from memory
\item No data loss

\textbf{Comparison:}

\begin{itemize}
\item Read miss: Old block discarded, data available in memory
\item Write with write-through: Always updated memory, safe to discard

\paragraph{Advantage 4: PARALLEL WRITE AND TAG COMPARE NOW POSSIBLE!}

\textbf{Critical Insight:}
Can now overlap write and tag comparison. Why? Two scenarios:

\textbf{Scenario A: Write Hit}

\begin{itemize}
\item Written to cache, will also write to memory
\item Tag matches, write is correct
\item Both cache and memory updated
\item No problem

\textbf{Scenario B: Write Miss}

\begin{itemize}
\item Written to cache entry (possibly wrong block)
\item Tag mismatch detected
\item Will fetch correct block from memory anyway
\item Will overwrite cache entry with correct block
\item Corrupted data gets replaced immediately
\item Memory has correct version (wasn't corrupted)
\item No lasting damage

\textbf{Result:}

\begin{itemize}
\item Safe to write and tag compare in parallel
\item Saves time (hit latency reduced)
\item Both operations in same clock cycle
\item If hit: Saved time
\item If miss: No harm (will fix cache anyway)

\textbf{Timing Optimization:}

\begin{itemize}
\item Tag comparison time: T_comp
\item Write time: T_write
\item Without overlap: Total = T_comp + T_write
\item With overlap: Total = max(T_comp, T_write)
\item Typically similar delays $\rightarrow$ Nearly 2$\times$ speedup

\subsubsection{Disadvantages of Write-Through}

\paragraph{Disadvantage 1: EXCESSIVE WRITE TRAFFIC}

\begin{itemize}
\item EVERY write goes to memory
\item Memory writes are slow (10-100+ cycles)
\item Generates continuous memory traffic
\item Memory bus congestion

\paragraph{Disadvantage 2: CPU STALLS ON EVERY WRITE}

\textbf{Critical Problem:}

\begin{itemize}
\item Every write requires memory access
\item Memory much slower than cache
\item CPU must stall for EVERY write
\item Wait for memory write to complete

\textbf{Stall Duration:}

\begin{itemize}
\item Memory write: 10-100 CPU clock cycles
\item Every store instruction causes stall
\item Even on write HIT!

\textbf{Example:}

\begin{itemize}
\item Store instruction hits in cache
\item Still must wait for memory write
\item 50 cycle stall for every store
\item Pipeline essentially stops

\textbf{Impact on Programs with Many Writes:}

\begin{itemize}
\item Programs with frequent store instructions
\item Array updates, structure modifications
\item Loop counters being updated
\item String manipulation
\item All suffer severe performance degradation

\textbf{Performance Comparison:}

\begin{itemize}
\item Read hit: < 1 cycle (fast!)
\item Write hit with write-through: 50+ cycles (slow!)
\item Asymmetry: Reads fast, writes catastrophically slow

\textbf{Pipeline Impact:}

\begin{itemize}
\item Recall pipelining lectures: Minimized stalls
\item Worked hard to avoid 1-2 cycle stalls
\item Write-through introduces 50+ cycle stalls regularly
\item Contradicts pipeline optimization goals
\item "Doesn't add up" - unacceptable performance loss

\textbf{Real-World Issue:}

\begin{itemize}
\item Write-through used in some systems
\item But with additional optimizations (write buffers, discussed later)
\item Pure write-through too slow for modern systems

\paragraph{Disadvantage 3: POWER CONSUMPTION}

\begin{itemize}
\item Memory accesses consume power
\item Every write $\rightarrow$ Memory access $\rightarrow$ Power consumption
\item Unnecessary power usage
\item Critical for mobile/embedded systems

\paragraph{Disadvantage 4: MEMORY WEAR}

\begin{itemize}
\item Flash memory: Limited write cycles
\item SSDs wear out with writes
\item Write-through accelerates wear
\item Reduces memory lifespan

\subsection{Resolving the Old Block Question}

\subsubsection{The Question Revisited}

\textbf{Original Question:}

> "What happens to the old block when we fetch a new block from memory on a miss?"

\textbf{Context:}

\begin{itemize}
\item Read or write miss occurs
\item Need to fetch missing block from memory
\item Old block occupies target cache entry
\item Must replace old block with new block
\item Is it safe to discard old block?

\subsubsection{Answer with Write-Through Policy}

\textbf{YES, Safe to Discard}

\paragraph{Reason 1: Memory Has Updated Version}

\begin{itemize}
\item Write-through ensures every write goes to memory
\item All modifications reflected in memory
\item Memory always has latest version of all blocks
\item Old block's latest state is in memory

\paragraph{Reason 2: Can Re-fetch If Needed}

\begin{itemize}
\item Future access to old block's address
\item Will miss in cache (block was replaced)
\item Can fetch from memory again
\item Memory has correct, up-to-date data
\item No data loss

\subsubsection{Example Scenario}

\begin{enumerate}
\item Block A in cache at index 3
\item Block A modified several times
\item Each modification written to cache AND memory
\item Block B (also maps to index 3) is requested
\item Miss occurs for Block B
\item Fetch Block B from memory
\item Replace Block A with Block B at index 3
\item Block A discarded from cache
\item Block A's data safe in memory
\end{enumerate}

10. Later access to Block A: Miss, fetch from memory again

\subsubsection{Comparison with Invalid Entry}

\begin{itemize}
\item If miss due to invalid bit: Obviously safe to replace
\item If miss due to tag mismatch: Safe because of write-through

\subsubsection{Contrast with Future Policy (Teaser)}

\begin{itemize}
\item With other write policies (write-back), answer may differ
\item May NOT be safe to discard old block
\item Will discuss in next lecture

\textbf{Conclusion:}

\begin{itemize}
\item Write-through simplifies replacement
\item No special checks needed before replacing block
\item Always safe to overwrite cache entry
\item Memory serves as reliable backup

\subsection{Parallelism in Write Access with Write-Through}

\subsubsection{The Parallel Write Problem Solved}

\textbf{Original Concern:}

\begin{itemize}
\item Want to overlap write operation and tag comparison
\item Reduce hit latency
\item But risk corrupting data if tag mismatch

\subsubsection{With Write-Through Policy}

\paragraph{Case 1: Write Hit}

\begin{itemize}
\item Write to cache and tag compare happen in parallel
\item Tag matches $\rightarrow$ It was a hit
\item Cache entry correctly updated
\item Also write to memory (per write-through policy)
\item Both cache and memory consistent
\item Time saved: One cycle
\item No problem!

\paragraph{Case 2: Write Miss}

\begin{itemize}
\item Write to cache and tag compare happen in parallel
\item Tag doesn't match $\rightarrow$ It was a miss
\item Cache entry might be corrupted (wrote to wrong block)
\item \textbf{BUT:} About to fetch correct block from memory
\item Will OVERWRITE this cache entry with new block
\item Corrupted data disappears immediately
\item Also, write goes to memory (correct address in memory)
\item End result: Cache fixed, memory correct

\subsubsection{Key Insight}

\begin{itemize}
\item Write-through to memory preserves correctness
\item Memory write goes to CORRECT address (from address bus)
\item Even if cache entry temporarily corrupted
\item Cache entry will be fixed when correct block loaded
\item Memory never corrupted

\subsubsection{Timeline for Write Miss}

Cycle 1: Write to cache (possibly wrong block) + Tag compare
Cycle 1: Also initiate memory write (correct address)
Cycle 2-50: Fetch correct block from memory
Cycle 51: Overwrite cache entry with correct block
Result: Cache correct, memory correct

\subsubsection{Safety Guarantee}

\begin{itemize}
\item \textbf{Memory write:} Targets address from address bus (always correct)
\item \textbf{Cache write:} Targets index (might be for different block)
\item \textbf{If miss:} Cache mistake corrected by fetch
\item \textbf{If hit:} No mistake, everything correct
\item \textbf{In both cases:} End state correct

\subsubsection{Performance Benefit}

\begin{itemize}
\item Saved cycles on write hit path
\item Write and tag compare: Parallel instead of sequential
\item Approximately 2$\times$ faster hit determination
\item Critical for frequent write hits

\subsubsection{Enabled by Write-Through}

\begin{itemize}
\item Only possible because memory updated on every write
\item Other policies may not allow this optimization
\item Write-through sacrifices write performance for simplicity
\item But enables some optimizations

\subsection{Summary of Cache Operations}

\subsubsection{Complete Cache Operation Overview}

\paragraph{READ HIT}

\begin{itemize}
\item Index $\rightarrow$ Tag compare + Valid check $\rightarrow$ Match
\item Extract data block $\rightarrow$ Select word $\rightarrow$ Send to CPU
\item Time: < 1 cycle (hit latency only)
\item No stall
\item Pipeline continues

\paragraph{READ MISS}

\begin{itemize}
\item Index $\rightarrow$ Tag compare + Valid check $\rightarrow$ No match
\item Stall CPU
\item Fetch block from memory (10-100+ cycles)
\item Update cache: Data + Tag + Valid bit
\item Extract word $\rightarrow$ Send to CPU
\item Clear stall
\item Time: Hit latency + Miss penalty
\item Major pipeline disruption

\paragraph{WRITE HIT (with Write-Through)}

\begin{itemize}
\item Index $\rightarrow$ Tag compare + Valid check (parallel with write)
\item Write word to cache block
\item Also write to memory (10-100+ cycles)
\item Stall CPU until memory write completes
\item Time: Hit latency + Memory write time
\item Slower than read hit!

\paragraph{WRITE MISS (with Write-Through)}

\begin{itemize}
\item Index $\rightarrow$ Tag compare + Valid check $\rightarrow$ No match
\item Stall CPU
\item Fetch block from memory
\item Update cache: Data + Tag + Valid bit
\item Write word to cache block
\item Also write to memory
\item Clear stall
\item Time: Hit latency + Miss penalty + Memory write time
\item Even slower than read miss!

\subsubsection{Performance Characteristics}

| Case                                           | Time        | Comment                                                           |
| ---------------------------------------------- | ----------- | ----------------------------------------------------------------- |
| \textbf{Best Case (Read Hit)}                       | < 1 cycle   | Optimal performance. Want this to be most common case             |
| \textbf{Moderate Case (Read Miss)}                  | 50+ cycles  | Acceptable if infrequent. Reason for high hit rate requirement    |
| \textbf{Poor Case (Write Hit with Write-Through)}   | 50+ cycles  | Every write hits this case. Unacceptable for write-heavy programs |
| \textbf{Worst Case (Write Miss with Write-Through)} | 100+ cycles | Rare but extremely slow. Catastrophic when occurs                 |

\textbf{Performance Goal:}

\begin{itemize}
\item Maximize read hits
\item Minimize write impact (better policy needed)
\item Overall hit rate > 99.9%

\subsection{Write-Through Policy Evaluation}

\subsubsection{Summary of Write-Through}

\textbf{Mechanism:}

\begin{itemize}
\item Write to cache (if hit) AND memory
\item Always keep both consistent
\item Memory is authoritative backup

\textbf{Implementation Complexity:}

\begin{itemize}
\item Simple cache controller logic
\item No complex state tracking
\item Straightforward consistency maintenance

\subsubsection{Advantages}

| Advantage                    | Description                                                                                                               |
| ---------------------------- | ------------------------------------------------------------------------------------------------------------------------- |
| \textbf{1. Simplicity}            | Easy to understand, simple to implement, minimal controller complexity, aligns with design principle (simple cache)       |
| \textbf{2. Consistency}           | Cache and memory always consistent, no special synchronization needed, can discard blocks anytime, memory always reliable |
| \textbf{3. Data Safety}           | No data loss on block replacement, memory has all updates, crash recovery simpler, I/O devices see correct data           |
| \textbf{4. Enables Optimizations} | Can overlap write and tag compare, reduces hit latency, safe due to memory backup                                         |

\subsubsection{Disadvantages}

| Disadvantage               | Description                                                                                                                                |
| -------------------------- | ------------------------------------------------------------------------------------------------------------------------------------------ |
| \textbf{1. Performance Penalty} | Every write stalls CPU, 10-100+ cycle stalls per write, unacceptable for write-intensive programs, contradicts pipeline optimization goals |
| \textbf{2. Memory Traffic}      | Excessive write traffic to memory, memory bus congestion, reduces available bandwidth for read misses, slows down entire system            |
| \textbf{3. Power Consumption}   | Every write powers up memory, unnecessary power usage, battery drain in mobile devices, heat generation                                    |
| \textbf{4. Memory Wear}         | Flash/SSD: Limited write cycles, accelerated wear-out, reduced memory lifespan, particularly bad for SSDs                                  |

\subsubsection{When Write-Through Used}

\paragraph{Suitable Applications}

\begin{itemize}
\item Read-heavy workloads
\item Simple embedded systems
\item Systems requiring guaranteed consistency
\item Safety-critical applications

\paragraph{Real-World Usage}

\begin{itemize}
\item Often combined with write buffers
\item Write buffer: Small queue for pending writes
\item CPU continues after writing to buffer
\item Buffer drains to memory in background
\item Reduces stall impact (will discuss if time permits)

\paragraph{Modern Systems}

\begin{itemize}
\item Pure write-through rarely used alone
\item Too slow for general-purpose computing
\item Alternative: Write-back policy (next lecture)
\item Trade complexity for performance

\subsection{The Need for Alternative Write Policies}

\subsubsection{The Performance Problem}

\paragraph{Write-Heavy Programs}

Many programming patterns involve frequent writes:

\begin{itemize}
\item Array updates in loops
\item Data structure modifications
\item Counter increments
\item Accumulator updates
\item String/buffer operations

\textbf{Example Code:}

\begin{lstlisting}[language=c]
for (int i = 0; i < 1000; i++) {
    array[i] = compute(i);  // Store in every iteration
    sum += array[i];         // Read, accumulate, store
}
\end{verbatim}

\paragraph{With Write-Through}

\begin{itemize}
\item Loop iterations: 1000
\item Stores per iteration: 2 (array[i], sum)
\item Total stores: 2000
\item Cycles per store: 50 (memory write)
\item \textbf{Total stall cycles: 100,000!}
\item Versus computation cycles: Maybe 10,000
\item \textbf{Performance: 10$\times$ slower than necessary!}

\subsubsection{Pipeline Impact}

\begin{itemize}
\item Pipelining designed to execute 1 instruction/cycle (ideal)
\item Write-through: 50 cycles per store instruction
\item Pipeline utilization: ~2% (1/50)
\item Completely defeats pipelining benefits

\subsubsection{Comparison with Read Operations}

| Operation  | Time        | Frequency | Acceptability    |
| ---------- | ----------- | --------- | ---------------- |
| Read hit   | < 1 cycle   | Common    | Fast             |
| Read miss  | 50 cycles   | Rare      | Acceptable       |
| Write hit  | 50 cycles   | Frequent  | \textbf{Unacceptable} |
| Write miss | 100+ cycles | Rare      | Terrible         |

\subsubsection{The Contradiction}

\begin{itemize}
\item Spent lectures optimizing pipeline
\item Minimized hazards, used forwarding, prediction
\item Eliminated 1-2 cycle stalls
\item Now introducing 50+ cycle stalls on every write!
\item "Doesn't add up" - need better solution

\subsubsection{Question Raised}

\textbf{"What can we do to avoid this situation?"}

\textbf{Student Insight:}

> "We can write to memory only when we want to replace that cache block with different data"

\textbf{Instructor Response:}

> "Exactly! That becomes a different write policy."

\subsubsection{Teaser for Next Lecture}

\begin{itemize}
\item Alternative policy: \textbf{Write-Back}
\item Write to cache only, not memory immediately
\item Write to memory only when necessary
\item Much better performance
\item Added complexity in return
\item Will discuss in detail next class

\subsection{Lecture Conclusion}

\subsubsection{Topics Covered}

\paragraph{1. Complete Read Access Process}

\begin{itemize}
\item Index $\rightarrow$ Tag compare $\rightarrow$ Valid check $\rightarrow$ Hit/Miss
\item Parallel data extraction and word selection
\item Hit: Send data immediately
\item Miss: Fetch from memory, stall CPU

\paragraph{2. Read Miss Handling}

Six-step process:

\begin{enumerate}
\item Stall CPU
\item Request block from memory
\item Wait for response
\item Update cache entry (data, tag, valid)
\item Send data to CPU
\item Clear stall

\begin{itemize}
\item Miss penalty: 10-100+ cycles

\paragraph{3. Write Access Process}

\begin{itemize}
\item Similar to read: Index $\rightarrow$ Tag compare $\rightarrow$ Valid check
\item Difference: Must write data to cache
\item Use demultiplexer to direct data to correct word

\paragraph{4. Data Consistency Problem}

\begin{itemize}
\item Writing to cache creates inconsistency
\item Cache has new value, memory has old value
\item Need policy to maintain consistency

\paragraph{5. Write-Through Policy}

\begin{itemize}
\item Write to both cache and memory on every write
\item Advantages: Simple, consistent, safe
\item Disadvantages: Slow, excessive traffic, poor performance

\paragraph{6. Old Block Question Resolved}

\begin{itemize}
\item With write-through: Safe to discard
\item Memory has updated version
\item Can re-fetch if needed later

\paragraph{7. Parallel Write Optimization}

\begin{itemize}
\item Can overlap write and tag compare
\item Write-through makes this safe
\item Reduces hit latency

\paragraph{8. Performance Issues}

\begin{itemize}
\item Write-through too slow for write-intensive programs
\item Every write causes long stall
\item Need better policy

\subsubsection{Next Lecture Preview}

\textbf{Topics to Cover:}

\begin{itemize}
\item Write-Back policy (delayed writes)
\item Dirty bit concept
\item When to write back to memory
\item Performance improvements
\item Complexity tradeoffs
\item Block replacement with write-back
\item Comparison: Write-through vs. Write-back
\item Real-world cache designs

\textbf{Implementation Details:}

\begin{itemize}
\item Write buffer optimization for write-through
\item Handling dirty blocks on replacement
\item Write-back state machine
\item Performance analysis

\textbf{Advanced Topics (if time):}

\begin{itemize}
\item Write-allocate vs. no-write-allocate
\item Write-combining
\item Victim caches
\item Multi-level caches with different policies

\textbf{The Goal:}

\begin{itemize}
\item Understand tradeoffs between simplicity and performance
\item Choose appropriate policy for application
\item Design efficient cache systems

\textbf{Key Insight:}
Write-through sacrifices performance for simplicity. In modern systems, performance is critical, so more complex policies are necessary despite added complexity.

\subsection{Key Takeaways}

\begin{enumerate}
\item \textbf{Cache read hit} completes in single cycle—tag match and valid bit set indicate data available immediately from cache.

\begin{enumerate}
\item \textbf{Cache read miss} requires multiple cycles—must fetch entire block from main memory, update cache entry, set valid bit, then retry access.

\begin{enumerate}
\item \textbf{Cache controller} implements state machine—managing transitions between idle, compare tags, fetch block, and write cache states.

\begin{enumerate}
\item \textbf{Tag comparison} determines hit/miss—stored tag must match address tag AND valid bit must be set for successful hit.

\begin{enumerate}
\item \textbf{Block fetch} retrieves entire block from memory—exploiting spatial locality by bringing multiple words that will likely be accessed soon.

\begin{enumerate}
\item \textbf{Valid bit initialization} crucial at startup—all valid bits cleared to zero, preventing false hits on random cache data.

\begin{enumerate}
\item \textbf{Write operations} complicate cache design—must maintain consistency between cache and main memory through careful policy choices.

\begin{enumerate}
\item \textbf{Write-through policy} updates both cache and memory on every write—simple consistency but severe performance penalty.

\begin{enumerate}
\item \textbf{Write-through advantages}: Simple implementation, main memory always current, no dirty bit needed, straightforward crash recovery.

10. \textbf{Write-through disadvantages}: Every write causes slow memory access (~100 ns), dramatically reduces performance, wastes memory bandwidth.

11. \textbf{Write buffers} partially mitigate write-through penalty—CPU writes to buffer and continues, buffer writes to memory asynchronously.

12. \textbf{Write buffer depth} typically 4-8 entries—balances performance improvement against hardware cost and complexity.

13. \textbf{Write buffer full} forces CPU stall—occurs during write-intensive code sections, limiting write-through effectiveness.

14. \textbf{Write miss policies} determine cache behavior—write-allocate (fetch block first) versus no-write-allocate (write directly to memory).

15. \textbf{Write-allocate} exploits temporal locality—if just written location likely accessed again soon, fetching to cache improves future performance.

16. \textbf{No-write-allocate} avoids fetch overhead—appropriate when written locations unlikely to be accessed soon.

17. \textbf{Policy combinations} affect overall performance—write-through typically paired with no-write-allocate for consistency.

18. \textbf{Cache consistency} means cache and memory agree on data values—critical correctness requirement across all cache operations.

19. \textbf{Performance impact} of write policies substantial—write-through can increase memory traffic by 15-20% in typical programs.

20. \textbf{Write-back policy} introduced as superior alternative—defers memory writes until block eviction, dramatically reducing memory traffic.

\subsection{Summary}

Detailed examination of cache memory operations reveals the sophisticated control logic required to manage read and write accesses while maintaining data consistency between cache and main memory. Read operations follow straightforward paths: hits deliver data in single cycle via tag comparison confirming both tag match and valid bit set, while misses trigger multi-cycle sequences fetching entire blocks from main memory, updating cache entries, setting valid bits, and retrying accesses. The cache controller implements these sequences through state machine logic managing transitions between idle, tag comparison, block fetching, and cache writing states. Write operations introduce significant complexity and performance implications through policy choices determining how cache and memory stay synchronized. Write-through policy, updating both cache and memory on every write, offers simplicity and guaranteed consistency—main memory always reflects current data state, enabling straightforward crash recovery and multi-processor coherence. However, write-through's performance penalty proves severe: every write operation incurs ~100 nanosecond memory access delay, effectively eliminating cache benefit for write-heavy code sections and wasting substantial memory bandwidth on updates. Write buffers provide partial mitigation by decoupling CPU from memory write delays, allowing processors to write to small hardware queues and continue execution while buffer contents asynchronously propagate to main memory. Typical write buffers holding 4-8 entries balance performance improvement against hardware cost, though write-intensive code can still fill buffers and force CPU stalls. Write miss policies—write-allocate (fetch block before writing) versus no-write-allocate (write directly to memory)—represent additional design choices affecting performance based on program access patterns. Write-allocate exploits temporal locality, benefiting code that writes then soon reads same locations, while no-write-allocate avoids fetch overhead for write-once scenarios. Write-through typically pairs with no-write-allocate for policy consistency. The fundamental limitation—that write-through forces memory access on every write regardless of whether data will be accessed again—motivates write-back policies introduced in subsequent lectures, which defer memory writes until block eviction and thereby dramatically reduce memory traffic. Understanding these operational details and policy tradeoffs proves essential for appreciating how real cache implementations balance performance, complexity, consistency, and correctness requirements in practical computer systems.

% % Lecture 16: Associative Cache Control
% CO224 - Computer Architecture

\chapter{Associative Cache Control}

\section{Introduction}

% TODO: Add content from markdown

\section{1. Recap: Associativity Comparison Results}

% TODO: Add content from markdown

\section{2. Cache Configuration Parameters}

% TODO: Add content from markdown

\section{3. Improving Cache Performance - Comprehensive Review}

% TODO: Add content from markdown

\section{4. Hit Rate Improvement}

% TODO: Add content from markdown

\section{5. Hit Latency Optimization}

% TODO: Add content from markdown

\section{6. Miss Penalty Improvement}

% TODO: Add content from markdown

\section{7. Cache Hierarchy (Multi-Level Caches)}

% TODO: Add content from markdown

\section{8. Optimization Strategies for Multi-Level Caches}

% TODO: Add content from markdown

\section{9. L1 Cache Optimization - Optimize for Hit Latency}

% TODO: Add content from markdown

\section{10. L2 Cache Optimization - Optimize for Hit Rate}

% TODO: Add content from markdown

\section{11. Associativity Comparison}

% TODO: Add content from markdown

\section{12. Physical Implementation of Cache Hierarchy}

% TODO: Add content from markdown

\section{13. Real World Example: Intel Skylake Architecture}

% TODO: Add content from markdown

\section{14. Recommendations for Further Study}

% TODO: Add content from markdown

\section{15. Next Topics}

% TODO: Add content from markdown

\section{Key Takeaways}

% TODO: Add content from markdown

\section{Summary}

% TODO: Add content from markdown

% Note: Convert markdown content manually for best results
% Remember to:
% - Replace markdown images with \includegraphics
% - Convert code blocks to \begin{lstlisting}
% - Convert lists to \begin{itemize} or \begin{enumerate}
% - Convert bold/italic with \textbf{} and \textit{}

% % Lecture 17: Multi-Level Caching
% CO224 - Computer Architecture

\chapter{Multi-Level Caching}

\section{Introduction}

% TODO: Add content from markdown

\section{1. Introduction to Virtual Memory}

% TODO: Add content from markdown

\section{2. CPU Word Size and Address Space}

% TODO: Add content from markdown

\section{3. Virtual vs Physical Addresses}

% TODO: Add content from markdown

\section{4. Memory Hierarchy with Virtual Memory}

% TODO: Add content from markdown

\section{5. Terminology}

% TODO: Add content from markdown

\section{6. Access Latencies}

% TODO: Add content from markdown

\section{7. Virtual and Physical Address Structure}

% TODO: Add content from markdown

\section{8. Supporting Multiple Programs}

% TODO: Add content from markdown

\section{9. Page Table}

% TODO: Add content from markdown

\section{10. Address Translation Process}

% TODO: Add content from markdown

\section{11. Page Table Size Calculation}

% TODO: Add content from markdown

\section{12. Write Policy for Virtual Memory}

% TODO: Add content from markdown

\section{13. Placement Policy}

% TODO: Add content from markdown

\section{14. Page Fault Handling}

% TODO: Add content from markdown

\section{15. Translation Lookaside Buffer (TLB)}

% TODO: Add content from markdown

\section{16. Complete Memory Access with TLB}

% TODO: Add content from markdown

\section{17. Approach 1: Virtually Addressed Cache}

% TODO: Add content from markdown

\section{18. Approach 2: Physically Addressed Cache}

% TODO: Add content from markdown

\section{Key Takeaways}

% TODO: Add content from markdown

\section{Summary}

% TODO: Add content from markdown

% Note: Convert markdown content manually for best results
% Remember to:
% - Replace markdown images with \includegraphics
% - Convert code blocks to \begin{lstlisting}
% - Convert lists to \begin{itemize} or \begin{enumerate}
% - Convert bold/italic with \textbf{} and \textit{}

% \input{lecture-18}

% Chapter 5: Advanced Topics
% \chapter{Advanced Topics}

% % Lecture 19: Multiprocessors
% CO224 - Computer Architecture

\chapter{Multiprocessors}

\section{Introduction}

% TODO: Add content from markdown

\section{1. Introduction to Multiprocessors}

% TODO: Add content from markdown

\section{2. Performance Evolution Background}

% TODO: Add content from markdown

\section{3. Multiprocessor Approach}

% TODO: Add content from markdown

\section{4. Shared Memory Multiprocessors (SMM)}

% TODO: Add content from markdown

\section{5. Memory Contention Problem}

% TODO: Add content from markdown

\section{6. Uniform Memory Access (UMA)}

% TODO: Add content from markdown

\section{7. Solution to Contention: Caches}

% TODO: Add content from markdown

\section{8. Cache Coherence Problem}

% TODO: Add content from markdown

\section{9. Bus Snooping}

% TODO: Add content from markdown

\section{10. Write Invalidate Protocol}

% TODO: Add content from markdown

\section{11. Write Update Protocol}

% TODO: Add content from markdown

\section{12. Real Protocol Implementations}

% TODO: Add content from markdown

\section{13. MESI Protocol Details}

% TODO: Add content from markdown

\section{14. MESI Protocol State Transitions}

% TODO: Add content from markdown

\section{15. Scalability of UMA Systems}

% TODO: Add content from markdown

\section{16. Non-Uniform Memory Access (NUMA)}

% TODO: Add content from markdown

\section{17. Two Types of NUMA}

% TODO: Add content from markdown

\section{18. Directory-Based Cache Coherence}

% TODO: Add content from markdown

\section{Key Takeaways}

% TODO: Add content from markdown

\section{Summary}

% TODO: Add content from markdown

% Note: Convert markdown content manually for best results
% Remember to:
% - Replace markdown images with \includegraphics
% - Convert code blocks to \begin{lstlisting}
% - Convert lists to \begin{itemize} or \begin{enumerate}
% - Convert bold/italic with \textbf{} and \textit{}

% \section{Lecture 20: Storage and Input/Output Systems}

\emph{By Dr. Swarnalatha Radhakrishnan}

\subsection{Introduction}

This lecture completes our exploration of computer architecture by examining storage devices and input/output (I/O) systems that enable computers to interact with external devices and provide persistent data storage beyond volatile main memory. We explore storage technologies from mechanical magnetic disks to solid-state flash memory, understanding their performance characteristics, reliability metrics, and cost tradeoffs. The lecture covers I/O communication methods including polling, interrupts, and direct memory access (DMA), analyzes RAID configurations that improve both performance and dependability, and examines how storage systems connect to processors through memory-mapped I/O or dedicated I/O instructions. Understanding these peripheral systems reveals how complete computer systems integrate computation, memory, and external interaction into cohesive platforms.

\subsection{I/O Device Characteristics}

I/O devices can be characterized by three fundamental factors:

\subsubsection{Behavior}

\textbf{Input Devices}:

\begin{itemize}
\item Provide data to system
\item Examples: keyboards, mice, sensors

\textbf{Output Devices}:

\begin{itemize}
\item Receive data from system
\item Examples: displays, printers, speakers

\textbf{Storage Devices}:

\begin{itemize}
\item Store and retrieve data
\item Examples: disks, flash drives

\subsubsection{Partner}

\textbf{Human Devices}:

\begin{itemize}
\item Communicate with humans
\item Examples: keyboards, displays, audio

\textbf{Machine Devices}:

\begin{itemize}
\item Communicate with other machines
\item Examples: networks, controllers

\subsubsection{Data Rate}

\begin{itemize}
\item Measured in bytes per second or transfers per second
\item Wide variation across device types
\item Affects system design and communication methods

\subsection{I/O Bus Connections}

\subsubsection{Simplified System Architecture}

\paragraph{Components}

\begin{itemize}
\item \textbf{Processor (CPU)}
\item \textbf{Cache}
\item \textbf{Memory I/O Interconnect (Bus)}
\item \textbf{Main Memory}
\item \textbf{Multiple I/O Controllers}
\item \textbf{Various I/O Devices}

\paragraph{Bus Structure}

\begin{itemize}
\item Processor and cache connected to bus
\item Main memory connected to bus
\item I/O controllers connected to bus
\item Each controller manages specific devices

\paragraph{Connections}

\begin{itemize}
\item Processor receives interrupts from bus/devices
\item \textbf{I/O Controller 1}: Connected to disk
\item \textbf{I/O Controller 2}: Connected to graphic output
\item \textbf{I/O Controller 3}: Connected to network channel

Multiple controllers allow parallel device operation while sharing common interconnect.

\subsection{Dependability}

Critical for I/O systems, especially storage devices.

\subsubsection{Why Dependability Matters}

\begin{itemize}
\item Storage devices hold data that must be reliable
\item Users depend on devices being available
\item Data loss is unacceptable
\item Systems must continue functioning despite component failures

\subsubsection{Dependability is Particularly Important For}

\begin{itemize}
\item Storage devices (data integrity)
\item Critical systems (servers, embedded systems)
\item Systems with high availability requirements

\subsection{Service States}

\subsubsection{Two Primary States}

\paragraph{1. Service Accomplishment State}

\begin{itemize}
\item Device is working correctly
\item Providing expected service
\item Normal operational state

\paragraph{2. Service Interruption State}

\begin{itemize}
\item Device has failed
\item Not providing service
\item Requires repair/restoration

\subsubsection{State Transitions}

\begin{itemize}
\item \textbf{From Service Accomplishment to Service Interruption}: Due to failure
\item \textbf{From Service Interruption to Service Accomplishment}: After restoration/repair

\subsection{Fault Terminology}

\subsubsection{Fault Definition}

\textbf{Characteristics}:

\begin{itemize}
\item Failure of a component
\item May or may not affect the system
\item May or may not lead to system failure
\item System can continue running with faulty component
\item May produce correct or wrong output

\subsubsection{Distinction}

\begin{itemize}
\item \textbf{Component failure $\neq$ System failure}
\item Fault tolerance allows operation despite faults

\subsection{Dependability Measures}

\subsubsection{Key Metrics}

\paragraph{1. MTTF (Mean Time To Failure)}

\textbf{Definition}:

\begin{itemize}
\item Reliability measure
\item Average time device operates before failing
\item Measures how long system stays in Service Accomplishment state
\item Higher MTTF = more reliable

\paragraph{2. MTTR (Mean Time To Repair)}

\textbf{Definition}:

\begin{itemize}
\item Service interruption measure
\item Average time to restore service after failure
\item How long device stays in Service Interruption state
\item Lower MTTR = faster recovery

\paragraph{3. MTBF (Mean Time Between Failures)}

\textbf{Formula}:

MTBF = MTTF + MTTR

\textbf{Definition}:

\begin{itemize}
\item Complete cycle: operation + repair
\item Time from one failure to next failure
\item Includes both operational and repair time

\paragraph{4. Availability}

\textbf{Formula}:

Availability = MTTF / (MTTF + MTTR)

\textbf{Definition}:

\begin{itemize}
\item Proportion of time machine is available
\item Ratio of operational time to total time
\item Expressed as percentage or decimal

\subsection{Improving Availability}

\subsubsection{Two Approaches}

\begin{itemize}
\item MTTF
\item MTTR

\subsection{Increase MTTF (Mean Time To Failure)}

\paragraph{a) Fault Avoidance}

\textbf{Methods}:

\begin{itemize}
\item Prevent faults before they occur
\item Better design and manufacturing
\item Quality components
\item Proper operating conditions

\paragraph{b) Fault Tolerance}

\textbf{Methods}:

\begin{itemize}
\item Design system to withstand faults
\item Redundancy (duplicate components)
\item Error correction mechanisms
\item Graceful degradation

\paragraph{c) Fault Forecasting}

\textbf{Methods}:

\begin{itemize}
\item Predict when faults will occur
\item Preventive maintenance
\item Monitor component health
\item Replace before failure

\subsection{Reduce MTTR (Mean Time To Repair)}

\subsubsection{Methods}

\begin{itemize}
\item Improve tools and processes for diagnosis
\item Better diagnostic capabilities
\item Easier repair procedures
\item Quick replacement mechanisms
\item Automated recovery systems
\item Skilled maintenance personnel

\subsubsection{Example Problems}

\begin{itemize}
\item Book provides examples with specific MTTF and MTTR values
\item Calculate availability
\item Analyze improvement strategies
\item Students should practice these calculations

\subsection{Magnetic Disk Storage}

Traditional secondary storage technology using magnetic recording.

\subsubsection{Physical Structure}

\paragraph{Disk Shape}

\begin{itemize}
\item Circular/round shape
\item Platter rotates on spindle

\paragraph{Tracks}

\begin{itemize}
\item Concentric circles on disk surface
\item From periphery (outer edge) to center
\item Multiple tracks like ribbons arranged concentrically
\item Similar to running tracks in sports (Olympics)

\paragraph{Sectors}

\begin{itemize}
\item Tracks divided by radial lines (from center to periphery)
\item Cross-sectional cuts across tracks
\item Portion between two separation lines = one sector
\item Smallest addressable unit on disk

\subsubsection{Sector Contents}

\begin{itemize}
\item \textbf{Sector ID} (identification)
\item \textbf{Data} (512 bytes to 4096 bytes typical)
\item \textbf{Error Correcting Code (ECC)}
\item Hides defects
\item Corrects recording errors
\item \textbf{Gaps} between sectors (unused spaces)

\subsection{Disk Access Process}

\subsubsection{Access Components and Timing}

\paragraph{1. Queuing Delay}

\begin{itemize}
\item If other accesses are pending
\item Wait for previous operations to complete
\item Managed by disk controller

\paragraph{2. Seek Time}

\begin{itemize}
\item Moving head to correct track
\item Head positioned on right sector
\item Physical movement of read/write head
\item Head placed diagonally on disc
\item Time to "seek" the target sector
\item Typically several milliseconds

\paragraph{3. Rotational Latency}

\begin{itemize}
\item Rotating disk to position correct sector under head
\item Disk spins to align sector with head
\item Choose closest direction (shortest rotation)
\item Sectors arranged diagonally on disk
\item Multiple sectors per track
\item Can rotate either direction (clockwise or counterclockwise)

\paragraph{4. Transfer Time}

\begin{itemize}
\item Actual data read/write
\item Depends on sector size and transfer rate
\item Usually small compared to seek and rotation

\paragraph{5. Controller Overhead}

\begin{itemize}
\item Processing by disk controller
\item Command interpretation
\item Error checking
\item Generally small (fraction of millisecond)

\subsubsection{Access Coordination}

\begin{itemize}
\item Processor initiates access
\item Memory Management Unit (MMU) handles translation
\item Involves both hardware and operating system
\item Reading page from disk to memory: millions of cycles
\item Much slower than memory access

\subsection{Disk Access Example Calculation}

\subsubsection{Given Parameters}

\begin{itemize}
\item \textbf{Sector size}: 512 bytes
\item \textbf{Rotational speed}: 15,000 RPM (rotations per minute)
\item \textbf{Seek time}: 4 milliseconds
\item \textbf{Transfer rate}: 100 MB/s
\item \textbf{Controller overhead}: 0.2 milliseconds
\item Assume idle disk (no queuing)

\subsubsection{Average Read Time Calculation}

\paragraph{1. Seek Time}

\begin{itemize}
\item 4 ms (given)

\paragraph{2. Rotational Latency}

\begin{itemize}
\item Average = Half rotation time
\item Full rotation = 60 seconds / 15,000 RPM = 4 ms
\item Average = 4 ms / 2 = \textbf{2 ms}
\item Why half? Can choose closest direction

\paragraph{3. Transfer Time}

\begin{itemize}
\item Size / Rate = 512 bytes / 100 MB/s
\item = \textbf{0.005 ms}

\paragraph{4. Controller Delay}

\begin{itemize}
\item 0.2 ms (given)

\subsubsection{Total Average Read Time}

Total = 4 + 2 + 0.005 + 0.2 = 6.2 milliseconds

\subsubsection{Real Case Variation}

\begin{itemize}
\item Actual average seek time might be 1 ms (not 4 ms)
\item Depends on:
\item Which sector being accessed
\item Current head position
\item Distance head must travel
\item With 1 ms seek: Total = \textbf{3.2 ms}
\item Significant variation based on access patterns

\subsubsection{Additional Examples}

\begin{itemize}
\item Book provides more practice problems
\item Students should try different scenarios
\item Understand impact of each component on total time

\subsection{Flash Storage}

Modern non-volatile semiconductor storage technology.

\subsubsection{Characteristics}

\paragraph{Advantages}

\begin{itemize}
\item Non-volatile (retains data without power)
\item 1000x faster than magnetic disk
\item Smaller physical size
\item Lower power consumption
\item More robust (no moving parts)
\item Can be carried around easily
\item Shock resistant

\paragraph{Disadvantages}

\begin{itemize}
\item More expensive than magnetic disk
\item Limited write cycles (wears out over time)
\item Technology cost higher

\subsection{Types of Flash Storage}

\subsubsection{NOR Flash}

\paragraph{Structure}

\begin{itemize}
\item Bit cell like NOR gate
\item Random read/write access
\item Can access individual bytes

\paragraph{Characteristics}

\begin{itemize}
\item Byte-level access
\item Faster read access
\item More expensive

\paragraph{Applications}

\begin{itemize}
\item Instruction memory in embedded systems
\item Code storage
\item Execute-in-place applications

\subsubsection{NAND Flash}

\paragraph{Structure}

\begin{itemize}
\item Bit cell like NAND gate
\item Block-at-a-time access
\item Cannot access individual bytes directly

\paragraph{Characteristics}

\begin{itemize}
\item Denser (more storage per area)
\item Block-level access
\item Reading and writing done in blocks
\item Cheaper per GB

\paragraph{Applications}

\begin{itemize}
\item USB keys/drives
\item Media storage (photos, videos)
\item Solid-state drives (SSDs)
\item Memory cards

\textbf{Note}: Values in lecture slides may be outdated as flash storage technology rapidly evolves.

\subsection{Memory-Mapped I/O}

Method of accessing I/O devices using memory addresses.

\subsubsection{Concept}

\begin{itemize}
\item Reserve some address space for I/O devices
\item I/O device registers appear as memory locations
\item Same address space as memory
\item Address decoder distinguishes between memory and I/O

\subsubsection{Example with 8 Address Lines}

\begin{itemize}
\item \textbf{Total addressable locations}: 256 (2^8)
\item \textbf{Reserve 128 locations for memory}
\item \textbf{Reserve 128 locations for I/O devices}
\item Same load/store instructions access both

\subsubsection{Access Mechanism}

\begin{itemize}
\item Use load/store instructions for both memory and I/O
\item Operating system controls access
\item Uses address translation mechanism
\item Can make I/O addresses accessible only to kernel
\item Protection mechanism prevents user programs from direct access

\subsubsection{Advantages}

\begin{itemize}
\item Unified programming model
\item Same instructions for memory and I/O
\item Simpler instruction set

\subsubsection{Disadvantages}

\begin{itemize}
\item Reduces available memory address space
\item Must reserve addresses for I/O

\subsection{I/O Instructions}

Alternative to memory-mapped I/O: separate I/O instructions.

\subsubsection{Characteristics}

\begin{itemize}
\item Separate instructions specifically for I/O operations
\item Distinct from load/store (memory) instructions
\item Can duplicate addresses:
\item Same address can refer to memory location
\item Same address can refer to I/O device
\item Instruction type determines which is accessed

\subsubsection{Access Control}

\begin{itemize}
\item I/O instructions can only execute in kernel mode
\item User programs cannot directly access I/O
\item Protection mechanism
\item Operating system mediates I/O access

\subsubsection{Example Architecture}

\begin{itemize}
\item \textbf{x86 (Intel/AMD processors)}
\item Has special IN and OUT instructions for I/O
\item Separate I/O address space

\subsubsection{Advantages}

\begin{itemize}
\item Full memory address space available
\item No address space conflict
\item Clear distinction between memory and I/O

\subsubsection{Disadvantages}

\begin{itemize}
\item More complex instruction set
\item Additional instructions needed

\subsection{Polling}

Method for processor to communicate with I/O devices.

\subsubsection{How Polling Works}

\paragraph{1. Periodically Check I/O Status Register}

\begin{itemize}
\item Processor repeatedly reads device status
\item Check if device is ready
\item Continuous monitoring in loop

\paragraph{2. If Device Ready}

\begin{itemize}
\item Perform requested operation
\item Read data or write data
\item Continue with next task

\paragraph{3. If Error Detected}

\begin{itemize}
\item Take appropriate action
\item Error handling
\item Retry or report error

\subsubsection{Characteristics}

\paragraph{When Used}

\begin{itemize}
\item Small or low-performance systems
\item Real-time embedded systems
\item Simple applications

\paragraph{Advantages}

\textbf{Predictable Timing}:

\begin{itemize}
\item Know exactly when device checked
\item Deterministic behavior
\item Important for real-time systems

\textbf{Low Hardware Cost}:

\begin{itemize}
\item Software handles communication
\item No additional hardware needed
\item Simple implementation

\paragraph{Disadvantages}

\textbf{Wastes CPU Time}:

\begin{itemize}
\item CPU continuously loops checking device
\item Can't do other work while polling
\item Inefficient for high-performance systems

\textbf{Not Suitable for Complex Systems}:

\begin{itemize}
\item Multiple devices difficult to manage
\item CPU time wasted on idle devices

\subsubsection{Programming Model}

\begin{itemize}
\item Can write program to:
\item Read status bit from device
\item Check if device free
\item Make decisions based on status
\item Simple control flow

\subsection{Interrupts}

Alternative to polling: device-initiated communication.

\subsubsection{How Interrupts Work}

\paragraph{1. Device Initialization}

\begin{itemize}
\item Device sends signal/request to processor
\item Request for service
\item Happens when device ready or error occurs

\paragraph{2. Controller Interrupts CPU}

\begin{itemize}
\item Device controller signals processor
\item Processor stops current work
\item Handles interrupt

\paragraph{3. Handler Execution}

\begin{itemize}
\item Special interrupt handler routine runs
\item Services device request
\item Returns to original program

\subsubsection{Characteristics}

\paragraph{Asynchronous}

\begin{itemize}
\item Not synchronized to instruction execution
\item Unlike exceptions (which are synchronous)
\item Can occur between any two instructions
\item Handler invoked between instructions

\paragraph{Fast Identification}

\begin{itemize}
\item Interrupt often identifies device
\item Know which device needs service
\item Can be handled quickly

\paragraph{Priority System}

\begin{itemize}
\item Not all devices have same urgency
\item Devices categorized by priority levels
\item Devices needing urgent attention get higher priority
\item High-priority interrupts can preempt low-priority handlers

\subsubsection{Advantages}

\textbf{Efficient CPU Use}:

\begin{itemize}
\item No wasted time polling
\item CPU does other work until interrupt

\textbf{Good for Multiple Devices}:

\begin{itemize}
\item Each device interrupts when ready
\item No continuous checking needed

\textbf{Responsive}:

\begin{itemize}
\item Quick response to device events

\subsubsection{Disadvantages}

\textbf{More Complex Hardware}:

\begin{itemize}
\item Interrupt controller needed
\item Priority management

\textbf{Context Switching Overhead}:

\begin{itemize}
\item Save/restore processor state
\item Handler invocation takes time

\subsubsection{Execution Model}

\begin{itemize}
\item Main program running
\item Instruction completes
\item Interrupt checked
\item If interrupt pending:
\item Current state saved
\item Interrupt handler runs
\item State restored
\item Resume main program at next instruction

\subsection{I/O Data Transfer Methods}

Three approaches for transferring data between memory and I/O:

\subsection{Polling-Driven I/O}

\subsubsection{Process}

\begin{itemize}
\item CPU polls device repeatedly
\item When ready, CPU transfers data
\item CPU moves data between memory and I/O registers

\subsubsection{Issues}

\begin{itemize}
\item Time consuming
\item CPU fully involved in transfer
\item Inefficient for high-speed devices
\item Wastes CPU cycles

\subsection{Interrupt-Driven I/O}

\subsubsection{Process}

\begin{itemize}
\item Device interrupts when ready
\item CPU services interrupt
\item CPU transfers data between memory and I/O registers

\subsubsection{Issues}

\begin{itemize}
\item Still CPU-intensive for data transfer
\item CPU must move every byte
\item Better than polling but still inefficient for bulk transfers

\subsection{Direct Memory Access (DMA)}

\subsubsection{Process}

\textbf{Setup}:

\begin{itemize}
\item DMA controller handles transfer
\item Removes CPU from data movement
\item Processor hands off transfer job to DMA controller
\item DMA controller transfers data autonomously

\subsubsection{DMA Operation}

\textbf{CPU Provides}:

\begin{itemize}
\item Starting address in memory
\item Transfer size
\item Direction (memory$\rightarrow$device or device$\rightarrow$memory)

\textbf{DMA Controller}:

\begin{itemize}
\item Transfers data independently
\item Operates in parallel with CPU
\item No CPU intervention during transfer

\textbf{Controller Interrupts CPU On}:

\begin{itemize}
\item Completion of transfer
\item Error occurrence

\subsubsection{Advantages}

\begin{itemize}
\item CPU free to do other work
\item Efficient bulk data transfers
\item Essential for high-speed devices
\item Reduces CPU overhead significantly

\subsubsection{When Used}

\begin{itemize}
\item High-speed devices (disks, network)
\item Large data transfers
\item When CPU time is valuable

\subsubsection{Comparison}

\begin{itemize}
\item \textbf{Polling}: Simple, predictable, inefficient
\item \textbf{Interrupts}: Responsive, better than polling, CPU still involved in transfer
\item \textbf{DMA}: Most efficient, essential for high-performance I/O

\subsection{RAID (Redundant Array of Independent Disks)}

Technology to improve storage performance and dependability.

\subsubsection{Purpose}

\begin{itemize}
\item Improve performance through parallelism
\item Improve dependability through redundancy
\item Use multiple disks together as single logical unit

\subsubsection{Benefits}

\paragraph{Performance Improvement}

\begin{itemize}
\item Parallel access to multiple disks
\item Higher throughput
\item Faster data access

\paragraph{Dependability Improvement}

\begin{itemize}
\item Redundancy protects against disk failure
\item Data not lost if one disk fails
\item Improved reliability

\subsection{Key Takeaways}

\begin{enumerate}
\item I/O systems connect computers to external devices and storage
\item Dependability is critical for storage systems
\item MTTF, MTTR, and availability are key metrics
\item Magnetic disks use mechanical components with millisecond access times
\item Flash storage is faster but more expensive than magnetic storage
\item Memory-mapped I/O and separate I/O instructions are two access methods
\item Polling is simple but inefficient
\item Interrupts improve CPU efficiency
\item DMA is essential for high-speed bulk data transfers
\end{enumerate}

10. RAID improves both performance and reliability

\subsection{Summary}

This concludes the processor and memory sections of the lecture, covering the complete spectrum from CPU design through memory hierarchy to I/O systems. We have explored how computers are designed from the ground up, from basic arithmetic operations through pipelined execution, memory hierarchies, multiprocessor systems, and finally to storage and I/O mechanisms that enable computers to interact with the external world.


\end{document}
